\documentclass{templateNote}
\usepackage{tcolorbox}
\usepackage{hyperref}
% \usepackage{pgfplots}
\usepackage{amsmath}
\usepackage{soul}
\usepackage{amssymb}
\usepackage{parskip} % Add this line

\begin{document}
% \imagenlogoU{img/logoNGMFormal_sinF.png}
% \linklogoU{https://github.com/NicoxlkboUni} 
% \linkQRDoc{https://github.com/NicoxlkboUni} %TODO: Eliminar este qr
\titulo{Solución listado 5}
\asignatura{Calculo 1}

% \vDoc{1.0.0}
\portada

\margenes % Crear márgenes

\subsection*{\textbf{Ejercicio 1}:} 
Considere los conjuntos $A = \{x \in \mathbb{N}: 1 \leq x \leq 5\}$, $B = \{x \in \mathbb{N}: 50 \leq x^{2} \leq 100\}$
y $C = \{9,10,11\}$. Calcular: 


Para resolver estos problemas, primero definamos los conjuntos $A$, $B$ y $C$: \\
$A=\{1,2,3,4,5\}$ \\
$B=\{8,9,10\} \leftarrow$ Nota: $8^{2}=64$, $9^{2}=81$, $10^{2}=100$ ya que solo toma en cuenta los valores de $x$ que cumplen con la condición $50 \leq x^{2} \leq 100$ \\
$C=\{9,10,11\}$

\begin{itemize}
    \item \textbf{a) $A \times B$ y $B \times A$. Compare ambos resultados.}
\end{itemize}

Dado que \( A \) y \( B \) son conjuntos finitos, podemos enumerar todos sus elementos y luego encontrar sus productos cartesianos.

Para \( A \times B \):
\[ A \times B = \{(a, b) : a \in A, b \in B\} \]
Y para \( B \times A \):
\[ B \times A = \{(b, a) : b \in B, a \in A\} \]
Vamos a encontrar los elementos de \( A \times B \) y \( B \times A \) y luego comparar los resultados.

1. \textbf{Cálculo de \( A \times B \)}:

\[ A = \{1, 2, 3, 4, 5\} \]
\[ B = \{8, 9, 10\} \]

Entonces,
\[ A \times B = \begin{aligned}
    &\{(1, 8), (1, 9), (1, 10), \\
    &(2, 8), (2, 9), (2, 10), \\
    &(3, 8), (3, 9), (3, 10), \\
    &(4, 8), (4, 9), (4, 10), \\
    &(5, 8), (5, 9), (5, 10)\}
    \end{aligned} \]
    
2. \textbf{Cálculo de \( B \times A \)}:

\[ B = \{8, 9, 10\} \]
\[ A = \{1, 2, 3, 4, 5\} \]

Entonces,

\[ B \times A = \begin{aligned}
    \{& (8, 1), (8, 2), (8, 3), (8, 4), (8, 5), \\
    & (9, 1), (9, 2), (9, 3), (9, 4), (9, 5), \\
    & (10, 1), (10, 2), (10, 3), (10, 4), (10, 5)\}
    \end{aligned} \]
Comparando ambos resultados, notamos que los pares ordenados en \( A \times B \) son los mismos que en \( B \times A \), solo que en orden inverso. Esto es porque el producto cartesiano es una operación conmutativa.

\begin{itemize}
    \item \textbf{b) \(A \times (B \cup C)\) y \((A \times B) \cup (A \times C)\):}
\end{itemize}

\[
\begin{aligned}
B \cup C &= \{8, 9, 10\} \cup \{9, 10, 11\} \\
&= \{8, 9, 10, 11\}
\end{aligned}
\]

\[ A \times (B \cup C) = \{(1, 8), (1, 9), (1, 10), (1, 11), ..., (5, 11)\} \]

Para \( (A \times B) \cup (A \times C) \):

\[ A \times B = \{(1, 8), (1, 9), (1, 10), ..., (5, 10)\} \]
\[ A \times C = \{(1, 9), (1, 10), (1, 11), ..., (5, 11)\} \]
\[ (A \times B) \cup (A \times C) = \{(1, 8), (1, 9), (1, 10), ..., (5, 11)\} \]

Ambos resultados son iguales.

\begin{itemize}
    \item \textbf{c) \(A \times (B \cap C)\) y \((A \times B) \cap (A \times C)\):}
\end{itemize}

Para \( B \cap C \):

\[ B \cap C = \{9, 10\} \]

\[ A \times (B \cap C) = \{(1, 9), (1, 10), ..., (5, 10)\} \]

Para \( (A \times B) \cap (A \times C) \):

\[ (A \times B) \cap (A \times C) = \{(1, 9), (1, 10), ..., (5, 10)\} \]

Ambos resultados son iguales.

\begin{itemize}
    \item \textbf{d) \(A \times B \times C\):}
\end{itemize}

\[ A \times B \times C = \{(1, 8, 9), (1, 8, 10), ..., (5, 10, 11)\} \]

\begin{itemize}
    \item \textbf{e) \(P(A \times B)\):}
\end{itemize}

Si \(A\) tiene 5 elementos y \(B\) tiene 3 elementos, entonces \(A \times B\) tiene \(5 \times 3 = 15\) elementos. Por lo tanto, \(P(A \times B)\) tiene \(2^{15}\) elementos.
Demuestre las siguientes propiedades del producto cartesiano:

\subsection*{\textbf{Ejercicio 2}:} 
\begin{itemize}
    \item \textbf{a) $(A \times (B \cap C)) = (A \times B) \cap (A \times C)$}
\end{itemize}

Para demostrar esto, primero recordemos la definición de los conjuntos involucrados:

$(A \times (B \cap C))$ representa el conjunto de todos los pares ordenados ((a, b)) donde $(a \in A)$ y (b) está en la intersección de los conjuntos (B) y (C).
$((A \times B) \cap (A \times C))$ representa el conjunto de todos los pares ordenados ((a, b)) que están en $(A \times B)$ y también en $(A \times C)$.
Ahora, procedamos con la demostración:

Tomemos un elemento $(a, b)$ en $A \times (B \cap C)$. Esto significa que $a \in A$ y $b \in B \cap C$.
Como $b \in B \cap C$, tenemos $b \in B$ y $b \in C$.
Por lo tanto, $(a, b) \in A \times B$ y $(a, b) \in A \times C$.
Esto implica que $(a, b) \in (A \times B) \cap (A \times C)$.
Hemos demostrado que cualquier elemento en $A \times (B \cap C)$ también está en $(A \times B) \cap (A \times C)$. Ahora, debemos demostrar la otra dirección:

Tomemos un elemento $(a, b)$ en $(A \times B) \cap (A \times C)$. Esto significa que $(a, b) \in A \times B$ y $(a, b) \in A \times C$.
Esto implica que $a \in A$, $b \in B$, y también $b \in C$.
Por lo tanto, $b \in B \cap C$.
Así que $(a, b) \in A \times (B \cap C)$.
Hemos demostrado que cualquier elemento en $(A \times B) \cap (A \times C)$ también está en $A \times (B \cap C)$.

Matemáticamente:

$ A \times (B \cap C) = \{ (a, b) \ | \ a \in A, \ b \in (B \cap C) \} $

$ \Rightarrow A \times (B \cap C) = \{ (a, b) \ | \ a \in A, \ b \in B \land b \in C \} $

$ \Rightarrow A \times (B \cap C) = \{ (a, b) \ | \ (a \in A \land b \in B) \land (a \in A \land b \in C) \} $

$ \Rightarrow A \times (B \cap C) = \{ (a, b) \ | \ (a \in A \land b \in B) \land (a \in A \land b \in C) \} $

$ \Rightarrow A \times (B \cap C) = \{ (a, b) \ | \ (a \in A \land b \in B) \} \cap \{ (a, c) \ | \ (a \in A \land c \in C) \} $

$ \Rightarrow A \times (B \cap C) = (A \times B) \cap (A \times C) $

En resumen, hemos demostrado que $A \times (B \cap C) = (A \times B) \cap (A \times C)$.

\begin{itemize}
    \item \textbf{b) $A \times (B - C) = (A \times B) - (A \times C)$}
\end{itemize}

Para demostrar esto, utilizaremos la definición de los conjuntos involucrados:

$A \times (B - C)$ representa el conjunto de todos los pares ordenados $(a, b)$ donde $a \in A$ y $b$ está en la diferencia de los conjuntos $B$ y $C$.
$(A \times B) - (A \times C)$ representa el conjunto de todos los pares ordenados $(a, b)$ que están en $A \times B$ pero no en $A \times C$.
La demostración sigue un razonamiento similar al caso anterior.

Matemáticamente:

$A \times (B - C) = \{ (a, b) \ | \ a \in A, \ b \in (B - C) \}$

$\Rightarrow A \times (B - C) = \{ (a, b) \ | \ a \in A, \ b \in B, \ b \notin C \}$

$\Rightarrow A \times (B - C) = \{ (a, b) \ | \ (a \in A \land b \in B) \land (a \in A \land b \notin C) \}$ 

$\Rightarrow A \times (B - C) = \{ (a, b) \ | \ (a \in A \land b \in B) \land (a \notin A \lor b \notin C) \}$

$\Rightarrow A \times (B - C) = \{ (a, b) \ | \ (a \in A \land b \in B) \land (a \notin A \lor b \in B \land b \notin C) \}$

$\Rightarrow A \times (B - C) = \{ (a, b) \ | \ (a \in A \land b \in B) \land (a \notin A \lor (b \in B \land b \notin C)) \}$

$\Rightarrow A \times (B - C) = \{ (a, b) \ | \ (a \in A \land b \in B) \land ((a \notin A \lor b \in B) \land (a \notin A \lor b \notin C)) \} $

$\Rightarrow A \times (B - C) = \{ (a, b) \ | \ (a \in A \land b \in B) \land (a \notin A \lor b \in B) \land (a \notin A \lor b \notin C) \}$

$\Rightarrow A \times (B - C) = \{ (a, b) \ | \ (a \in A \land b \in B) \land (a \notin A \lor b \notin C) \}$

$\Rightarrow A \times (B - C) = \{ (a, b) \ | \ a \in A, \ b \in B, \ b \notin C \}$

$\Rightarrow A \times (B - C) = (A \times B) - (A \times C)$

Hemos demostrado que \( A \times (B - C) = (A \times B) - (A \times C) \), como se quería demostrar.

\begin{itemize}
    \item \textbf{c) $(A - B) \times C = (A \times C) - (B \times C)$}
\end{itemize}

Para demostrar esto, utilizaremos la definición de los conjuntos involucrados:

$(A - B) \times C$ representa el conjunto de todos los pares ordenados $(a, c)$ donde $a \in A$ pero $a \notin B$, y $c \in C$.
$(A \times C) - (B \times C)$ representa el conjunto de todos los pares ordenados $(a, c)$ que están en $A \times C$ pero no en $B \times C$.
Demostración:

Tomemos un elemento $(a, c)$ en $(A - B) \times C$. Esto significa que $a \in A$ pero $a \notin B$, y $c \in C$.
Como $a \notin B$, tenemos que $a \in A$ y $a \notin B$.
Por lo tanto, $(a, c) \in A \times C$.
Además, como $a \notin B$, tenemos que $(a, c) \notin B \times C$.
Por lo tanto, $(a, c) \in (A \times C) - (B \times C)$.
Hemos demostrado que cualquier elemento en $(A - B) \times C$ también está en $(A \times C) - (B \times C)$. Ahora, debemos demostrar la otra dirección:

Tomemos un elemento $(a, c)$ en $(A \times C) - (B \times C)$. Esto significa que $(a, c) \in A \times C$ pero no en $B \times C$.
Esto implica que $a \in A$ y $c \in C$.
Además, como $(a, c) \notin B \times C$, tenemos que $a \notin B$.
Por lo tanto, $(a, c) \in (A - B) \times C$.
Hemos demostrado que cualquier elemento en $(A \times C) - (B \times C)$ también está en $(A - B) \times C$.

Matemáticamente:

$ (A - B) \times C = \{ (a, c) \ | \ a \in (A - B), \ c \in C \} $

$ \Rightarrow (A - B) \times C = \{ (a, c) \ | \ (a \in A \land a \notin B), \ c \in C \} $

$ \Rightarrow (A - B) \times C = \{ (a, c) \ | \ (a \in A \land a \notin B) \land (c \in C) \} $

$ \Rightarrow (A - B) \times C = \{ (a, c) \ | \ (a \in A \land c \in C) \land (a \notin B) \} $

$ \Rightarrow (A - B) \times C = \{ (a, c) \ | \ (a \in A \land c \in C) \} \setminus \{ (b, c) \ | \ (b \in B \land c \in C) \} $

$ \Rightarrow (A - B) \times C = (A \times C) - (B \times C) $

En resumen, hemos demostrado que $(A - B) \times C = (A \times C) - (B \times C)$.

\begin{itemize}
    \item \textbf{d) $(A \cap B) \times (C \cap D) = (A \times C) \cap (B \times D)$}
\end{itemize}

Para demostrar esto, utilizaremos la definición de los conjuntos involucrados:

$(A \cap B) \times (C \cap D)$ representa el conjunto de todos los pares ordenados $(a, c)$ donde $a \in A \cap B$ y $c \in C \cap D$.
$(A \times C) \cap (B \times D)$ representa el conjunto de todos los pares ordenados $(a, c)$ que están en $A \times C$ y también en $B \times D$.
La demostración sigue un razonamiento similar a los casos anteriores.

Matemáticamente:

$ (A \cap B) \times (C \cap D) = \{ (x, y) \ | \ x \in (A \cap B), \ y \in (C \cap D) \} $

$ \Rightarrow (A \cap B) \times (C \cap D) = \{ (x, y) \ | \ (x \in A \land x \in B), \ (y \in C \land y \in D) \} $

$ \Rightarrow (A \cap B) \times (C \cap D) = \{ (x, y) \ | \ (x \in A \land y \in C) \land (x \in B \land y \in D) \} $

$ \Rightarrow (A \cap B) \times (C \cap D) = \{ (x, y) \ | \ (x \in A \land y \in C) \} \cap \{ (x, y) \ | \ (x \in B \land y \in D) \} $

$ \Rightarrow (A \cap B) \times (C \cap D) = (A \times C) \cap (B \times D) $


\subsection*{\textbf{Ejercicio 3:}} Indicar cuál de los siguientes conjuntos representa una función: \\\\

\begin{tcolorbox}[colback=orange!10!white,colframe=orange!75!black,title=Observaciones]
    Cuando se dice que un conjunto \hl{representa una función}, significa que cada elemento del dominio tiene exactamente \hl{una imagen en el codominio}. En otras palabras, 
    para cada valor de entrada (dominio), hay un único valor de salida (imagen) asociado. Si un conjunto de pares ordenados cumple con esta propiedad, se considera que
    representa una función. Por ejemplo, si tenemos un conjunto de pares (x, y), es una función si para cada valor de “x”, hay un solo valor correspondiente de “y”. 
    Si hay más de una imagen posible para un valor de “x”, entonces no se trata de una función. Es una forma de describir cómo los elementos de un conjunto se relacionan 
    entre sí de manera consistente y predecible .
\end{tcolorbox}

\begin{itemize}
    \item a) $A = \{(a,b) \in \mathbb{N}^{2} : b = a^{2}\}$
\end{itemize}
Este conjunto \textbf{representa una función}, ya que para cada valor de \(a\) en el dominio, hay exactamente un valor de \(b\) en el codominio determinado por la relación \(b = a^{2}\).

\begin{itemize}
    \item b) $B = \{(x,y) \in \mathbb{R}^{2} : x = y^2 + 2y + 1\}$
\end{itemize}
En este caso, la relación entre “x” e “y” está dada por una ecuación cuadrática. Sin embargo, no todos los valores de “x” tienen una única imagen en forma de “y”. Por ejemplo, si tomamos “x = 1”, hay dos posibles valores para “y” (0 y -2). Por lo tanto, \textbf{B no representa una función}.

\begin{itemize}
    \item c) $C = \{(y,x) \in \mathbb{R}^2 : x = (y + 1)^2\}$
\end{itemize}
Este conjunto \textbf{no representa una función}, ya que la relación entre x e y también representa una ecuación cuadrática pero de otra manera, de tal forma que para algunos elementos en el dominio pueden existir 2 imágenes.

\begin{itemize}
    \item d) $D = \{(x,y) \in \mathbb{R}^2 : x = y^3\}$
\end{itemize}
Este conjunto, al igual que el primero, \textbf{representa una función}, ya que para cada valor de \(x\) en el dominio, hay exactamente un valor de \(y\) en el codominio determinado por la relación \(x = y^{3}\).

\subsection*{\textbf{Ejercicio 4:}}
Sea $E \neq \phi$ un conjunto fijo. Para todo subconjunto A de E se define la función
característica de A como:
\[\delta_A : E \rightarrow \{0,1\}\text{ tal que } x \mapsto \delta_A (x)=\begin{cases} 1 & \text{si } x \in A \\ 0 & \text{si } x \notin A \end{cases}\]

\begin{itemize}
    \item a) Describa $\delta_E(x)$ y $\delta_{\phi}(x)$ para todo $x \in E$.
    \begin{itemize}
        \item La función característica $\delta_E(x)$ para un conjunto $E$ asigna el valor 1 si el elemento $x$ pertenece a $E$, y 0 si no.
        \item Para todo $x \in E$, $\delta_E(x)$ se define como:
        \begin{itemize}
            \item $\delta_E(x) = 1$ si $x \in E$
            \item $\delta_E(x) = 0$ si $x \notin E$ 
        \end{itemize}
        \item La función característica $\delta_{\phi}(x)$ para el conjunto vacío $\phi$ (conjunto sin elementos) asigna siempre el valor 0, ya que no hay elementos en $\phi$.
        \item Para todo $x \in E$, $\delta_{\phi}(x)$ se define como:
        \begin{itemize}
            \item $\delta_{\phi}(x) = 0$
        \end{itemize}
    \end{itemize}
\end{itemize}

\begin{itemize}
    \item b) Demuestre que $\forall x \in E, \delta_{A\cap B}(x) = \delta_A(x)\cdot \delta_B(x)$.
\end{itemize}
Para demostrar que 
\(\forall x \in E, \delta_{A\cap B}(x) = \delta_A(x)\cdot \delta_B(x)\), consideremos dos casos:

1. Si \(x \in A \cap B\): En este caso, \(x\) pertenece tanto a \(A\) como a \(B\). Por lo tanto, \(\delta_A(x) = 1\) y \(\delta_B(x) = 1\). Entonces, \(\delta_A(x) \cdot \delta_B(x) = 1\). Además, \(x\) también pertenece a \(A \cap B\), lo que significa que \(\delta_{A \cap B}(x) = 1\). Por lo tanto, \(\delta_{A \cap B}(x) = \delta_A(x) \cdot \delta_B(x)\) en este caso.

2. Si \(x \notin A \cap B\): En este caso, \(x\) no pertenece a \(A \cap B\). Esto implica que al menos uno de los conjuntos \(A\) o \(B\) no contiene a \(x\). Por lo tanto, al menos uno de \(\delta_A(x)\) o \(\delta_B(x)\) es 0. Por lo tanto, \(\delta_A(x) \cdot \delta_B(x) = 0\). Y como \(x\) no está en \(A \cap B\), \(\delta_{A \cap B}(x) = 0\). Entonces, nuevamente, \(\delta_{A \cap B}(x) = \delta_A(x) \cdot \delta_B(x)\) en este caso.

Por lo tanto, en ambos casos, \(\delta_{A\cap B}(x) = \delta_A(x)\cdot \delta_B(x)\) para todo \(x \in E\).

\begin{itemize}
    \item c) Si $C, D \subseteq E$, entonces $C \subseteq D \Leftrightarrow \forall x \in E, \delta_C(x) \leq \delta_D(x)$
\end{itemize}

La afirmación \(C \subseteq D\) significa que todos los elementos de \(C\) también están en \(D\). Esto se puede expresar utilizando las funciones características como \(\forall x \in E, \delta_C(x) \leq \delta_D(x)\).
\begin{tcolorbox}
    \textbf{Observación:} la expresión 
    $\delta_C(x) \leq \delta_D(x)$ significa que el valor de la función característica $\delta_C(x)$ es menor o igual al valor de la función característica $\delta_D(x)$ para cada $x$ en $E$. En términos más simples, esto indica que para cada elemento $x$ en $E$, si $x$ está en $C$, entonces $x$ también estará en $D$, o en otras palabras, que $C$ está contenido en $D$.
\end{tcolorbox}
\begin{itemize}
    \item Demostración:
    \begin{itemize}
        \item Si un elemento \(x\) está en \(C\) pero no en \(D\), entonces \(\delta_C(x) = 1\) y \(\delta_D(x) = 0\), lo que cumple con la desigualdad.
        \item Si un elemento \(x\) está en \(C\) y también en \(D\), entonces \(\delta_C(x) = 1\) y \(\delta_D(x) = 1\), lo que también cumple con la desigualdad.
        \item Si un elemento \(x\) no está en \(C\), entonces \(\delta_C(x) = 0\), lo que también cumple con la desigualdad.
    \end{itemize}
\end{itemize}

Por lo tanto, \(C \subseteq D\) si y solo si \(\forall x \in E, \delta_C(x) \leq \delta_D(x)\).

\subsection*{\textbf{Ejercicio 5:}}
Dadas las funciones:
\[
\begin{aligned}
    \bullet f(x) &= 3x^2 + 5x + 3 & \quad\quad \bullet f(x) &= x^2 + 2x + 1\\
    \bullet f(x) &= -4x - 3 & \quad\quad \bullet f(x) &= \sqrt{12 - x^2}\\
    \bullet f(x) &= \frac{x-1}{x+1}
\end{aligned}
\]

Encuentre, si es posible, en cada caso:
\begin{itemize}
    \item[a)] $f(-10)$
    \item $f(x) = 3x^2 + 5x + 3$
    \begin{alignat*}{2}
        f(-10) &= 3(-10)^2 + 5(-10) + 3 \\
        &= 3(100) - 50 + 3 \\
        &= 300 - 50 + 3 \\
        &= 253
    \end{alignat*}
    \item $f(x) = x^2 + 2x + 1$
    \begin{alignat*}{2}
        f(-10) &= (-10)^2 + 2(-10) + 1 \\
        &= 100 - 20 + 1 \\
        &= 81
    \end{alignat*}
    \item $f(x) = -4x - 3$
    \begin{alignat*}{2}
        f(-10) &= -4(-10) - 3 \\
        &= 40 - 3 \\
        &= 37
    \end{alignat*}
    \item $f(x) = \sqrt{12 - x^2}$
    \begin{alignat*}{2}
        f(-10) &= \sqrt{12 - (-10)^2} \\
        &= \sqrt{12 - 100} \\
        &= \sqrt{-88} \text{ (No es un número real)}   
    \end{alignat*}
    \item $f(x) = \frac{x-1}{x+1}$
    \begin{alignat*}{2}
        f(-10) &= \frac{-10 - 1}{-10 + 1} \\
        &= \frac{-11}{-9} \\
        &= \frac{11}{9}
    \end{alignat*}
\end{itemize}

\begin{itemize}
    \item[b)] $f(\frac{1}{3})$
    \item $f(x) = 3x^2 + 5x + 3$
    \begin{alignat*}{2}
        f(\frac{1}{3}) &= 3(\frac{1}{3})^2 + 5(\frac{1}{3}) + 3 \\
        &= 3(\frac{1}{9}) + \frac{5}{3} + 3 \\
        &= \frac{1}{3} + \frac{5}{3} + 3 \\
        &= \frac{9}{3} \\
        &= 3
    \end{alignat*}
    \item $f(x) = x^2 + 2x + 1$
    \begin{alignat*}{2}
        f(\frac{1}{3}) &= (\frac{1}{3})^2 + 2(\frac{1}{3}) + 1 \\
        &= \frac{1}{9} + \frac{2}{3} + 1 \\
        &= \frac{1}{9} + \frac{6}{9} + \frac{9}{9} \\
        &= \frac{16}{9}
    \end{alignat*}
    \item $f(x) = -4x - 3$
    \begin{alignat*}{2}
        f(\frac{1}{3}) &= -4(\frac{1}{3}) - 3 \\
        &= -\frac{4}{3} - 3 \\
        &= -\frac{4}{3} - \frac{9}{3} \\
        &= -\frac{13}{3}
    \end{alignat*}
    \item $f(x) = \sqrt{12 - x^2}$
    \begin{alignat*}{2}
        f(\frac{1}{3}) &= \sqrt{12 - (\frac{1}{3})^2} \\
        &= \sqrt{12 - \frac{1}{9}} \\
        &= \sqrt{\frac{107}{9}}
    \end{alignat*}
    \item $f(x) = \frac{x-1}{x+1}$
    \begin{alignat*}{2}
        f(\frac{1}{3}) &= \frac{\frac{1}{3} - 1}{\frac{1}{3} + 1} \\
        &= \frac{\frac{1}{3} - \frac{3}{3}}{\frac{1}{3} + \frac{3}{3}} \\
        &= \frac{-\frac{2}{3}}{\frac{4}{3}} \\
        &= -\frac{1}{2}
    \end{alignat*}
\end{itemize}

\begin{itemize}
    \item[c)] $f(2)$
    \item $f(x) = 3x^2 + 5x + 3$
    \begin{alignat*}{2}
        f(2) &= 3(2)^2 + 5(2) + 3 \\
        &= 3(4) + 10 + 3 \\
        &= 12 + 10 + 3 \\
        &= 25
    \end{alignat*}
    \item $f(x) = x^2 + 2x + 1$ 
    \begin{alignat*}{2}
        f(2) &= (2)^2 + 2(2) + 1 \\
        &= 4 + 4 + 1 \\
        &= 9
    \end{alignat*}
    \item $f(x) = -4x - 3$
    \begin{alignat*}{2}
        f(2) &= -4(2) - 3 \\
        &= -8 - 3 \\
        &= -11
    \end{alignat*}
    \item $f(x) = \sqrt{12 - x^2}$  
    \begin{alignat*}{2}
        f(2) &= \sqrt{12 - (2)^2} \\
        &= \sqrt{12 - 4} \\
        &= \sqrt{8}
    \end{alignat*}
    \item $f(x) = \frac{x-1}{x+1}$
    \begin{alignat*}{2}
        f(2) &= \frac{2 - 1}{2 + 1} \\
        &= \frac{1}{3}
    \end{alignat*}
\end{itemize}

\begin{itemize}
    \item[d)] $f(a + 1)$
    \item $f(x) = 3x^2 + 5x + 3$
    \begin{alignat*}{2}
        f(a + 1) &= 3(a + 1)^2 + 5(a + 1) + 3 \\
        &= 3(a^2 + 2a + 1) + 5a + 5 + 3 \\
        &= 3a^2 + 6a + 3 + 5a + 8 \\
        &= 3a^2 + 11a + 11
    \end{alignat*}
    \item $f(x) = x^2 + 2x + 1$
    \begin{alignat*}{2}
        f(a + 1) &= (a + 1)^2 + 2(a + 1) + 1 \\
        &= a^2 + 2a + 1 + 2a + 2 + 1 \\
        &= a^2 + 4a + 4
    \end{alignat*}
    \item $f(x) = -4x - 3$
    \begin{alignat*}{2}
        f(a + 1) &= -4(a + 1) - 3 \\
        &= -4a - 4 - 3 \\
        &= -4a - 7
    \end{alignat*}
    \item $f(x) = \sqrt{12 - x^2}$
    \begin{alignat*}{2}
        f(a + 1) &= \sqrt{12 - (a + 1)^2} \\
        &= \sqrt{12 - (a^2 + 2a + 1)} \\
        &= \sqrt{11 - a^2 - 2a}
    \end{alignat*}
    \item $f(x) = \frac{x-1}{x+1}$
    \begin{alignat*}{2}
        f(a + 1) &= \frac{a + 1 - 1}{a + 1 + 1} \\
        &= \frac{a}{a + 2}
    \end{alignat*}
\end{itemize}

\begin{itemize}
    \item[e)] $f(\frac{1}{a})$
    \item $f(x) = 3x^2 + 5x + 3$
    \begin{alignat*}{2}
        f(\frac{1}{a}) &= 3(\frac{1}{a})^2 + 5(\frac{1}{a}) + 3 \\
        &= 3(\frac{1}{a^2}) + \frac{5}{a} + 3 \\
        &= \frac{3}{a^2} + \frac{5}{a} + 3
    \end{alignat*}
    \item $f(x) = x^2 + 2x + 1$
    \begin{alignat*}{2}
        f(\frac{1}{a}) &= (\frac{1}{a})^2 + 2(\frac{1}{a}) + 1 \\
        &= \frac{1}{a^2} + \frac{2}{a} + 1
    \end{alignat*}
    \item $f(x) = -4x - 3$
    \begin{alignat*}{2}
        f(\frac{1}{a}) &= -4(\frac{1}{a}) - 3 \\
        &= -\frac{4}{a} - 3
    \end{alignat*}
    \item $f(x) = \sqrt{12 - x^2}$
    \begin{alignat*}{2}
        f(\frac{1}{a}) &= \sqrt{12 - (\frac{1}{a})^2} \\
        &= \sqrt{12 - \frac{1}{a^2}} \\
        &= \sqrt{\frac{12a^2 - 1}{a^2}}
    \end{alignat*}
    \item $f(x) = \frac{x-1}{x+1}$
    \begin{alignat*}{2}
        f(\frac{1}{a}) &= \frac{\frac{1}{a} - 1}{\frac{1}{a} + 1} \\
        &= \frac{\frac{1 - a}{a}}{\frac{1 + a}{a}} \\
        &= \frac{1 - a}{1 + a}
    \end{alignat*}
\end{itemize}

\begin{itemize}
    \item[f)] $f(a + 2) - f(a)$
    \item $f(x) = 3x^2 + 5x + 3$
    \begin{alignat*}{2}
        f(a + 2) - f(a) &= (3(a + 2)^2 + 5(a + 2) + 3) - (3a^2 + 5a + 3) \\
        &= (3(a^2 + 4a + 4) + 5a + 10 + 3) - (3a^2 + 5a + 3) \\
        &= 3a^2 + 12a + 12 + 5a + 13 - 3a^2 - 5a - 3 \\
        &= 7a + 22
    \end{alignat*}
    \item $f(x) = x^2 + 2x + 1$
    \begin{alignat*}{2}
        f(a + 2) - f(a) &= ((a + 2)^2 + 2(a + 2) + 1) - (a^2 + 2a + 1) \\
        &= (a^2 + 4a + 4 + 2a + 4 + 1) - (a^2 + 2a + 1) \\
        &= a^2 + 6a + 9 - a^2 - 2a - 1 \\
        &= 4a + 8
    \end{alignat*}
    \item $f(x) = -4x - 3$
    \begin{alignat*}{2}
        f(a + 2) - f(a) &= (-4(a + 2) - 3) - (-4a - 3) \\
        &= (-4a - 8 - 3) - (-4a - 3) \\
        &= -4a - 11 + 4a + 3 \\
        &= -8
    \end{alignat*}
    \item $f(x) = \sqrt{12 - x^2}$
    \begin{alignat*}{2}
        f(a + 2) - f(a) &= (\sqrt{12 - (a + 2)^2}) - (\sqrt{12 - a^2}) \\
        &= (\sqrt{12 - (a^2 + 4a + 4)}) - (\sqrt{12 - a^2}) \\
        &= (\sqrt{8 - a^2 - 4a}) - (\sqrt{12 - a^2})
    \end{alignat*}
    \item $f(x) = \frac{x-1}{x+1}$
    \begin{alignat*}{2}
        f(a + 2) - f(a) &= (\frac{a + 2 - 1}{a + 2 + 1}) - (\frac{a - 1}{a + 1}) \\
        &= (\frac{a + 1}{a + 3}) - (\frac{a - 1}{a + 1}) \\
        &= \frac{(a + 1)^2 - (a - 1)(a + 3)}{(a + 3)(a + 1)} \\
        &= \frac{a^2 + 2a + 1 - (a^2 + 3a - a - 3)}{a^2 + 4a + 3} \\
        &= \frac{2a + 4}{a^2 + 4a + 3}
    \end{alignat*}
\end{itemize}

\begin{itemize}
    \item[g)] $\frac{f(x+h)-f(x)}{h}$
    \item $f(x) = 3x^2 + 5x + 3$
    \begin{alignat*}{2}
        \frac{f(x+h)-f(x)}{h} &= \frac{3(x + h)^2 + 5(x + h) + 3 - (3x^2 + 5x + 3)}{h} \\
        &= \frac{3(x^2 + 2xh + h^2) + 5x + 5h + 3 - 3x^2 - 5x - 3}{h} \\
        &= \frac{3x^2 + 6xh + 3h^2 + 5x + 5h + 3 - 3x^2 - 5x - 3}{h} \\
        &= \frac{6xh + 3h^2 + 5h}{h} \\
        &= 6x + 3h + 5
    \end{alignat*}
    \item $f(x) = x^2 + 2x + 1$
    \begin{alignat*}{2}
        \frac{f(x+h)-f(x)}{h} &= \frac{(x + h)^2 + 2(x + h) + 1 - (x^2 + 2x + 1)}{h} \\
        &= \frac{x^2 + 2xh + h^2 + 2x + 2h + 1 - x^2 - 2x - 1}{h} \\
        &= \frac{2xh + h^2 + 2h}{h} \\
        &= 2x + h + 2
    \end{alignat*}
    \item $f(x) = -4x - 3$
    \begin{alignat*}{2}
        \frac{f(x+h)-f(x)}{h} &= \frac{-4(x + h) - 3 - (-4x - 3)}{h} \\
        &= \frac{-4x - 4h - 3 + 4x + 3}{h} \\
        &= \frac{-4h}{h} \\
        &= -4
    \end{alignat*}
    \item $f(x) = \sqrt{12 - x^2}$
    \begin{alignat*}{2}
        \frac{f(x+h)-f(x)}{h} &= \frac{\sqrt{12 - (x + h)^2} - \sqrt{12 - x^2}}{h} \\
        &= \frac{\sqrt{12 - x^2 - 2xh - h^2} - \sqrt{12 - x^2}}{h} \\
        &= \frac{\sqrt{12 - x^2 - 2xh - h^2} - \sqrt{12 - x^2}}{h} \cdot \frac{\sqrt{12 - x^2 - 2xh - h^2} + \sqrt{12 - x^2}}{\sqrt{12 - x^2 - 2xh - h^2} + \sqrt{12 - x^2}} \\
        &= \frac{12 - x^2 - 2xh - h^2 - 12 + x^2}{h(\sqrt{12 - x^2 - 2xh - h^2} + \sqrt{12 - x^2})} \\
        &= \frac{-2xh - h^2}{h(\sqrt{12 - x^2 - 2xh - h^2} + \sqrt{12 - x^2})} \\
        &= \frac{-h(2x + h)}{h(\sqrt{12 - x^2 - 2xh - h^2} + \sqrt{12 - x^2})} \\
        &= \frac{-2x - h}{\sqrt{12 - x^2 - 2xh - h^2} + \sqrt{12 - x^2}}
    \end{alignat*}
    \item $f(x) = \frac{x-1}{x+1}$
    \begin{alignat*}{2}
        \frac{f(x+h)-f(x)}{h} &= \frac{\frac{x + h - 1}{x + h + 1} - \frac{x - 1}{x + 1}}{h} \\
        &= \frac{\frac{x + h - 1}{x + h + 1} - \frac{x - 1}{x + 1}}{h} \cdot \frac{(x + h + 1)(x + 1)}{(x + h + 1)(x + 1)} \\
        &= \frac{(x + h - 1)(x + 1) - (x - 1)(x + h + 1)}{h(x + h + 1)(x + 1)} \\
        &= \frac{x^2 + xh + x - x + hx - 1 - x^2 - hx + x + xh - 1}{h(x + h + 1)(x + 1)} \\
        &= \frac{2xh - 2}{h(x + h + 1)(x + 1)} \\
        &= \frac{2(xh - 1)}{h(x + h + 1)(x + 1)} \\
        &= \frac{2}{(x + h + 1)(x + 1)}
    \end{alignat*}
\end{itemize}

\subsection*{\textbf{Ejercicio 6:}}
Encuentre el dominio, recorrido y ceros de las siguientes funciones:
\begin{itemize}
    \item[a)] $f(x) = 5x + 1$
    \begin{itemize}
        \item \textbf{Dominio:}
        \begin{alignat*}{2}
            \text{El dominio de } f(x) &= \mathbb{R}
        \end{alignat*}
        Ya que es una función lineal, el dominio es todo el conjunto de los números reales.
        \item \textbf{Recorrido:}
        \begin{alignat*}{2}
            \text{El recorrido de } f(x) &= \mathbb{R}
        \end{alignat*}
        El recorrido de una función lineal es todo el conjunto de los números reales.
        \item \textbf{Ceros:}
        \begin{alignat*}{2}
            5x + 1 &= 0 \\
            5x &= -1 \\
            x &= -\frac{1}{5}
        \end{alignat*}
    \end{itemize}
    \item[b)] $f(x) = 2x^2 + 3x - 1$
    \begin{itemize}
        \item \textbf{Dominio:}
        \begin{alignat*}{2}
            \text{El dominio de } f(x) &= \mathbb{R}
        \end{alignat*}
        El dominio de un polinomio es todo el conjunto de los números reales.
        \item \textbf{Recorrido:}
        \begin{align*}
            f(x) &= 2x^2 + 3x - 1 \\
            &= 2(x^2 + \frac{3}{2}x) - 1 \\
            &= 2(x^2 + \frac{3}{2}x + \frac{9}{16}) - 1 - 2 \cdot \frac{9}{16} \\
            &= 2(x + \frac{3}{4})^2 - 1 - \frac{9}{8} \\
            &= 2(x + \frac{3}{4})^2 - \frac{17}{8}
        \end{align*}
        Para minimizar \( f(x) \), el término \( (x + \frac{3}{4})^2 \) debe ser 0, lo que ocurre cuando \( x = -\frac{3}{4} \). Por lo tanto, el valor mínimo de \( f(x) \) es \( f(-\frac{3}{4}) = -\frac{17}{8} \).\\
        El recorrido de \( f(x) \) es entonces el conjunto de todos los valores \( y \) en los números reales mayores o iguales a \( -\frac{17}{8} \):
        \begin{align*}
            \text{Recorrido de } f(x) &= \{y \in \mathbb{R} : y \geq -\frac{17}{8}\}
        \end{align*}
        \item \textbf{Ceros:}
        \begin{alignat*}{2}
            2x^2 + 3x - 1 &= 0 \\
            x &= \frac{-3 \pm \sqrt{3^2 - 4(2)(-1)}}{2(2)} \\
            x &= \frac{-3 \pm \sqrt{9 + 8}}{4} \\
            x &= \frac{-3 \pm \sqrt{17}}{4}
        \end{alignat*}
    \end{itemize}
    \item[c)] $f(x) = -x^2 - 4x - 1$
    \begin{itemize}
        \item \textbf{Dominio:}
        \begin{alignat*}{2}
            \text{El dominio de } f(x) &= \mathbb{R}
        \end{alignat*}
        El dominio de un polinomio es todo el conjunto de los números reales.
        \item \textbf{Recorrido:}
        \begin{align*}
            f(x) &= -x^2 - 4x - 1 \\
            &= -(x^2 + 4x) - 1 \\
            &= -(x^2 + 4x + 4) - 1 + 4 \\
            &= -(x + 2)^2 + 3
        \end{align*}
        Para maximizar \( f(x) \), el término \( (x + 2)^2 \) debe ser 0, lo que ocurre cuando \( x = -2 \). Por lo tanto, el valor máximo de \( f(x) \) es \( f(-2) = 3 \).\\
        El recorrido de \( f(x) \) es entonces el conjunto de todos los valores \( y \) en los números reales menores o iguales a 3:
        \begin{align*}
            \text{Recorrido de } f(x) &= \{y \in \mathbb{R} : y \leq 3\}
        \end{align*}
        \item \textbf{Ceros:}
        \begin{alignat*}{2}
            -x^2 - 4x - 1 &= 0 \\
            x &= \frac{-(-4) \pm \sqrt{(-4)^2 - 4(-1)(-1)}}{2(-1)} \\
            x &= \frac{4 \pm \sqrt{16 - 4}}{-2} \\
            x &= \frac{4 \pm \sqrt{12}}{-2} \\
            x &= \frac{4 \pm 2\sqrt{3}}{-2} \\
            x &= -2 \pm \sqrt{3}
        \end{alignat*}
    \end{itemize}
    \item[d)] $f(x) = \frac{2}{x - 1}$
    \begin{itemize}
        \item \textbf{Dominio:}
        \begin{alignat*}{2}
            x - 1 &\neq 0 \\
            x &\neq 1
        \end{alignat*}
        \begin{alignat*}{2}
            \text{El dominio de } f(x) &= \{x \in \mathbb{R} : x \neq 1\} \\
            &= (-\infty, 1) \cup (1, +\infty)
        \end{alignat*}
        \item \textbf{Recorrido:}
        \begin{align*}
            f(x) &= \frac{2}{x - 1} \\
            &= \frac{2}{x - 1} \cdot \frac{x + 1}{x + 1} \\
            &= \frac{2(x + 1)}{x^2 - 1} \\
            &= \frac{2x + 2}{x^2 - 1}
        \end{align*}
        El recorrido de \( f(x) \) es entonces el conjunto de todos los valores \( y \) en los números reales:
        \begin{align*}
            \text{Recorrido de } f(x) &= \mathbb{R}
        \end{align*}
        \item \textbf{Ceros:}
        \begin{alignat*}{2}
            \frac{2}{x - 1} &= 0 \\
            2 &= 0 \cdot (x - 1) \\
            2 &= 0
        \end{alignat*}
        No hay ceros para esta función.
    \end{itemize}
    \item[e)] $f(x) = \frac{x^2}{x^2 - 1}$
    \begin{itemize}
        \item \textbf{Dominio:}
        \begin{alignat*}{2}
            x^2 - 1 &\neq 0 \\
            x^2 &\neq 1 \\
            x &\neq \pm 1
        \end{alignat*}
        \begin{alignat*}{2}
            \text{El dominio de } f(x) &= \{x \in \mathbb{R} : x \neq \pm 1\} \\
            &= (-\infty, -1) \cup (-1, 1) \cup (1, +\infty)
        \end{alignat*}

        \item \textbf{Recorrido:}
        \begin{align*}
            f(x) &= \frac{x^2}{x^2 - 1} \\
            &= \frac{x^2}{(x - 1)(x + 1)} \\
            &= \frac{x^2}{(x - 1)(x + 1)} \cdot \frac{x + 1}{x + 1} \\
            &= \frac{x^3 + x^2}{x^2 + x - x - 1} \\
            &= \frac{x^3 + x^2}{x^2 - 1} \\
            &= x + \frac{x^2}{x^2 - 1}
        \end{align*}
        El recorrido de \( f(x) \) es entonces el conjunto de todos los valores \( y \) en los números reales:
        \begin{align*}
            \text{Recorrido de } f(x) &= \mathbb{R}
        \end{align*}
        \item \textbf{Ceros:}
        \begin{alignat*}{2}
            \frac{x^2}{x^2 - 1} &= 0 \\
            x^2 &= 0 \\
            x &= 0
        \end{alignat*}
    \end{itemize}
    \item[f)] $f(x) = \frac{3}{x^2 + 5x + 6}$
    \begin{itemize}
        \item \textbf{Dominio:}
        \begin{align*}
            x^2 + 5x + 6 &\neq 0 \\
            (x + 2)(x + 3) &\neq 0 \\
            x &\neq -2, -3
        \end{align*}
        \begin{align*}
            \text{El dominio de } f(x) &= \{x \in \mathbb{R} : x \neq -2, -3\} \\
            &= (-\infty, -3) \cup (-3, -2) \cup (-2, +\infty)
        \end{align*}
        \item \textbf{Recorrido:}
        \begin{align*}
            f(x) &= \frac{3}{x^2 + 5x + 6} \\
            &= \frac{3}{(x + 2)(x + 3)}
        \end{align*}
        El recorrido de \( f(x) \) es entonces el conjunto de todos los valores \( y \) en los números reales:
        \begin{align*}
            \text{Recorrido de } f(x) &= \mathbb{R}
        \end{align*}
        \item \textbf{Ceros:}
        \begin{align*}
            \frac{3}{x^2 + 5x + 6} &= 0 \\
            3 &= 0 \cdot (x^2 + 5x + 6) \\
            3 &= 0
        \end{align*}
        No hay ceros para esta función.
    \end{itemize}
    \item[g)] $f(x) = \frac{3x - 3}{x^2 - x - 56}$
    \begin{itemize}
        \item \textbf{Dominio:}
        \begin{align*}
            x^2 - x - 56 &\neq 0 \\
            (x - 8)(x + 7) &\neq 0 \\
            x &\neq 8, -7
        \end{align*}
        \begin{align*}
            \text{El dominio de } f(x) &= \{x \in \mathbb{R} : x \neq 8, -7\} \\
            &= (-\infty, -7) \cup (-7, 8) \cup (8, +\infty)
        \end{align*}
        \item \textbf{Recorrido:}
        \begin{align*}
            f(x) &= \frac{3x - 3}{x^2 - x - 56} \\
            &= \frac{3(x - 1)}{(x - 8)(x + 7)}
        \end{align*}
        El recorrido de \( f(x) \) es entonces el conjunto de todos los valores \( y \) en los números reales:
        \begin{align*}
            \text{Recorrido de } f(x) &= \mathbb{R}
        \end{align*}
        \item \textbf{Ceros:}
        \begin{align*}
            \frac{3x - 3}{x^2 - x - 56} &= 0 \\
            3x - 3 &= 0 \\
            3x &= 3 \\
            x &= 1
        \end{align*}
    \end{itemize}
    \item[h)] $f(x) = \sqrt{x^2 - 9}$
    \begin{itemize}
        \item \textbf{Dominio:}
        \begin{align*}
            x^2 - 9 &\geq 0 \\
            (x - 3)(x + 3) &\geq 0 \\
            x &\in (-\infty, -3] \cup [3, +\infty)
        \end{align*}
        \begin{align*}
            \text{El dominio de } f(x) &= \{x \in \mathbb{R} : x \in (-\infty, -3] \cup [3, +\infty)\}
        \end{align*}
        \item \textbf{Recorrido:}
        \begin{align*}
            f(x) &= \sqrt{x^2 - 9} \\
            &= \sqrt{(x - 3)(x + 3)}
        \end{align*}
        El recorrido de \( f(x) \) es entonces el conjunto de todos los valores \( y \) en los números reales mayores o iguales a 0:
        \begin{align*}
            \text{Recorrido de } f(x) &= \{y \in \mathbb{R} : y \geq 0\}
        \end{align*}
        \item \textbf{Ceros:}
        \begin{align*}
            \sqrt{x^2 - 9} &= 0 \\
            x^2 - 9 &= 0 \\
            x^2 &= 9 \\
            x &= \pm 3
        \end{align*}
    \end{itemize}
    \item[i)] $f(x) = \sqrt{16 - x^2}$
    \begin{itemize}
        \item \textbf{Dominio:}
        \begin{align*}
            16 - x^2 &\geq 0 \\
            (4 - x)(4 + x) &\geq 0 \\
            x &\in [-4, 4]
        \end{align*}
        \begin{align*}
            \text{El dominio de } f(x) &= \{x \in \mathbb{R} : x \in [-4, 4]\}
        \end{align*}
        \item \textbf{Recorrido:}
        \begin{align*}
            f(x) &= \sqrt{16 - x^2}
        \end{align*}
        El recorrido de \( f(x) \) es entonces el conjunto de todos los valores \( y \) en los números reales mayores o iguales a 0:
        \begin{align*}
            \text{Recorrido de } f(x) &= \{y \in \mathbb{R} : y \geq 0\}
        \end{align*}
        \item \textbf{Ceros:}
        \begin{align*}
            \sqrt{16 - x^2} &= 0 \\
            16 - x^2 &= 0 \\
            x^2 &= 16 \\
            x &= \pm 4
        \end{align*}
    \end{itemize}
    \item[j)] $f(x) = \frac{6 - 11x}{\sqrt{-3x - 15}}$
    \begin{itemize}
        \item \textbf{Dominio:}
        \begin{align*}
            -3x - 15 &> 0 \\
            -3x &> 15 \\
            x &< -5
        \end{align*}
        \begin{align*}
            \text{El dominio de } f(x) &= \{x \in \mathbb{R} : x < -5\}
        \end{align*}
        \item \textbf{Recorrido:}
        \begin{align*}
            f(x) &= \frac{6 - 11x}{\sqrt{-3x - 15}} \\
            y &= \frac{6 - 11x}{\sqrt{-3x - 15}} \\
            y \cdot \sqrt{-3x - 15} &= 6 - 11x \\
            y^2(-3x - 15) &= (6 - 11x)^2 \\
            -3xy^2 - 15y^2 &= 36 - 132x + 121x^2 \\
            121x^2 - 132x - 3xy^2 - 15y^2 - 36 &= 0 \\
            x &= \frac{132 \pm \sqrt{132^2 - 4(121)(-3y^2 - 15y^2 - 36)}}{2(121)} \\
            x &= \frac{132 \pm \sqrt{17424 + 1452y^2 + 7260y^2 + 1452 \cdot 36}}{242} \\
            x &= \frac{132 \pm \sqrt{17424 + 8712y^2 + 52272y^2 + 52272}}{242} \\
            x &= \frac{132 \pm \sqrt{69696 + 60984y^2}}{242} \\
            x &= \frac{132 \pm 2\sqrt{17424 + 15246y^2}}{242} \\
            x &= \frac{66 \pm \sqrt{17424 + 15246y^2}}{121}
        \end{align*}
        El recorrido de \( f(x) \) es entonces el conjunto de todos los valores \( y \) en los números reales:
        \begin{align*}
            \text{Recorrido de } f(x) &= \mathbb{R}
        \end{align*}
        \item \textbf{Ceros:}
        \begin{align*}
            \frac{6 - 11x}{\sqrt{-3x - 15}} &= 0 \\
            &\downarrow \text{ Solo nos interesa cuando el numerador sea 0.}\\
            6 - 11x &= 0 \\
            11x &= 6 \\
            x &= \frac{6}{11}
        \end{align*}
    \end{itemize}
    \item[k)] $f(x) = \frac{1}{\sqrt{3x + 5}}$
    \begin{itemize}
        \item \textbf{Dominio:}
        \begin{align*}
            3x + 5 &> 0 \\
            3x &> -5 \\
            x &> -\frac{5}{3}
        \end{align*}
        \begin{align*}
            \text{El dominio de } f(x) &= \{x \in \mathbb{R} : x > -\frac{5}{3}\}
        \end{align*}
        \item \textbf{Recorrido:}
        \begin{align*}
            f(x) &= \frac{1}{\sqrt{3x + 5}}
        \end{align*}
        El recorrido de \( f(x) \) es entonces el conjunto de todos los valores \( y \) en los números reales mayores a 0:
        \begin{align*}
            \text{Recorrido de } f(x) &= \{y \in \mathbb{R} : y > 0\}
        \end{align*}
        \item \textbf{Ceros:}
        \begin{align*}
            \frac{1}{\sqrt{3x + 5}} &= 0
        \end{align*}
        No hay ceros para esta función.
    \end{itemize}
    \item[l)] $f(x) = \frac{x^2 + x}{x^2 - x}$
    \begin{itemize}
        \item \textbf{Dominio:}
        \begin{align*}
            x^2 - x &\neq 0 \\
            x(x - 1) &\neq 0 \\
            x &\neq 0, 1
        \end{align*}
        \begin{align*}
            \text{El dominio de } f(x) &= \{x \in \mathbb{R} : x \neq 0, 1\}
        \end{align*}
        \item \textbf{Recorrido:}
        \begin{align*}
            f(x) &= \frac{x^2 + x}{x^2 - x} \\
            &= \frac{x(x + 1)}{x(x - 1)} \\
            &= \frac{x + 1}{x - 1}
        \end{align*}
        El recorrido de \( f(x) \) es entonces el conjunto de todos los valores \( y \) en los números reales:
        \begin{align*}
            \text{Recorrido de } f(x) &= \mathbb{R}
        \end{align*}
        \item \textbf{Ceros:}
        \begin{align*}
            \frac{x^2 + x}{x^2 - x} &= 0
        \end{align*}
        No hay ceros para esta función.
    \end{itemize}
    \item[m)] $f(x) = \sqrt{1 - (x + 1)(x - 1)}$
    \begin{itemize}
        \item \textbf{Dominio:}
        \begin{align*}
            1 - (x + 1)(x - 1) &\geq 0 \\
            1 - (x^2 - 1) &\geq 0 \\
            1 - x^2 + 1 &\geq 0 \\
            2 - x^2 &\geq 0 \\
            x^2 &\leq 2 \\
            x &\in [-\sqrt{2}, \sqrt{2}]
        \end{align*}
        \begin{align*}
            \text{El dominio de } f(x) &= \{x \in \mathbb{R} : x \in [-\sqrt{2}, \sqrt{2}]\}
        \end{align*}
        \item \textbf{Recorrido:}
        \begin{align*}
            f(x) &= \sqrt{1 - (x + 1)(x - 1)} \\
            &= \sqrt{1 - (x^2 - 1)} \\
            &= \sqrt{2 - x^2}
        \end{align*}
        El recorrido de \( f(x) \) es entonces el conjunto de todos los valores \( y \) en los números reales mayores o iguales a 0:
        \begin{align*}
            \text{Recorrido de } f(x) &= \{y \in \mathbb{R} : y \geq 0\}
        \end{align*}
        \item \textbf{Ceros:}
        \begin{align*}
            \sqrt{1 - (x + 1)(x - 1)} &= 0
        \end{align*}
        No hay ceros para esta función.
    \end{itemize}
    \item[n)] $f(x) = \sqrt{|x - 1|} - 1$
    \begin{itemize}
        \item \textbf{Dominio:}
        \begin{align*}
            |x - 1| &\geq 0
        \end{align*}
        \begin{align*}
            \text{El dominio de } f(x) &= \mathbb{R}
        \end{align*}
        \item \textbf{Recorrido:}
        \begin{align*}
            f(x) &= \sqrt{|x - 1|} - 1
        \end{align*}
        El recorrido de \( f(x) \) es entonces el conjunto de todos los valores \( y \) en los números reales mayores o iguales a -1:
        \begin{align*}
            \text{Recorrido de } f(x) &= \{y \in \mathbb{R} : y \geq -1\}
        \end{align*}
        \item \textbf{Ceros:}
        \begin{align*}
            \sqrt{|x - 1|} - 1 &= 0 \\
            \sqrt{|x - 1|} &= 1 \\
            |x - 1| &= 1 
            \begin{cases} \text{si } (x-1=1), \text{entonces } (x=2) \\ \text{si } (x-1=-1), \text{entonces } (x=0)\end{cases}
        \end{align*}
        Por lo tanto los ceros para esta función son 2 y 0.
    \end{itemize}
\end{itemize}

\subsection*{\textbf{Ejercicio 7:}}
Para las siguientes funciones encuentre el dominio, recorrido, ceros, paridad,
crecimiento, decrecimiento, acotamiento y el signo:

\begin{itemize}
    \item[a)] $f(x) = x^3$
        \begin{itemize}
            \item \text{Dominio:} El dominio de la función es todos los números reales, ya que puedes sustituir cualquier número real por $x$ en la función $f(x) = x^3$.
                \begin{align*}
                    \text{Dom}(f) &= \{x \in \mathbb{R} : f(x) = y\} \\
                    \text{Dom}(f) &= \mathbb{R}    
                \end{align*}
            \item \text{Recorrido:} El recorrido también es todos los números reales, ya que la función cúbica puede tomar cualquier valor real para cualquier $x$.
                \begin{align*}
                    \text{Rec}(f) &= \{y \in \mathbb{R} : f(x) = y\} \\
                    \text{Rec}(f) &= \mathbb{R}
                \end{align*}
            \item \text{Ceros:} La función es cero cuando $x = 0$, por lo que el cero de la función es $x = 0$.
                \begin{align*}
                    f(x) &= x^3 \\
                    0 &= x^3 \\
                    x &= 0
                \end{align*}
            \item \text{Paridad:} La función es impar, ya que $f(-x) = -f(x)$ para todo $x$ en el dominio de $f$.
                \begin{align*}
                    f(-x) &= (-x)^3 \\
                    &= -x^3 \\
                    &= -f(x)
                \end{align*}
            Para crecimiento y decrecimiento definamos un $x_{1}=5 < x_2=7$
            \item \text{Crecimiento:} La función está creciendo en todo su dominio, ya que para cualquier par de números reales $a$ y $b$, si $a < b$, entonces $f(a) < f(b)$.
                \begin{align*}
                    f(x_1) &= 5^3 = 125 \\
                    f(x_2) &= 7^3 = 343 \\
                    125 &< 343
                \end{align*}
            \item \text{Decrecimiento:} La función no está decreciendo en ninguna parte de su dominio.
            \item \text{Acotamiento:} La función no está acotada, ya que puede tomar cualquier valor real.
                \begin{align*}
                    \lim_{x \to \infty} f(x) &= \infty \\
                    \lim_{x \to -\infty} f(x) &= -\infty
                \end{align*}
            \item \text{Signo:} La función es positiva para $x > 0$, negativa para $x < 0$, y cero para $x = 0$.
        \end{itemize}
    \item[b)] $f(x) = |x| - \sqrt{1 - x^2}$
        \begin{itemize}
            \item \text{Dominio:} El dominio de la función es $-1 \leq x \leq 1$, ya que la raíz cuadrada de un número negativo no es un número real.
                \begin{align*}
                    \text{Dom}(f) &= \{x \in \mathbb{R} : -1 \leq x \leq 1\}
                \end{align*}
            \item \text{Recorrido:} El recorrido de la función es $0 \leq y \leq 1$, ya que la raíz cuadrada de un número negativo no es un número real.
                \begin{align*}
                    \text{Rec}(f) &= \{y \in \mathbb{R} : 0 \leq y \leq 1\}
                \end{align*}
            \item \text{Ceros:} La función es cero cuando $x = 0$, por lo que el cero de la función es $x = 0$.
                \begin{align*}
                    f(x) &= |x| - \sqrt{1 - x^2} \\
                    0 &= |x| - \sqrt{1 - x^2} \\
                    |x| &= \sqrt{1 - x^2} \\
                    x^2 &= 1 - x^2 \\
                    2x^2 &= 1 \\
                    x^2 &= \frac{1}{2} \\
                    x &= \pm \frac{1}{\sqrt{2}}
                \end{align*}
            \item \text{Paridad:} La función es par, ya que $f(-x) = f(x)$ para todo $x$ en el dominio de $f$.
                \begin{align*}
                    f(-x) &= |-x| - \sqrt{1 - (-x)^2} \\
                    &= |x| - \sqrt{1 - x^2} \\
                    &= f(x)
                \end{align*}
            Para crecimiento y decrecimiento definamos un $x_{1}=-\frac{1}{\sqrt{2}} < x_2=0$
            \item \text{Crecimiento:} La función está creciendo en todo su dominio, ya que para cualquier par de números reales $a$ y $b$, si $a < b$, entonces $f(a) < f(b)$.
                \begin{align*}
                    f(x_1) &= |-1/\sqrt{2}| - \sqrt{1 - (-1/\sqrt{2})^2} \\
                    &= -1/\sqrt{2} - \sqrt{1 - 1/2} \\
                    &= -1/\sqrt{2} - \sqrt{1/2} \\
                    f(x_2) &= |0| - \sqrt{1 - 0^2} \\
                    &= 0 - \sqrt{1} \\
                    &= -1/\sqrt{2} - 1 < 0
                \end{align*}
            \item \text{Decrecimiento:} La función no está decreciendo en ninguna parte de su dominio.
            \item \text{Acotamiento:} La función no está acotada, ya que puede tomar cualquier valor real.
                \begin{align*}
                    \lim_{x \to 1} f(x) &= 0 \\
                    \lim_{x \to -1} f(x) &= 0
                \end{align*}
            \item \text{Signo:} La función es positiva para $-1 < x < 0$, negativa para $0 < x < 1$, y cero para $x = -1, 0, 1$.
        \end{itemize}
    \item[c)] $f(x) = \sqrt{1-\frac{2}{1 + x}}$
    \begin{itemize}
        \item \text{Dominio:} El dominio de la función es $x \geq 1$, ya que el denominador de la fracción no puede ser cero.
        \begin{align*}
            \text{Dom}(f) &= \{x \in \mathbb{R} : x \geq 1\} \\
            1 + x \neq 0 &\wedge 1-\frac{2}{1 + x} \geq 0 \\
            x \neq -1 &\wedge (1-\frac{2}{1 + x} \geq 0) \\
            x \neq -1 &\wedge \frac{1 + x - 2}{1 + x} \geq 0 \\
            x \neq -1 &\wedge \frac{x - 1}{1 + x} \geq 0 \\
            x \neq -1 &\wedge x - 1 \geq 0 \\
            x \neq -1 &\wedge x \geq 1 \\
            x &\geq 1
        \end{align*}
        \item \text{Recorrido:} El recorrido de la función son todos los reales positivos.
        \begin{align*}
            \text{Rec}(f) &= \{y \in \mathbb{R} : f(x) = y\} \\
            y &= \sqrt{1-\frac{2}{1 + x}} \\
            y^2 &= 1-\frac{2}{1 + x} \\
            y^2 &= \frac{1 + x - 2}{1 + x} \\
            y^2 &= \frac{x - 1}{1 + x} \\
            y^2(1 + x) &= x - 1 \\
            y^2 + xy^2 &= x - 1 \\
            x - xy^2 &= 1 - y^2 \\
            x(1 - y^2) &= 1 - y^2 \\
            x &= \frac{1 - y^2}{1 - y^2} \\
            x &= 1
        \end{align*}
    \end{itemize}
    \item[d)] $f(x) = \sqrt{x - 1}$
    \item[e)] $f(x) = \sqrt[3]{x - 1}$
    \item[f)] $f(x) = \frac{x^2}{1 + x^2}$
    \item[g)] $f(x) = \frac{x^2 - 1}{x + 1}$
    \item[h)] $f(x) = \frac{x^2}{x^2 - 1}$
    \item[i)] $f(x) = \frac{1}{|2x + 1|}$
    \begin{itemize}
        \item \text{Dominio:} El dominio de la función es $x \neq -\frac{1}{2}$, ya que el denominador de la fracción no puede ser cero.
        \begin{align*}
            \text{Dom}(f) &= \{x \in \mathbb{R} : x \neq -\frac{1}{2}\} \\
            &= (-\infty, -\frac{1}{2}) \cup (-\frac{1}{2}, \infty)
        \end{align*}
        \item \text{Recorrido:} El recorrido de la función son todos los reales positivos.
        \begin{align*}
            \text{Rec}(f) &= \{y \in \mathbb{R} : f(x) = y\} \\
            y &= \frac{1}{|2x + 1|} \\
            |2x + 1| &= \frac{1}{y} \\
            (2x + 1) &= \frac{1}{y} \\
            2x &= \frac{1}{y} - 1 \\
            x &= \frac{\frac{1}{y} - 1}{2} \\
            y &> 0 \\
            \text{Rec}(f) &= \{y \in \mathbb{R} : (0,\infty)\} 
        \end{align*}
        \item \text{Ceros:} La función no tiene ceros definidos.
        \begin{align*}
            f(x) &= \frac{1}{|2x + 1|} \\
            0 &= \frac{1}{|2x + 1|} \\
            0 &\neq 1 \\
        \end{align*}
        \item \text{Paridad:} La función es par, ya que $f(-x) = f(x)$ para todo $x$ en el dominio de $f$.
        \begin{align*}
            f(-x) &= \frac{1}{|2(-x) + 1|} \\
            &= \frac{1}{|-2x + 1|} \\
            &= \frac{1}{|2x + 1|} \\
            &= f(x)
        \end{align*}
        Definamos un $x_{1}=-1 < x_2=-0.75$ pertenecientes al dominio de la función en el intervalo $(-\infty, -\frac{1}{2})$
        \begin{align*}
            f(x_1) &= \frac{1}{|2(-1) + 1|} = \frac{1}{|-2 + 1|} = \frac{1}{1} = 1 \\
            f(x_2) &= \frac{1}{|2(-0.75) + 1|} = \frac{1}{|-1.5 + 1|} = \frac{1}{0.5} = 2 \\
            f(x_1) &< f(x_2)
        \end{align*}
        Definamos un $x_{1}=-0.3 < x_2=-0.25$ pertenecientes al dominio de la función en el intervalo $(-\frac{1}{2}, \infty)$
        \begin{align*}
            f(x_1) &= \frac{1}{|2(-0.3) + 1|} = \frac{1}{|-0.6+1|} = \frac{1}{0.4} = 2.5 \\
            f(x_2) &= \frac{1}{|2(-0.25) + 1|} = \frac{1}{|-0.5 + 1|} = \frac{1}{0.5} = 2 \\
            f(x_1) &> f(x_2)
        \end{align*}
        \item \text{Crecimiento:} Esta función se encuentra en crecimiento en el intervalo $(-\infty, -\frac{1}{2})$.
        \item \text{Decrecimiento:} Esta función se encuentra en decrecimiento en el intervalo $(-\frac{1}{2}, \infty)$.
        \item \text{Acotamiento:} La función esta acotada inferiormente por 0 y no esta acotada superiormente.
        \begin{align*}
            \lim_{x \to -\frac{1}{2}} f(x) &= \infty &&\text{ Limite Superior}\\
            \lim_{x \to \infty} f(x) &= 0 &&\text{ Limite Inferior}
        \end{align*}
        \item \text{Signo:} La función es positiva en todo su dominio.
    \end{itemize}
    \item[j)] $f(x) = \frac{x^2 + x}{x^2 - x}$
    \item[k)] $f(x) = \sqrt{|x - 1|} - 1$
    \item[l)] $f(x) = \frac{x + 1}{1 + x^4}$
    \item[m)] $f(x) = 1 - \sqrt{1 - x^2}$
\end{itemize}

\subsection*{\textbf{Ejercicio 8:}}
Dados dos conjuntos A y B, determine si las siguientes son funciones y si son inyectivas, sobreyectivas o biyectivas: 
\begin{itemize}
    \item[a)] $\pi_A : A \times B \rightarrow A$, tal que $(a,b) \mapsto \pi_A(a,b) = a$.
    \begin{itemize}
        \item Inyectividad: La función no es inyectiva, ya que para un par ordenado de elementos, como lo puede ser (a,b) y para otro par ordenado, dígase (a,c), comparten la misma imagen, en este caso a.
        \item Sobreyectividad: La función es sobreyectiva, ya que cualquier elemento del conjunto $A$ existe como imagen de al menos un par ordenado del conjunto $A \times B$. 
        \item Biyectividad: La función no es biyectiva, ya que no es inyectiva.
    \end{itemize}
    \item[b)] $d_A : A \rightarrow A \times B$, tal que $a \mapsto d_A(a) = (a,a)$.
    \item[c)] $\tau : A \times B \rightarrow B \times A$, tal que $(a,b) \mapsto \tau(a,b) = (b,a)$.
\end{itemize}

\subsection*{\textbf{Ejercicio 9:}}
Decida si las siguientes funciones son invertibles, si su respuesta es afirmativa
encuentre la función inversa, en cambio si es negativa haga las restricciones necesarias para
que sea invertible y defina formalmente la inversa:
\begin{itemize}
    \item[a)] $f(x) = \frac{1}{1 - x}$
    \item[b)] $f(x) = 2x^2 - 2x - 4$
    \item[c)] $f(x) = \frac{x^2 - 1}{x^2 + 2x - 3}$
    \item[d)] $f(x) = x^2 - 5x + 6$ 
    \\
    Para este caso la función no es inyectiva, ya que para $x = 2$ y $x = 3$ se tiene que $f(2) = f(3) = 0$. Por lo tanto es necesario aplicar restricciones.\\
    Ya que esta función representa una parabola, hemos de definir la restricción tomando en cuenta el vértice de esta misma, el cual se calcula con la formula $x = \frac{-b}{2a}$.
    \begin{itemize}
        \item Vértice: $x = \frac{5}{2} = 2.5$
        \item Podemos restringir el dominio de la función a $x \in (-\infty, 2.5] \cup [2.5, +\infty)$, para este caso se trabajara con la restricción $x \in (-\infty, 2.5]$.
        \item $f: [2.5, \infty) \rightarrow \mathbb{R}, \text{ definida por } f(x) = x^2 - 5x + 6$.
        \item Ahora con la función restringida, encontraremos la inversa invirtiendo la misma:
        \begin{align*}
            y &= x^2 - 5x + 6 \\
            x &= y^2 - 5y + 6 \\
            x - 6 + 6.25 &= y^2 - 5y + 6.25 \\
            x - 6 + 6.25 &= (y - 2.5)^2 \\
            \pm\sqrt{x - 0.25} &= y - 2.5 \\
            y &= \pm \sqrt{x - 0.25} + 2.5 \\
            f^{-1}(x) &= \sqrt{x - 0.25} + 2.5
        \end{align*}
        Como restringimos la función solo tomamos la parte positiva de la raíz cuadrada.
        \item La función inversa es $f^{-1}: [0.25, \infty) \rightarrow [2.5, \infty)$. 
        \\Definida por: $f^{-1}(x) = \sqrt{x - 0.25} + 2.5$.
    \end{itemize}
    \item[e)] $g(x) = \sqrt{(x - 1)(x + 2)}$
    \item[f)] $f(x) = \frac{x^2}{x^2 - 1}$
    \item[g)] $f(x) = \frac{3}{x^2 + 5x + 6}$
    \\
    Para este caso la función no es biyectiva ya que para algunos valores la función se indetermina. Por lo tanto es necesario aplicar restricciones.\\
    \begin{itemize}
        \item Para este caso se tiene que $x^2 + 5x + 6 > 0$.
        \begin{align*}
            x^2 + 5x + 6 &> 0 \\
            (x + 2)(x + 3) &> 0 \\
            x = -2 &\wedge x=-3 
        \end{align*}
        \item Se puede restringir el dominio de la función a \\ \[x \in (-\infty, -3) \cup (-3,-2) \cup (-2, +\infty) \] \\Para este caso se trabajara con la restricción $x \in (-\infty, -3)$.
        \item $f:(-\infty, -3) \rightarrow \mathbb{R}, \text{ definida por } f(x) = \frac{3}{x^2 + 5x + 6}$.
        \item Ahora con la función restringida, encontraremos la inversa invirtiendo la misma:
        \begin{align*}
            y &= \frac{3}{x^2 + 5x + 6} \\
            x &= \frac{3}{y^2+5x+6} \\
            x(y^2+5x+6) &= 3 \\
            xy^2 + 5xy^2 + 6x &= 3 \\
            xy^2 + 5xy^2 + 6x - 3 &= 0 \\
            y &= \frac{-5x \pm \sqrt{(5x)^2 - 4(x)(6x - 3)}}{2(x)} \\
            y &= \frac{-5x \pm \sqrt{25x^2 - 24x + 12}}{2(x)} \\
            y &= \frac{-5x \pm \sqrt{25x^2 - 24x + 12}}{2(x)} \\
        \end{align*}
        Como restringimos la función solo tomamos la parte negativa de la raíz cuadrada.
        \item La función inversa es $f^{-1}: \mathbb{R} \rightarrow (-\infty, -3)$.
        \\Definida por: $f^{-1}(x) = \frac{-5x - \sqrt{25x^2 - 24x + 12}}{2(x)}$.
    \end{itemize}
    \item[h)] $f(x) = \frac{3x - 3}{x^2 - x - 56}$
    \item[i)] $f(x) = \sqrt{x^2 - 9}$
    \item[j)] $f(x) = \sqrt{16 - x^2}$
    \\
    Para este caso la función no es biyectiva ya que para algunos valores la función se indetermina. Por lo tanto es necesario aplicar restricciones.\\
    \begin{itemize}
        \item Para este caso se tiene que $16 - x^2 \geq 0$.
        \begin{align*}
            16 - x^2 &\geq 0 \\
            16 &\geq x^2 \\
            x^2 &\leq 16 \\
            x &\in [-4, 4]
        \end{align*}
        \item Se puede restringir el dominio de la función a \\ \[x \in [-4, 4] \] \\Para este caso se trabajara con la restricción $x \in [-4, 4]$.
        \item $f:[-4, 4] \rightarrow \mathbb{R}, \text{ definida por } f(x) = \sqrt{16 - x^2}$.
        \item Ahora con la función restringida, encontraremos la inversa invirtiendo la misma:
        \begin{align*}
            y &= \sqrt{16 - x^2} \\
            x &= \sqrt{16 - y^2} \\
            x^2 &= 16 - y^2 \\
            x^2 + y^2 &= 16 \\
            y^2 &= 16 - x^2 \\
            y &= \sqrt{16 - x^2} \\
        \end{align*}
        \item La función inversa es $f^{-1}: \mathbb{R} \rightarrow [-4, 4]$.
        \\Definida por: $f^{-1}(x) = \sqrt{16 - x^2}$.
    \end{itemize}
    \item[k)] $f(x) = \frac{6 - 11x}{\sqrt{-3x - 15}}$
    \\
    Para este caso la función no es biyectiva ya que para algunos valores la función se indetermina. Por lo tanto es necesario aplicar restricciones.\\
    \begin{itemize}
        \item Para este caso se tiene que $-3x - 15 \geq 0$.
        \begin{align*}
            -3x - 15 &\geq 0 \\
            -3x &\geq 15 \\
            x &\leq -5
        \end{align*}
        \item Se puede restringir el dominio de la función a \\ \[x \in (-\infty, -5] \] \\Para este caso se trabajara con la restricción $x \in (-\infty, -5]$.
        \item $f:(-\infty, -5] \rightarrow \mathbb{R}, \text{ definida por } f(x) = 6 - 11x\sqrt{-3x - 15}$.
        \item Ahora con la función restringida, encontraremos la inversa invirtiendo la misma:
        \begin{align*}
            y &= \frac{6 - 11x}{\sqrt{-3x - 15}} \\
            x &= \frac{6 - 11y}{\sqrt{-3y - 15}} \\
            x^2 &= \frac{(6 - 11y)^2}{-3y - 15} \\
            x^2(-3y - 15) &= (6 - 11y)^2 \\
            -3x^2y - 15x^2 &= 36 - 132y + 121y^2 \\
            121y^2 - 132y + 36 + 3x^2y + 15x^2 &= 0 \\
            y &= \frac{-(-132+3x) \pm \sqrt{(-132+3x)^2 - 4(121)(36 + 15x^2)}}{2(121)} \\
            y &= \frac{132-3x \pm \sqrt{(132-3x)^2 - 4(121)(36 + 15x^2)}}{242} \\
            y &= \frac{132-3x \pm \sqrt{17424 - 792x + 9x^2 - 17424 - 7260x^2}}{242} \\
        \end{align*}
        \item La función inversa es $f^{-1}: \mathbb{R} \rightarrow (-\infty, -5]$.
        \item Definida por: $f^{-1}(x) = \sqrt{\frac{12x - x^2 + 36}{363(y + 5)}}$.
    \end{itemize}
\end{itemize}
\newpage
\subsection*{Ejercicio décimas}
Demostrar que:
\[x^{\ln y - \ln z} \cdot y^{\ln z - \ln x} \cdot z^{\ln x - \ln y} = 1\]
\[\ln (x^{\ln y - \ln z} \cdot y^{\ln z - \ln x} \cdot z^{\ln x - \ln y}) = \ln 1\]
\[\ln (x^{\ln y - \ln z}) + \ln (y^{\ln z - \ln x}) + \ln (z^{\ln x - \ln y}) = \ln 1\]
\[(\ln y - \ln z)\ln x + (\ln z - \ln x)\ln y + (\ln x - \ln y)\ln z = 0\]
\[\ln y \ln x - \ln z \ln x + \ln z \ln y - \ln x \ln y + \ln x \ln z - \ln y \ln z = 0\]
\[\ln y \ln x - \ln x \ln y + \ln z \ln y - \ln y \ln z + \ln x \ln z - \ln z \ln x = 0\]
\[\ln y \ln x - \ln y \ln x + \ln z \ln y - \ln y \ln z + \ln x \ln z - \ln z \ln x = 0\]
\[0 = 0\]

Por lo tanto, se concluye que la igualdad es verdadera para cualquier valor de $x$, $y$ y $z$.

\end{document}