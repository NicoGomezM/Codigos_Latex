\documentclass{templateNote}

\usepackage{enumitem}

\begin{document}

\imagenlogoU{img/logoUBB.png}
\linklogoU{https://www.ubiobio.cl/w/}
% \linkQRDoc{https://github.com/MarceloPazPezo/MyRepo/tree/main/Icinf}
\titulo{Tarea 4}
\asignatura{Gestión Estratégica}
\autor{
    Nicolás Gómez Morgado\\
    Marcelo Paz Pezo\\
    Bastian Rodríguez Campusano
}
% \vDoc{1.0.0}

% Metadatos del PDF
\title{[\asignatura]-\titulo}
\author{
    \autor
}
\portada
\margenes % Crear márgenes

\subsection*{Desarrollar lo siguiente:}
\begin{enumerate}
    \item Plantear un Plan Estratégico para que una empresa tecnológica abarque más territorios (especifique cuál empresa). 
\end{enumerate} 

\begin{figure}[H]
    \centering
    \textbf{Plan Estratégico de Expansión Territorial para PC Factory.} \\
    \includegraphics[height=1cm]{img/PC_Factory_899_bef58adaa2.png}
\end{figure}

\begin{enumerate}[label=\alph*)]
    \item \textbf{Determinar donde estamos (FODA).} \\
    \underline{\textbf{Análisis Externo:}}
    \begin{itemize}
        \item \textbf{Oportunidades:}
        \begin{itemize}
            \item Crecimiento de la demanda de dispositivos electrónicos en mercados emergentes.
            \item Asociaciones con empresas locales de telecomunicaciones y electronica.
            \item Expansión en áreas de dispositivos exclusivos.
        \end{itemize}
        \item \textbf{Amenazas:} 
        \begin{itemize}
            \item Regulaciones locales restrictivas.
            \item Competencia con multitiendas y electrónicas internacionales.
            \item Productos sustitutos y piratería.
        \end{itemize}
    \end{itemize}
    \underline{\textbf{Análisis Interno:}}
    \begin{itemize}
        \item \textbf{Fortalezas:} 
        \begin{itemize}
            \item Amplia base de usuarios y distribuidora reconocida nacionalmente.
            \item Asociación tecnológica robusta y confiable.
            \item Amplio catálogo de dispositivos y elementos de investigación.
        \end{itemize}
        \item \textbf{Debilidades:}
        \begin{itemize}
            \item Alta competencia en algunos mercados.
            \item Desafíos en la adaptación a diferentes normativas locales.
            \item Dependencia de asociaciones con negocios extranjeros.
        \end{itemize}
    \end{itemize}

    \newpage    
    \item \textbf{Decidir donde queremos llegar (Misión, Visión y Objetivos).} 
    \begin{itemize}
        \item \textbf{Misión:} Proveer productos tecnológicos innovadores y servicios de alta calidad que mejoren la vida de nuestros clientes, expandiendo nuestra presencia para satisfacer la demanda creciente en nuevos mercados.
        \item \textbf{Vision:} Ser la empresa líder en tecnología en América Latina, reconocida por nuestra innovación, calidad de servicio y compromiso con el desarrollo tecnológico de la región.
        \item \textbf{Objetivos:}
        \begin{itemize}
            \item Expandir la presencia de PC Factory en al menos tres nuevos países de América Latina en los próximos cinco años.
            \item Incrementar las ventas internacionales en un 30\% anual durante los primeros tres años de expansión.
            \item Establecer alianzas estratégicas con al menos dos empresas tecnológicas locales en cada nuevo mercado.
        \end{itemize}
    \end{itemize}

    \item \textbf{Donde queremos ir (Estrategias).} \\
    \noindent Se plantea una expansion territorial de la empresa, por lo que la estrategia de liderazgo en costos puede no ser la mejor opción debido a la reducción en ganancias que implicaría, la diferenciación tampoco entraría a ser una opción viable ya que esta empresa no produce sus propios productos, mas bien se comporta como distribuidora de otras empresas mas grandes. Por lo que la estrategia de enfoque sería la más adecuada, aplicando una estrategia de mejora continua en los servicios ofrecidos y asegurando la calidad de los productos por lo menos hasta tener un servicio de calidad reconocible en los nuevos mercados a los que la empresa busca adentrarse.

    \item \textbf{Cómo llegaremos (Planes de Acción).} \\
    \noindent Para lograr la expansión territorial de PC Factory, se deben seguir los siguientes planes de acción:
    \begin{itemize}
        \item \textbf{Plan de Marketing:} Realizar estudios de mercado en los países a los que se busca expandir, para conocer las necesidades y preferencias de los consumidores locales.
        \item \textbf{Plan de Alianzas Estratégicas:} Establecer alianzas con empresas locales de tecnología, para facilitar la entrada en nuevos mercados y adaptarse a las normativas locales.
        \item \textbf{Plan de Capacitación:} Capacitar al personal en las nuevas regulaciones y normativas de los países a los que se busca expandir, para asegurar el cumplimiento de las leyes locales.
        \item \textbf{Plan de Logística:} Implementar un sistema de logística eficiente, para garantizar la entrega oportuna de los productos en los nuevos mercados.
    \end{itemize}
    
\end{enumerate}

\newpage
\begin{enumerate}
    \item[2.] Cuáles serían los planes tácticos (2) y operativos (2 ) a implementar?
\end{enumerate} 

\begin{itemize}
    \item \textbf{Planes Tácticos:}
    \begin{enumerate}
        \item \textbf{Desarrollo de la Plataforma de E-commerce} El objetivo es optimizar y personalizar la plataforma de e-commerce de PC Factory para los nuevos mercados. Las acciones incluyen adaptar la plataforma para soportar múltiples idiomas y monedas, implementar métodos de pago locales y añadir funcionalidades de marketing digital como recomendaciones personalizadas y promociones segmentadas. El equipo de TI y Marketing será responsable de estas acciones, con un plazo de 6 meses.
        \item \textbf{Campaña de Marketing Digital} El objetivo es aumentar la visibilidad de la marca y atraer nuevos clientes en los mercados objetivo. Las acciones incluyen realizar campañas específicas en redes sociales, colaborar con influencers locales, e implementar estrategias de SEO y SEM. Estas acciones serán responsabilidad del equipo de Marketing y Relaciones Públicas y se llevarán a cabo inicialmente en 6 meses, continuando durante un año.
    \end{enumerate}
    \item \textbf{Planes Operativos:}
    \begin{enumerate}
        \item \textbf{Establecimiento de Centros de Distribución} El objetivo es garantizar una logística eficiente y tiempos de entrega rápidos. Las acciones incluyen identificar ubicaciones estratégicas para los centros de distribución, establecer contratos con proveedores logísticos locales, e implementar sistemas de gestión de inventarios y control de calidad. El equipo de Logística y Operaciones será responsable, con un plazo de 12 meses para su implementación.
        \item \textbf{Capacitación del Personal Local} El objetivo es asegurar que el personal local esté bien capacitado y alineado con los estándares de calidad de PC Factory. Las acciones incluyen desarrollar programas de capacitación intensivos en áreas clave, realizar talleres de formación continua y establecer un sistema de evaluación de desempeño. El Departamento de Recursos Humanos será responsable de estas acciones, con un plazo inicial de 6 meses y continuo en el tiempo.
    \end{enumerate}
\end{itemize}

\end{document}