\documentclass{templateNote}
\usepackage{tcolorbox}
\usepackage{tabularx}
\usepackage{hyperref}
\usepackage{amsmath}
\usepackage{amssymb}
\usepackage{pdflscape}
\usepackage{tikz}
\usepackage{soul}
\usepackage{media9}
\usepackage{adjustbox}
\usepackage{pdfpages}
\usepackage{enumitem}
% \usepackage[spanish,es-noquoting]{babel}

\begin{document}
\linklogoU{https://www.ubiobio.cl/w/}
\linklogoD{https://github.com/NicoGomezM}
\imagenlogoU{img/logo-ubb-txt-face.png}
\imagenlogoD{img/logoNGMFormal_sinF.png}
\titulo{Tarea 3}
\asignatura{Gestión Estratégica}
\autor{
    Nicolás \textsc{Gómez Morgado}
}

\portada
\margenes

\begin{enumerate}
    \item Desarrollar 3 estrategias competitivas para que la UBB logre el aumento de matriculas de alumnos nuevos el año 2025, argumentando y explicando cuáles son las 3 variables más importantes a considerar en el análisis. \\\\
    \noindent Para desarrollar estrategias competitivas en la toma de decisiones que aumenten la matrícula de nuevos alumnos en la Universidad del Bío-Bío, es necesario realizar un análisis FODA. Este análisis permitirá identificar variables específicas y centradas en el entorno interno y externo de la institución.\\
    \begin{center}
        \textbf{Elementos internos del análisis} \\
        \vspace{0.5cm} % Ajusta este valor para cambiar la cantidad de espacio
        \begin{tabularx}{\textwidth}{|X|X|}
            \hline
            \textbf{Fortalezas} & \textbf{Debilidades} \\
            \hline
            \begin{itemize}[leftmargin=*]
                \item Su tamaño y locaciones en dos regiones del país
                \item Recorrido histórico y prestigio en la región
                \item Oferta académica variada
                \item Cercanía con empresas y organizaciones
            \end{itemize} 
            & 
            \begin{itemize}
                \item Descontento estudiantil en mas de una oportunidad
                \item Seguridad del campus y alrededores
                \item Publicidad negativa
                \item Desconocimiento de la institución misma
            \end{itemize}
            \\
            \hline
        \end{tabularx}   

        \vspace{0.5cm} 

        \textbf{Elementos externos del análisis} 

        \vspace{0.5cm} 
        \begin{tabularx}{\textwidth}{|X|X|}
            \hline
            \textbf{Oportunidades} & \textbf{Amenazas} \\
            \hline
            \begin{itemize}[leftmargin=*]
                \item Multitud de nuevos estudiantes que buscan educación superior
                \item Crecimiento de la población en la región
                \item Aumento de la demanda de educación superior
                \item Nuevas ofertas academicas y tecnologicas en el mercado
            \end{itemize} 
            & 
            \begin{itemize}
                \item Competencia con otras universidades
                \item Aumento de la exigencia de calidad en la educación
                \item Cambio en las politicas educativas del estado
                \item Dependencia del financiamiento estatal
            \end{itemize}
            \\
            \hline
        \end{tabularx}
    \end{center}
    \newpage
    Una vez identificadas estas variables, nos enfocaremos especialmente en las fortalezas, debilidades y oportunidades. Estos son los factores sobre los cuales la institución puede influir, considerando especialmente aquellos que lógicamente afectarán más el nivel de matrículas en los próximos años.
    \begin{enumerate}
        \item \textbf{Estrategia 1:} \textit{Aumentar la oferta académica} \\
        \noindent La Universidad del Bío-Bío cuenta con una oferta académica variada, sin embargo, se puede mejorar y aumentar la cantidad de carreras y programas que se ofrecen, de esta manera se puede atraer a un mayor número de estudiantes que buscan una carrera en específico.\\
        \textbf{Variables a considerar:} 
        \begin{itemize}
            \item Oferta académica
            \item Demanda de educación superior
            \item Crecimiento de la población en la región
        \end{itemize}
        \item \textbf{Estrategia 2:} \textit{Mejorar la publicidad y la imagen de la institución} \\
        \noindent La publicidad negativa y el desconocimiento de la institución misma son factores que pueden afectar la decisión de los estudiantes de elegir la UBB como su casa de estudios. Por lo tanto, es necesario mejorar la imagen de la institución y realizar una campaña publicitaria que muestre ventajas y elementos de aprendizaje que ofrece la universidad.\\
        \textbf{Variables a considerar:} 
        \begin{itemize}
            \item Publicidad
            \item Desconocimiento de la institución
            \item Crecimiento de la población en la región
        \end{itemize}
        \item \textbf{Estrategia 3:} \textit{Establecer alianzas con empresas y organizaciones} \\
        \noindent La cercanía con empresas y organizaciones es una fortaleza de la UBB, por lo que se puede aprovechar esta oportunidad para establecer alianzas que permitan a los estudiantes realizar prácticas profesionales y acceder a oportunidades laborales una vez que se gradúen.\\
        \textbf{Variables a considerar:} 
        \begin{itemize}
            \item Cercanía con empresas y organizaciones
            \item Demanda de educación superior
            \item Aumento de la exigencia de calidad en la educación
        \end{itemize}
    \end{enumerate}

    En conclusion, se pueden desarrollar muchas estrategias competitivas para aumentar o fomentar la matricula de alumnos nuevos en la Universidad del Bío-Bío, sin embargo, es importante considerar las variables más importantes para la toma de desiciones, ya que estas pueden afectar directamente el éxito o fracaso de la estrategia.\\
\end{enumerate}



\end{document}