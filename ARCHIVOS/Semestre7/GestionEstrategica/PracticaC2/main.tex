\documentclass{templateNote}

\begin{document}

\imagenlogoU{img/LogoElNube.png}
\linklogoU{https://github.com/MarceloPazPezo}
\imagenlogoD{img/LogoNGM.png}
\linklogoD{https://github.com/NicoGomezM}
% \linkQRDoc{https://github.com/MarceloPazPezo/MyRepo/tree/main/Icinf}
\titulo{Certamen 2}
\asignatura{Gestión Estratégica}
\autor{
Nicolás Gómez Morgado\\
}
% \vDoc{1.0.0}

% Metadatos del PDF
\title{[\asignatura]-\titulo}
\author{
    \autor
}
\portada
\margenes % Crear márgenes

\subsection*{ITEM I (3 Ptos. c/u)}
\noindent Responda las siguientes aseveraciones, marcando con un “V” si la pregunta es verdadera o con una “F” si es 
falsa. Justifique las falsas.

\begin{enumerate}
    \item (\textcolor{red}{F}) El triangulo de la estrategia tiene: el análisis FODA, implementación y elección de la estrategia.
    \\ Falso, el triangulo de la estrategia tiene: análisis estratégico, elección de la estrategia y implementación de la estrategia.
    \item (\textcolor{green}{V}) El análisis estratégico implica analizar el entorno, los recurso y las expectativas.
    \\ 
    \item (\textcolor{red}{F}) La Misión y la Vision, representan el presente de una empresa.
    \\ La misión si representa el presente pero la vision representa el futuro de una empresa.
    \item (\textcolor{green}{V}) La vision estratégica de una empresa, es el mapa de rutas del futuro.
    \\ 
    \item (\textcolor{red}{F}) Las características de los objetivos son: jerárquicos, cuantitativos, cualitativos y consistentes.
    \\ Las características de los objetivos son: jerárquicos, cuantitativos, realistas y consistentes.
    \item (\textcolor{green}{V}) El objetivo financiero de una empresa es mas importante que el objetivo estratégico.
    \\ 
    \item (\textcolor{green}{V}) Los objetivos de largo plazo impulsan a ponderar las acciones de hoy en la rentabilidad a largo plazo.
    \\ 
    \item (\textcolor{green}{V}/\textcolor{red}{F}) El control aplicado a un plan estrategico, implica conservar el plan original.
    \\  
    \item (\textcolor{red}{F}) Controlar un proceso es mas importante que hacer un plan estratégico.
    \\ A falta de planificación, no hay nada sobre que ejercer un control.
    \item (\textcolor{green}{V}) Los tipos de control son: cuantitativos, de calidad, de tiempo y monetarios.
    \\ 
\end{enumerate}

\newpage
\subsection*{ITEM II (20 Ptos.)}
\noindent\textbf{Desarrollo:} Explique con sus palabras las etapas del control aplicadas al desarrollo de un software. \\

\noindent El control en el desarrollo de software se basa en un proceso cíclico y sistemático. Primero, se establecen los objetivos y metas del proyecto, que actúan como la base de control contra la cual se medirá el progreso. Estos objetivos deben ser claros y medibles para asegurar que se pueda evaluar adecuadamente el avance del proyecto.\\\\
A continuación, se verifica el trabajo realizado mediante la recolección de datos y métricas, como el progreso de las tareas y la calidad del código desarrollado. Esta información se utiliza para juzgar si el trabajo está alineado con lo planeado.\\\\
Luego, se comparan los datos recogidos con los objetivos iniciales para identificar cualquier discrepancia. Esta comparación permite evaluar si el proyecto está siguiendo el plan y detectar desviaciones entre lo programado y lo realizado.\\\\
Finalmente, se analizan las desviaciones identificadas para entender sus causas y se toman medidas correctivas. Estas acciones buscan ajustar el plan del proyecto y prevenir futuras desviaciones, mejorando continuamente el proceso de desarrollo del software.


\end{document}