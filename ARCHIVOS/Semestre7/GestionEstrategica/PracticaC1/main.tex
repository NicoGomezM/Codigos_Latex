\documentclass{templateNote}

\begin{document}

\imagenlogoU{img/LogoElNube.png}
\linklogoU{https://github.com/MarceloPazPezo}
\imagenlogoD{img/LogoNGM.png}
\linklogoD{https://github.com/NicoGomezM}
\linkQRDoc{https://github.com/MarceloPazPezo/MyRepo/tree/main/Icinf}
\titulo{Formativa 1}
\asignatura{Gestión Estratégica}
\autor{
Nicolás Gómez\\
Marcelo Paz
}
\vDoc{1.0.0}

% Metadatos del PDF
\title{[\asignatura]-\titulo}
\author{
    \autor
}
\portada
\margenes % Crear márgenes

\section{Certamen 1 - 2022} 

\subsection*{ITEM I (3 Ptos. c/u)}
Responda las siguientes aseveraciones, marcando con un “V” si la pregunta es verdadera o con una “F” si es 
falsa. Justifique las falsas.

\begin{enumerate}
    \item (\textcolor{green}{V}/\textcolor{red}{F}) El concepto de Estrategia implica un conjunto de decisiones y acciones que se llevan a cabo para lograr un determinado objetivo. 
    \\ Verdadero
    \item (\textcolor{green}{V}/\textcolor{red}{F}) La Estrategia proyectada en una empresa es lo mismo que la Estrategia realizada.
    \\ Falso, la estrategia proyectada se basa en planes futuros,mientras que la realizada esta mas enfocada en basarse en situaciones pasadas.
    \item (\textcolor{green}{V}/\textcolor{red}{F}) Las características de la Administración Estratégica implican un proceso dinámico y participativo, compromiso de los jefes y transversal, entre otras.
    \\ Verdadero
    \item (\textcolor{green}{V}/\textcolor{red}{F}) El proceso de Administración Estratégica incluye la selección de la misión y objetivos principales, análisis del ambiente competitivo externo y del ambiente operativo interno.
    \\ Verdadero
    \item (\textcolor{green}{V}/\textcolor{red}{F}) El triángulo de la estratégica contiene: el análisis estratégico, la elección y la implementación de la estrategia.
    \\ 
    \item (\textcolor{green}{V}/\textcolor{red}{F}) Las consecuencias de rivalidad entre los competidores existentes son disminución de precios, aumento de costos y reducción de rentabilidad.
    \\ 
    \item (\textcolor{green}{V}/\textcolor{red}{F}) Los factores críticos de éxito son elementos que ayudan a la empresa a lograr los objetivos. 
    \\ 
    \item (\textcolor{green}{V}/\textcolor{red}{F}) Los requisitos para el desarrollo de una estrategia son: Conocimientos, Capacidad para integración, Imaginación y lógica para elegir entre alternativas, entre otros.
    \\  
    \item (\textcolor{green}{V}/\textcolor{red}{F}) Al hacer la exploración ambiental y los 8 ambientes críticos de una empresa forestal se pueden concluir los mismos análisis que para la UBB. 
    \\ Falso, debido al rubro de cada empresa, los ambientes críticos serán distintos.
    \item (\textcolor{green}{V}/\textcolor{red}{F}) Una Unidad Estratégica de Negocios pertenece a una empresa productiva y tiene sus propios competidores.
    \\ 
\end{enumerate}

\newpage
\subsection*{ITEM II (5 Ptos. c/u)}
Explique con un ejemplo, cada uno de los siguientes conceptos: 

\begin{enumerate}
    \item Unidad Estratégica de Negocios
    \item Matriz de crecimiento-participación
    \item Ciclo de vida del producto
    \item Ventaja competitiva
    \item Fuerzas competitivas de Porter
\end{enumerate}

\newpage
\section{Certamen 1 - 2023} 

\subsection*{ITEM I (3 Ptos. c/u)}
Responda las siguientes aseveraciones, marcando con un “V” si la pregunta es verdadera o con una “F” si es falsa. Justifique las falsas.   
\begin{enumerate}
    \item (\textcolor{green}{V}/\textcolor{red}{F}) El producto Estrella, le indica a una empresa que el mercado lo acepta y por lo tanto, éste le reporta una alta participación de mercado.
    \item (\textcolor{green}{V}/\textcolor{red}{F}) Cuando un producto está en la etapa de introducción en el mercado, la empresa prácticamente no tiene que invertir en publicidad.
    \item (\textcolor{green}{V}/\textcolor{red}{F}) El análisis FODA le sirve a las empresas para determinar el punto máximo de ganancias.
    \item (\textcolor{green}{V}/\textcolor{red}{F}) Las áreas funcionales de una empresa, son Finanzas \& Contabilidad, RRHH \& Administración, Producción y Comercialización \& Marketing.
    \item (\textcolor{green}{V}/\textcolor{red}{F}) El proceso de la administración estratégica contiene: selección de misión y objetivos, formulación de estrategias e implementación de la estrategia.
    \item (\textcolor{green}{V}/\textcolor{red}{F}) El Modelo resumen de los elementos de la Dirección Estratégica son: Análisis Estratégico, Elección Estratégica e Implantación de la Estrategia.
    \item (\textcolor{green}{V}/\textcolor{red}{F}) El modelo de negocio de una empresa involucra sólo al entorno interno.
    \item (\textcolor{green}{V}/\textcolor{red}{F}) Un proceso de formulación de estrategias para una empresa, implica considerar el corto plazo.
    \item (\textcolor{green}{V}/\textcolor{red}{F}) El Análisis PESTA le sirve a una empresa para determinar el punto máximo de ganancias.
    \item (\textcolor{green}{V}/\textcolor{red}{F}) Una estrategia competitiva implica que la empresa tiene las máximas ganancias económicas del mercado.
\end{enumerate}

\newpage
\subsection*{ITEM II (5 Ptos. c/u)}
Explique con un ejemplo, cada uno de los siguientes conceptos:
\begin{enumerate}
    \item Estrategia Competitiva
    \item Segmentación del mercado
    \item Fuerzas competitivas de Porter
    \item Entorno de una empresa
    \item Análisis Estratégico
\end{enumerate}

\end{document}