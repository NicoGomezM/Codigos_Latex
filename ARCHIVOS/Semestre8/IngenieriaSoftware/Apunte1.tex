\documentclass{templateNote}
\usepackage{tcolorbox}
\usepackage{hyperref}
\usepackage{amsmath}
\usepackage{amssymb}
\usepackage{soul}
\usepackage{circuitikz}
\usepackage{parskip} % Para evitar la indentation y mejorar la separación entre párrafos


\begin{document}
\imagenlogoU{img/logoNGMFormal_sinF.png}
\linklogoU{https://github.com/NicoGomezM} 
% \imagenlogoD{img/logo-ubb-txt-face.png} 
\titulo{Apunte 1}
\asignatura{Ingeniería de Software}
\autor{
    \indent
    Nicolás {Gómez Morgado}
}


\portada
\margenes 
\tableofcontents
\newpage

\section{Importante}

\textbf{Certamen}
\begin{itemize}
    \item Tipos de software.
    \item Técnicas de recopilación de información.
\end{itemize}

\textbf{Conceptos}
\begin{itemize}
    \item \textbf{Proyecto:} Conjunto de actividades interrelacionadas que se realizan para alcanzar un objetivo.
\end{itemize}

\newpage
\section{Programa v/s Software}
Para esta asignatura y para el ámbito laborar futuro vamos a entender que:
\begin{itemize}
    \item \textbf{Programa:} 
    \begin{itemize}
        \item Es un resultado, un producto entregadle.
        \item Resultado de una de las etapas del \textbf{desarrollo de software}.
    \end{itemize}
    \item \textbf{Software:}
    \begin{itemize}
        \item Todo lo que se remunera (lo que el cliente valora).
        \item Conjunto de componentes (programas, datos, documentación, etc).
    \end{itemize}
\end{itemize}

\section{Desarrollo de software}
\subsection{Proyecto de desarrollo/manutención de software}
\begin{itemize}
    \item Presenta fecha de inicio y término.
    \item Presenta objetivos determinados loe cuales se han de cumplir antes de la fecha de termino asumida.
    \item Construcción se ajusta a las restricciones del cliente, su presupuesto, tiempo y personal.
\end{itemize}

\subsection{Proceso de desarrollo de software}
\begin{itemize}
    \item Ciclo de vida común para la creación de proyectos.
    \item Procesos de creación estandarizados ISO IEC IEEE:
    \begin{itemize}
        \item Procesos de acuerdos.
        \item Procesos de habilitación de proyectos organizacionales.
        \item Procesos de gestión técnica.
        \item Procesos técnicos.
    \end{itemize} 
\end{itemize}
La elección del proceso de desarrollo de software depende de:
\begin{itemize}
    \item \textbf{El cliente:} Que tan claro tiene el objetivo y que tanto conocimiento tiene respecto al manejo y desarrollo de software.
    \item \textbf{La complejidad del proyecto:} Que tan complejo es el proyecto y como se han de abordar los problemas.
\end{itemize}

\section{Métodos de análisis y diseño de software}
\subsection{Procesos técnicos ciclo de vida Iso/Iec 15289}

\begin{itemize}
    \item Proceso de análisis comercial o misión.
    \item Proceso de definición de requisitos y de las partes interesadas.
    \item Proceso de Definición de Requisitos del sistema software
    \item Proceso de Definición de la Arquitectura
    \item Proceso de Definición del diseño
    \item Proceso de Análisis del Sistema
    \item Proceso de Implementación
    \item Proceso de Integración
    \item Proceso de Verificación
    \item Proceso de Transición
    \item Proceso de Validación
    \item Proceso de Operación
    \item Proceso de Mantenimiento
    \item Proceso de Disposición
\end{itemize}

\subsection{Métodos / modelos}

\begin{itemize}
    \item CASCADA
    \item IE (Ingeniería de la empresa)
    \item SSE (Sistemas de software embebido)
    \item CASE METHOD
    \item Espiral 
    \item RAD (Rapid Application Development)
    \item Variantes de cascada:
    \begin{itemize}
        \item Entrega por etapas
        \item Diseño por planificación
        \item Cascada con subproyectos
        \item Entrega Evolutiva
        \item Cascada c/reducción de Riesgo
    \end{itemize}
    RMM :Relationship Management Methodology
    Métodos ÁGILES
    \begin{itemize}
        \item Scrum
        \item XP
        \item Cristal
        \item etc
    \end{itemize}
    \item Prototipo evolutivo
    \item RUP (Rational Unified Process)
    \item ENTRE OTRAS
\end{itemize}
\subsection{Técnicas}
\begin{itemize}
    \item Casos De Uso-UML, Dig Transición de estados –UML, Dig. De Actividad –
    \item UML, Dig. De colaboración -UML
    \item DFD
    \item MER
    \item Narrativa estructurada / simple
    \item Programación por pares
    \item Desarrollo basado en pruebas
    \item Tablas y Arboles de decisión
    \item BPMN
    \item Diagrama de procedimiento adm.
    \item Entrevistas
    \item Cuestionarios
    \item Técnicas de trabajo en grupos: Focus
    \item Group, Lluvia de ideas
    \item JAD (Joint Application Development)
    \item Observación en terreno
    \item Revisión de documentos
    \item Técnicas de consenso y decisiones:
    \item Delphi, mayoría, dictadura
    \item ENTRE OTRAS
\end{itemize}
\begin{tcolorbox}
    Las \textbf{técnicas} se pueden utilizar en varios \textbf{procesos}.
\end{tcolorbox}



\end{document}
