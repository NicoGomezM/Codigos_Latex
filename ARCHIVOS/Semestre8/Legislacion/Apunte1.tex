\documentclass{templateNote}
\usepackage{tcolorbox}
\usepackage{enumitem}
\usepackage{hyperref}
\usepackage{amsmath}
\usepackage{amssymb}
\usepackage{soul}
\usepackage{circuitikz}


\begin{document}
\imagenlogoU{img/logoNGMFormal_sinF.png}
\linklogoU{https://github.com/NicoGomezM} 
% \imagenlogoD{img/logo-ubb-txt-face.png} 
\titulo{Certamen 2}
\asignatura{Legislación}
\autor{
    \indent
    Nicolás {Gómez Morgado}
}


\portada
\margenes 
\tableofcontents
\newpage


% \section{Importante Certamen}
% \noindent\textbf{\textit{\href{https://www.bcn.cl/portal/}{Articulo 1º:}} La ley es una declaración de la voluntad soberana que, manifestada en la forma prescrita por la Constitución, manda, prohíbe o permite.}
% \begin{itemize}
%     \item \textbf{Norma:} Regla de conducta que se establece en la conciencia del individuo.
%     \item \textbf{Contrato:} Acuerdo de voluntades entre 2 o mas personas que crea o transfiere derechos y obligaciones.
% \end{itemize}


% \newpage
% \section{Ambientes}
% \begin{itemize}
%     \item \textbf{Ambiente natural}
%     \begin{itemize}
%         \item No cabe intervención del hombre, es decir, no se puede modificar.
%     \end{itemize}
    
%     \item \textbf{Ambiente social}
%     \begin{itemize}
%         \item Grupo de individuos que establecen vínculos y relaciones recíprocas.
%     \end{itemize}
% \end{itemize}

% \section{Orígenes de la sociedad}
% \noindent\textbf{$1^{era}$ Teoría: Institución natural}
% \noindent Necesita vivir en sociedad ya que necesita de otros para sobrevivir y satisfacer sus necesidades.

% \noindent\textbf{$2^{da}$ Teoría: Institución convencional}
% \noindent El hombre es un ser racional y libre, por lo que decide vivir en sociedad por voluntad propia instituida por un pacto social.


% \begin{tcolorbox}
%     El establecimiento de sociedades siempre va a conllevar a futuros conflictos, por lo que se necesita de un ente regulador que establezca normas y sanciones para mantener el orden.
% \end{tcolorbox}

% \section{Normas de conducta}
% \begin{itemize}
%     \item \textbf{Normas de trato social}
%     \begin{itemize}
%         \item Reglas de cortesía, respeto, decoro y educación.
%     \end{itemize}
    
%     \item \textbf{Normas morales}
%     \begin{itemize}
%         \item Reglas de conducta que se establecen en la conciencia del individuo.
%     \end{itemize}
    
%     \item \textbf{Normas religiosas}
    
%     \item \textbf{Normas jurídicas}
%     \begin{itemize}
%         \item \textbf{Características:}
%         \begin{itemize}
%             \item Exteriores: Se manifiestan en la conducta de las personas.
%             \item Coercibles: Se pueden hacer cumplir por la fuerza.
%             \item Heterónomas: Son impuestas por un ente superior.
%             \item Bilaterales: Establecen derechos y obligaciones.
%         \end{itemize}
%     \end{itemize}
    
%     \item \textbf{Reglas técnicas}
%     \begin{itemize}
%         \item \textcolor{red}{reglas técnicas $\neq$ normas de conducta}
%     \end{itemize}
% \end{itemize}

% \newpage
% \section{Derecho}

% Viene del latín \textbf{directum} (lo que esta conforme a la ley o norma). Según \textit{Francisco Carnelutti}, el derecho es el conjunto de normas y mandatos jurídicos que se constituyen para garantizar, en un entorno social, la paz.


% \subsection*{Clasificación del derecho}

% \begin{itemize}
%     \item Frente
%     \begin{itemize}
%         \item \textbf{Escrito/Positivo:} Casi completamente redactado en papel, es decir, se puede leer y estudiar.
%         \item \textbf{Constitudinario:} Se basan en la costumbre, es decir, no está escrito. Comúnmente se da en países de origen anglosajones.
%         \begin{center}
%             \textcolor{red}{\textbf{Derecho positivo $\neq$ Derecho constitudinario}}    
%         \end{center}
%     \end{itemize}
%     \item País o extranjero
%     \begin{itemize}
%         \item \textbf{Nacional}
%         \item \textbf{Internacional}
%     \end{itemize}
%     \item Ámbito de aplicación
%     \begin{itemize}
%         \item \textbf{Publico:} \hl{Estado}.
%         \begin{itemize}
%             \item Derecho constitucional
%             \item Derecho administrativo
%             \item Derecho penal
%             \item Derecho financiero
%             \item Derecho económico
%             \item Derecho internacional público
%             \item Derecho procesal
%         \end{itemize}
%         \item \textbf{Privado:} Regula las relaciones entre particulares siempre que el estado no actué como ente soberano.
%         \begin{itemize}
%             \item \textbf{Derecho civil:} Principios y preceptos sobre la personalidad, relaciones patrimoniales (compra/venta de productos) y familiares (divorcio, pensiones, sucesión por causa de muerte).
%             \item \textbf{Derecho comercial:} Rige los actores de comercio, los comerciantes y los actos de comercio.
%             \item \textbf{Derecho laboral/del trabajo:} Rige las relaciones laborales para equilibrar la relación entre empleador y empleado.
%         \end{itemize}
%     \end{itemize}
% \end{itemize}

% \begin{tcolorbox}[colback=red!5!white,colframe=red!75!black]
%     En chile la pena de muerte si existe, pero solo se ha de aplicar con quorum calificado, es decir, con 2/3 de los votos de los parlamentarios.
% \end{tcolorbox}

% \subsection*{Fuentes del derecho (origen/precedencia)}

% \begin{itemize}
%     \item \textbf{Materiales del derecho:} Conjunto de factores que influyen en la creación de las normas jurídicas, como la religión, la moral, la economía, la política, etc.
%     \item \textbf{Formales del derecho:} Formas que tiene el derecho para manifestarse.
%     \begin{itemize}
%         \item La ley
%         \item La costumbre (no constituye ley a no ser que esta misma lo constituya)
%         \item Jurisprudencia (Sentencias solo tienen causa en las cuales se pronuncian)
%         \item Doctrina (Opiniones, comentarios, y de actores relativos a derecho)
%         \item Actos de las personas jurídicas 
%         \item Tratados internaciones
%         \begin{itemize}
%             \item Bilaterales
%             \item Multilaterales
            
%             \vspace*{0.4cm}
%             \begin{minipage}{0.45\textwidth}
%                 \begin{itemize}
%                     \item Generales
%                     \item Restringidos
%                 \end{itemize}
%             \end{minipage}
%             \hfill
%             \begin{minipage}{0.45\textwidth}
%                 \begin{itemize}
%                     \item Abiertos 
%                     \item Cerrados
%                 \end{itemize}
%             \end{minipage}
%         \end{itemize}
%     \end{itemize}
% \end{itemize}

% \newpage
% \section{Estado}

\section{Acto Jurídico}
Las Fuentes del Derecho se dividen en Fuentes Materiales y Fuentes Formales. Los \textbf{actos jurídicos de los particulares} son una de las Fuentes Formales del Derecho. Se distingue entre \textbf{hechos y hechos jurídicos}: un hecho es cualquier suceso, ya sea natural o humano, y cuando no produce consecuencias jurídicas, se le llama hecho simple o material, sin relevancia para el Derecho. En cambio, un hecho que sí genera consecuencias jurídicas se llama hecho jurídico, el cual puede ser natural (por ejemplo, nacimiento o muerte) o humano.

Dentro de los hechos jurídicos del hombre, se hace una distinción según si se realizan con o sin intención. Si no hay intención de generar efectos jurídicos, hablamos de \textbf{delitos, cuasidelitos (ilícitos) y cuasicontratos (lícitos)}. Cuando el hecho se realiza con la intención de producir efectos jurídicos, se denomina acto jurídico.

El \textbf{acto jurídico} es una manifestación o declaración de voluntad con el fin de crear, modificar o extinguir derechos y obligaciones, produciendo los efectos deseados por las partes.
\\\\
\noindent Podemos señalar que son rasgos distintivos o características de un Acto Jurídico:
\begin{enumerate}[label=\alph*)]
    \item Es una declaración o un conjunto de declaraciones de voluntad;
    \item La voluntad de los declarantes persigue un fin práctico lícito;
    \item Este fin práctico se traduce en efectos jurídicos, que se atribuyen o reconocen por el ordenamiento jurídico a la voluntad de los declarantes;
    \item Con el acto jurídico, los sujetos regulan sus propios intereses;
    \item Esta regulación es preceptiva, o sea, origina preceptos, impone normas de autonomía privada; y
    \item Las declaraciones que envuelven los actos jurídicos son vinculantes, comprometen, auto-obligan a los que las emiten.
\end{enumerate}

\subsection{Clasificación de los actos jurídicos}
Para referirnos a la clasificación de los Actos Jurídicos, previamente debemos indicar que existe una clasificación dada por el Código Civil, que, aun cuando las normas hablen o se refieran a los “contratos”, son enteramente aplicables a los actos jurídicos en general; y una clasificación más extensa, elaborada por la doctrina.

\subsubsection{Clasificación legal}

\begin{enumerate}
    \item \textbf{Actos Jurídicos Unilaterales y Bilaterales (art. 1439 CC):}
    \begin{itemize}
        \item Unilaterales: Requieren la voluntad de una sola parte (ejemplo: testamento).
        \item Bilaterales: Necesitan el acuerdo de dos o más partes (ejemplo: compraventa).
    \end{itemize}
    \item \textbf{Actos Jurídicos Gratuitos y Onerosos (art. 1440 CC):}
    \begin{itemize}
        \item Gratuitos: Benefician solo a una parte, causando un gravamen a la otra (ejemplo: donación).
        \item Onerosos: Benefician a ambas partes, con un intercambio mutuo (ejemplo: arrendamiento).
    \end{itemize}
    \item \textbf{Actos Jurídicos Conmutativos y Aleatorios (art. 1441 CC):}
    \begin{itemize}
        \item Conmutativos: Ambas partes se obligan a dar o hacer algo equivalente (ejemplo: compraventa de bienes).
        \item Aleatorios: El intercambio depende de una contingencia incierta (ejemplo: contrato de seguros).
    \end{itemize}
    \item \textbf{Actos Jurídicos Principales y Accesorios (art. 1442 CC):}
    \begin{itemize}
        \item Principales: Pueden subsistir por sí mismos (ejemplo: compraventa, arrendamiento).
        \item Accesorios: Aseguran el cumplimiento de una obligación principal (ejemplo: hipoteca, fianza).
        \item Dependientes: Dependen de otro acto para producir efectos, pero no garantizan el cumplimiento (ejemplo: capitulaciones matrimoniales).
    \end{itemize}
    \item \textbf{Actos Jurídicos Reales, Solemnes y Consensuales (art. 1443):}
    \begin{itemize}
        \item Reales: Se perfeccionan con la entrega de la cosa (ejemplo: comodato).
        \item Solemnes: Requieren formalidades legales para ser válidos (ejemplo: testamento).
        \item Consensuales: Se perfeccionan con el solo consentimiento de las partes (ejemplo: compraventa de bienes muebles).
    \end{itemize}
\end{enumerate}

\subsubsection{Clasificación doctrinaria}
\begin{enumerate}
    \item \textbf{Actos Jurídicos de Familia y Patrimoniales:}
    \begin{itemize}
        \item Familiares: Relacionados con la situación personal en la familia, como el matrimonio o el reconocimiento de un hijo.
        \item Patrimoniales: Afectan derechos patrimoniales o valores monetarios, como la mayoría de los contratos civiles.
    \end{itemize}
    \item \textbf{Actos Jurídicos Instantáneos, de Ejecución diferida, de Tracto sucesivo e Indefinidos:}
    \begin{itemize}
        \item Instantáneos: Producen efectos inmediatos, como la compraventa, donde el acuerdo y el pago se realizan de forma inmediata, aunque pueden subsistir obligaciones posteriores como la responsabilidad por defectos en el bien vendido.
        \item De Ejecución Diferida: Sus efectos se cumplen progresivamente según un plazo estipulado, como en un contrato de compraventa con pago a plazos o un contrato de construcción con entrega a futuro.
        \item De Tracto Sucesivo: Renovación continua de los efectos del contrato, como en un contrato de arrendamiento, que se renueva automáticamente por períodos determinados.
        \item De Duración Indefinida: No tienen un plazo de terminación específico y buscan una relación jurídica continua, como el contrato de trabajo indefinido o el contrato de sociedad.
    \end{itemize}
    \item \textbf{Actos Jurídicos Entre vivos y Por causa de muerte:}
    \begin{itemize}
        \item Entre Vivos: Se producen en vida, como un contrato de compraventa.
        \item Por Causa de Muerte: Solo producen efectos tras la muerte de una persona, como el testamento.
    \end{itemize}
    \item \textbf{Actos Jurídicos Puros y simples y Sujetos a modalidad:}
    \begin{itemize}
        \item Puros y Simples: Producen efectos inmediatos al ser celebrados, sin restricciones, generando derechos que pueden ejercerse de inmediato.
        \item Sujetos a Modalidad: Sus efectos dependen de condiciones restrictivas, que pueden ser:
        \begin{itemize}
            \item Condición: Hecho futuro e incierto que determina el inicio (condición suspensiva) o la extinción (condición resolutoria) de un derecho, como entregar un bien al cumplirse un requisito.
            \item Plazo: Hecho futuro y cierto que fija el momento para ejercer o extinguir un derecho, como pagos en cuotas. Puede ser expreso o tácito.
            \item Modo: Carga impuesta al beneficiario en actos gratuitos, como destinar un bien a un uso específico, por ejemplo, construir un centro de rehabilitación en un predio donado.
        \end{itemize}
    \end{itemize}
    \item \textbf{Actos Jurídicos Típicos o Nominados y Atípicos o Innominados:}
    \begin{itemize}
        \item Típicos o Nominados: Regidos por la ley, con estructura definida, como la compraventa.
        \item Atípicos o Innominados: No definidos por la ley, surgen de la autonomía de las partes, como el contrato de leasing mobiliario.
    \end{itemize}
\end{enumerate}

\subsection{Elementos del acto jurídico}

Según el artículo 1444 del Código Civil, los elementos de los actos jurídicos se clasifican en:

\begin{enumerate}
    \item \textbf{Esenciales}: Son indispensables para que el acto produzca efectos o no se transforme en otro contrato distinto.
    \item \textbf{De la Naturaleza}: No son esenciales, pero se presumen parte del contrato sin necesidad de una cláusula especial.
    \item \textbf{Accidentales}: No pertenecen al contrato ni esencial ni naturalmente, pero se incorporan mediante cláusulas específicas.
\end{enumerate}

\subsubsection{Elementos esenciales}
Los elementos esenciales de los actos jurídicos son aquellos sin los cuales no se generan efectos o el acto degenera en otro distinto. Se dividen en:

\subsubsection*{Elementos esenciales generales}
Aplican a todo acto jurídico e incluyen:
\begin{itemize}
    \item \textbf{Voluntad}: Es la intención de realizar un acto jurídico. En actos bilaterales se llama consentimiento. Debe ser seria y manifestarse exteriormente.
    \item \textbf{Objeto}: Es el fin perseguido por las partes al celebrar el acto.
    \item \textbf{Causa}: Es el motivo que induce a realizar el acto o contrato.
    \item \textbf{Solemnidades}: Requisitos formales exigidos por ley para la validez o existencia del acto.
    \item \textbf{Consentimiento sin vicios}: Debe estar exento de:
    \begin{itemize}
        \item Error: Falsa creencia sobre hechos o derecho.
        \item Fuerza: Coacción física o moral que anula la libertad de decisión.
        \item Dolo: Intención de causar daño.
        \item Lesión: Desproporción evidente en un contrato conmutativo.
    \end{itemize}
    \item \textbf{Capacidad}: Facultad de ejercer derechos y asumir obligaciones por sí mismo. La capacidad es la regla, y la incapacidad, la excepción (como en menores o dementes).
    \item \textbf{Objeto lícito}: Debe ser posible, comerciable y determinado, sin contradecir la ley o moral.
    \item \textbf{Causa lícita}: No debe ser ilegal, inmoral ni contraria al orden público.
\end{itemize}

\subsubsection*{Elementos esenciales específicos}
Son específicos de ciertos actos jurídicos y los distinguen según su naturaleza.
\begin{itemize}
    \item Por ejemplo, en un contrato de compraventa, los elementos esenciales son la cosa (objeto del contrato) y el precio (debe ser en dinero; si se paga mayoritariamente en bienes, se tratará de permuta).
\end{itemize}

\subsubsection{Elementos de la naturaleza}
Son aquellos que, no siendo esenciales en un acto jurídico, se entienden pertenecerle, sin necesidad de una cláusula especial. Están señalados en la ley. En otras palabras, si las partes desean excluir estos elementos, deben pactarlo en forma expresa.
Ejemplos de elementos de la naturaleza son:
\begin{enumerate}[label=\roman*.]
    \item Saneamiento de la evicción o de los vicios redhibitorios en la compraventa;
    \item Facultad de delegación en el mandato;
    \item En el mismo contrato de mandato, la remuneración u honorario a que tiene derecho el mandatario.
\end{enumerate}

\subsubsection{Elementos accidentales}
Son aquellas cosas que ni esencial ni naturalmente le pertenecen al acto jurídico, pero que pueden agregarse en virtud de una cláusula especial que así lo estipule.

\begin{itemize}
    \item Ejemplo: las modalidades, como el plazo, la condición o el modo. Así, en el retail, un cliente tiene la posibilidad de comprar un determinado bien en dinero efectivo, o mediante tarjeta de crédito en 6 cuotas mensuales, iguales y sucesivas (plazo).
\end{itemize}


\subsection{Efectos del acto jurídico}
Los \textbf{actos jurídicos} generan efectos como crear, modificar, transferir, transmitir o extinguir derechos y obligaciones, dependiendo de su tipo. En principio, estos efectos solo alcanzan a las partes involucradas, sin beneficiar ni perjudicar a terceros, dado el carácter relativo de los actos jurídicos. Sin embargo, existen situaciones donde puede ser necesario precisar cómo estos efectos impactan a terceros.

\subsubsection{Las partes}
Son quienes intervienen en la formación del acto jurídico, asumiendo los derechos y obligaciones que este genera, modifica, transfiere o extingue, ya sea en el ámbito patrimonial o familiar. El acto jurídico les aplica plenamente, conforme a principios como la "ley del contrato" (art. 1545 del Código Civil). Una parte puede estar compuesta por una o más personas con un interés común (art. 1438 del Código Civil). 
Se llama \textbf{autor} al que manifiesta su voluntad en un acto unilateral, y \textbf{parte} a quienes participan en un acto bilateral.

\subsubsection{La representación}
Es una modalidad en la que una persona, facultada por la ley o por voluntad del representado, celebra un acto jurídico en nombre y por cuenta de este último, generando efectos directos en su patrimonio. Puede ser \textbf{legal} (establecida por la ley) o \textbf{convencional} (derivada de acuerdos entre personas). Es esencial para que incapaces absolutos o relativos puedan actuar jurídicamente y para que personas plenamente capaces operen en varios lugares simultáneamente. Está regulada en los arts. 43 y 1448 del Código Civil.

\subsubsection{Los terceros}
Son quienes no participaron ni fueron representados en la formación del acto jurídico. Se dividen en:

\begin{itemize}
    \item \textbf{Terceros Absolutos}: Personas ajenas al acto, sin relación jurídica con las partes. El acto no les afecta, conforme al principio del art. 1545 del Código Civil.
    \item \textbf{Terceros Relativos o Interesados}: Aunque no participaron en el acto, tienen o tendrán relaciones jurídicas con las partes por voluntad propia o disposición legal, como los acreedores del deudor (arts. 2465 y 2469 del Código Civil).
\end{itemize}

\newpage
\section{Bienes y cosas}

La \textbf{cosa} es todo lo que ocupa un lugar y podemos percibir con nuestros sentidos, este concepto es aplicable a las cosas \textit{corporales}, que nuestros sentidos perciben.
De igualmente existen las cosas \textit{incorporales} que contrario a las anteriores no se pueden percibir con los sentidos. 
Discutiblemente nuestro código civil equipara o asimila a las cosas incorporales con los derechos. A raíz de esto se deja fuera de la tipología a las cosas incorporales que tampoco son derechos.
Estas cosas \underline{inmateriales} son aquellas que nuestros textos denominan producciones de talento o del ingenio. (art. 584 C.C.)

Los \textbf{bienes} son las cosas que le prestan \textit{utilidad} al hombre y son susceptibles de \textit{apropiación}. Se entiende por utilidad la aptitud de las cosas para satisfacer las necesidades humanas.

Para efectos prácticos, hablaremos indistintamente de cosas y bienes.

\subsection{Clasificación de los las cosas o bienes}

Los bienes se clasifican atendiendo 3 \textbf{causas o criterios}:
\begin{itemize}
    \item \textbf{Calidades físicas y jurídicas}: Calidades materiales de las cosas y las calidades que el derecho asigna a ciertos bienes. Aquí se comprenden las clasificaciones:
    \begin{itemize}
        \item \textbf{Bienes corporales e incorporales}
        \item \textbf{Bienes muebles e inmuebles}
        \item \textbf{Bienes fungibles y no fungibles}: Los fungibles son aquellos que pueden ser reemplazados por otros de la misma especie, calidad y cantidad. Los no fungibles son aquellos que no pueden ser reemplazados por otros de la misma especie, calidad y cantidad.
        \item \textbf{Bienes consumibles e inconsumibles}: 
        \item \textbf{Bienes específicos y genéricos}:
        \item \textbf{Bienes presentes y futuros}: 
        \item \textbf{Bienes registrables y no registrables}: 
    \end{itemize}
    \item \textbf{Relación del bien con otros bienes}: Criterio de distinción entre cosas:
    \begin{itemize}
        \item \textbf{Bienes singulares y universales}: Los singulares son aquellos que están determinados en su individualidad. Los universales son aquellos que están determinados por su especie.
        \item \textbf{Bienes principales y accesorios}: Los principales son aquellos que existen por sí mismos. Los accesorios son aquellos que existen en función de los principales.
        \item \textbf{Bienes divisibles e indivisibles}: Los divisibles son aquellos que pueden ser divididos en partes sin perder su esencia. Los indivisibles son aquellos que no pueden ser divididos en partes sin perder su esencia.
    \end{itemize}
    \item \textbf{Relación del bien con las personas}: Criterio de distinción entre cosas:
    \begin{itemize}
        \item \textbf{Bienes comerciables e incomerciables}: Los comerciables son aquellos que pueden ser objeto de comercio. Los incomerciables son aquellos que no pueden ser objeto
        \item \textbf{Cosas apropiables e inapropiables}: Las apropiables son aquellas que pueden ser objeto de apropiación. Las inapropiables son aquellas que no pueden ser objeto de apropiación.
        \item \textbf{Cosas particulares y nacionales}: Las particulares son aquellas que pueden ser objeto de propiedad privada. Las nacionales son aquellas que pertenecen a la nación.
    \end{itemize}
\end{itemize}

\noindent A continuación se detallan mas a fondo algunos de las clasificaciones mas relevantes:

\subsubsection*{Bienes corporales e incorporales}
Las cosas corporales son las que tienen un ser físico, que pueden ser percibidas por los sentidos. Las incorporales son las que no tienen un ser físico, pero que pueden ser objeto de derechos.
\begin{itemize}
    \item \textbf{Bienes corporales}: Son las cosas que tienen un ser físico, que pueden ser percibidas por los sentidos.
    \item \textbf{Bienes incorporales}: Son las cosas que no tienen un ser físico, pero que pueden ser objeto de derechos.
    \begin{itemize}
        \item Derechos reales: Son aquellos que recaen sobre una cosa sin relación a determinada persona (dominio, herencia, usufructo, uso de habitación, servidumbre, prenda, hipoteca).
        \item Derechos personales o créditos: Son aquellos que recaen sobre ciertas personas, que por hecho suyo o de la disposición de la ley contraen obligaciones.
    \end{itemize}
\end{itemize}

\subsubsection*{Bienes muebles e inmuebles}
Los muebles son aquellos que pueden trasladarse de un lugar a otro sin alterar su forma o sustancia. Los inmuebles son los que no pueden trasladarse de un lugar a otro sin alterar su forma o sustancia.
A su vez los \textbf{bienes muebles} se clasifican en:
\begin{itemize}
    \item Bienes muebles por naturaleza: Pueden ser trasladados de un lugar a otro y son semovientes o inanimados.
    \begin{itemize}
        \item Semovientes: Se mueven de un lugar a otro por sí mismos (animales).
        \item Inanimados: No se mueven por sí mismos (muebles).
    \end{itemize} 
    \item Bienes muebles por anticipación: Son aquellos inmuebles por naturaleza, por adherencia o destinación, que se reputan muebles aun antes de ser separados del inmueble del que forman parte o al cual se encuentran adheridos permanentemente destinados para su uso, beneficio o cultivo (art. 571 C.C.) (Madera del bosque antes de ser cortada, fruta antes de ser cosechada, etc).
\end{itemize}
De igual manera los \textbf{bienes inmuebles} se clasifican en:
\begin{itemize}
    \item Bienes inmuebles por naturaleza: Son aquellos que se adhieren permanentemente a un inmueble por naturaleza (arboles) o a un inmueble por adherencia (frutos de los arboles).
    \item Bienes inmuebles por destinación: Son aquellos muebles que la ley reputa inmuebles por una ficción, como consecuencia de estar destinadas permanentemente al uso, cultivo o beneficio de un inmueble, sin embargo pueden ser separados sin detrimento de la cosa principal (art. 570 C.C.).
\end{itemize}

\textbf{Derechos muebles e inmuebles}: Al disponer de los derechos se reputan bienes muebles o inmuebles, según sea lo sea la cosa en que han de ejercerse, se refiere evidentemente a los derechos reales, ya que son estos derechos los que se ejercen ``en'' las cosas (art. 580 C.C.).


\subsubsection*{Bienes específicos y genéricos}

Cosa específica o especie o cuerpo cierto es aquella que está determinada dentro de su genero también determinado. Por ejemplo un caballo de raza pura llamado JOSE, un cuadro de un pintor famoso llamado ``Estrellas estrelladas'', un tesla model s patente xxxxxx nro motor xxxxxxxxxxx y nro chasis xxxxxxxxxxx, etc.
\\\\
Cosa genérica es aquella que está determinada dentro de su genero también determinado. Estas cosas admiten mayor o menor determinación, pero siempre llega un momento en que se traspasa la linea que las separa de las cosas específicas. Por ejemplo un caballo de raza pura, un cuadro de un pintor famoso, un auto tesla model s, etc.

\subsubsection*{Bienes principales y accesorios}
Cosas principales son aquellas que existen por sí mismas, sin necesidad de otra cosa que las sostenga o complemente. 
Cosas accesorias son aquellas que existen en función de las principales, que las complementan o sostienen sin las cuales no pueden subsistir o no tienen razón de ser. Por ejemplo la caja de cambio de un auto, los adornos de una prenda, la vaina de una espada, etc.

No solo las cosas corporales pueden ser principales o accesorias, las cosas incorporales (derechos) también pueden serlo. Por ejemplo el derecho de prenda o hipoteca es un derecho accesorio, ya que son accesorios del crédito que garantizan.

\subsubsection*{Bienes divisibles e indivisibles}
Desde el punto de vista jurídico existen dos conceptos de divisibilidad: uno material y otro intelectual.
Son:
\begin{itemize}
    \item \textbf{Materialmente divisibles}: Aquellas cosas que sin destrucción pueden ser divididas en partes homogéneas entre si, no sufriendo menoscabo considerable el valor del conjunto de las partes. Por ejemplo una barra de metal, alimentos, agua, etc.
    \item \textbf{Materialmente indivisibles}: Aquellas cosas que no pueden ser divididas ya que se destruye el estado natural de la cosa. Por ejemplo los animales, las obras de arte, etc.
    \item \textbf{Intelectualmente divisibles}: Aquellas cosas que pueden ser divididas en partes ideales o imaginarias (cuotas), aunque no se pueda dividir materialmente. De este modo se puede decir que todos los bienes corporales e incorporales son divisibles intelectualmente.
\end{itemize}

\subsubsection*{Bienes presentes y futuros}
Los \textbf{bienes presentes} son aquellos que existen en el momento de constituirse la relación jurídica que los considera. Por ejemplo, un artefacto electrónico de tienda de retail.
Los \textbf{bienes futuros} son aquellos que no tienen existencia real al momento de constituirse la relación jurídica que los considera, pero se espera racionalmente que la tengan con mas o menos probabilidad en el futuro. Por ejemplo, la casa o terreno comprado en verde, etc. 

\subsubsection*{Bienes singulares y universales}
Los \textbf{bienes singulares} son aquellos que constituyen una unidad natural o artificial, simple o compleja pero con existencia real en la naturaleza. Por ejemplo, un auto, una casa, un cuadro, etc.
Los \textbf{bienes universales} son agrupaciones de bienes singulares sin conjunción o conexión física entre si, que por tener o considerarse que tienen un lazo vinculatorio forman un todo y reciben una denominación común, forman un todo funcional y están relacionados por un vinculo determinado. Pueden ser universales de hecho o de derecho. 

\begin{itemize}
    \item \textbf{Universales de hecho}: Un conjunto de bienes muebles, ya sean de naturaleza idéntica o diferente, que, aunque permanecen separados y conservan su individualidad, se consideran una unidad debido a su destino común, generalmente económico. Ejemplos de este tipo de universalidades son un rebaño de ovejas (bienes de naturaleza idéntica), una biblioteca (también de bienes idénticos), o un establecimiento de comercio (que incluye tanto bienes corporales como incorporales, como el derecho de llaves).
    \item \textbf{Universales de derecho}: es un conjunto de bienes y relaciones jurídicas activas y pasivas consideradas jurídicamente como formando un todo indivisible. Ejemplo clásico de este tipo de universalidades es el patrimonio de una persona difunta, llamado herencia, o el patrimonio de la sociedad conyugal habida por el hecho del matrimonio entre dos personas. 
\end{itemize}

\subsubsection*{Bienes comerciables e incomerciables}

Los \textbf{bienes comerciables} son los  que pueden ser objeto de relaciones jurídicas privadas, de manera que sobre ellos puede recaer un derecho real o puede constituirse a su respecto un derecho personal (arts. 1461 y 2498). Pueden incorporarse, por ende, al patrimonio de una persona. 
Los \textbf{bienes incomerciables} o no comerciables son las que no pueden ser objeto de relaciones jurídicas por los particulares. No puede existir a su respecto un derecho real ni personal. Por ende, no pueden incorporarse a patrimonio alguno.
Dentro de los bienes incomerciables existen: 
\begin{itemize}
    \item \textbf{Cosas comunes a todos los hombres}: son las únicas que no pueden ser objeto de relaciones jurídicas en general y por ende están fuera del comercio humano en términos absolutos y definitivos. Por ello, desde un punto de vista jurídico, no podemos considerar a estas cosas como “bienes”. Aquí encontramos el aire, la alta mar, etc.
    \item \textbf{Bienes nacionales de uso público}:  Son aquellos cuyo dominio pertenece a la nación toda y su uso a todos los habitantes de la misma. En este caso se trata de bienes que, siendo comerciables por naturaleza, han sido sustraídos del comercio jurídico para dedicarlas a un fin público. Con todo, los bienes nacionales de uso público pueden ser objeto de relaciones jurídicas de carácter público, como en el caso de las concesiones que otorga la autoridad. Por lo tanto, sólo desde el punto de vista del Derecho Privado, pueden considerarse como cosas incomerciables. Además, tampoco lo son en términos absolutos, porque cabe la posibilidad que sean desafectados, y se conviertan en bienes comerciables. Ejemplos de este tipo de bienes son las calles, los caminos, las plazas, los puentes, los ríos, las playas, etc. \\\\No deben confundirse con los \textit{bienes fiscales} que son aquellos pertenecientes al patrimonio del Estado, en cuanto este actúa como sujeto de relaciones privadas, caso en que se le denomina Fisco. De acuerdo al art. 589, los bienes fiscales son los bienes nacionales cuyo uso no pertenece a la nación toda, y por ende están dentro del comercio humano. Lo son, por ejemplo, los edificios afectos a los servicios públicos, los impuestos y contribuciones, entre otros.
\end{itemize}

\subsection{La Posesión}
\noindent Regulada en el C.C. entre los artículos 700 y 731. Son elementos constitutivos de la posesión el \textit{corpus} y el \textit{animus}.

\textbf{Corpus} corresponde al elemento material de la posesión, es la tenencia de una cosa determinada, precisa, lo cual da la posibilidad física de disponer materialmente de ella.

\textbf{Animus} corresponde al elemento psicológico, y quiere decir que el poseedor tiene una cosa con el ánimo de señor o dueño.

De igual manera el poseedor puede tener la cosa en su poder o bajo su dependencia inmediata, u otra persona puede tenerla, pero en nombre del poseedor; este se denomina “\textbf{mero tenedor}”
\\
La posesión habilita a llegar a adquirir el dominio mediante la \textbf{prescripción} después de un tiempo determinado.

\subsubsection{Clases}
Se distinguen las clases de posesión según si se habilita o no a adquirir el dominio por prescripción:
\begin{itemize}
    \item \textbf{Posesión útil:} Puede ser regular o irregular:
    \begin{itemize}
        \item \textbf{Regular:} A sido adquirida por buena fe, procede de justo titulo y habilita a adquirir el dominio por prescripción ordinaria.
        \item \textbf{Irregular:} Carece de uno o mas de los requisitos de la posesión regular, habilita a adquirir el dominio por prescripción extraordinaria.
    \end{itemize}
    \item \textbf{Posesión inútil:} No habilita a adquirir el dominio por prescripción, por ejemplo, la posesión de un ladrón. Puede ser:
    \begin{itemize}
        \item \textbf{Viciosa:} Se adquiere por medio de un vicio en el consentimiento.
        \item \textbf{Violenta:} Se adquiere por medio de la fuerza actual o inminente.
        \item \textbf{Clandestina:} Se adquiere por medio de la ocultación a quien tiene derecho a la cosa.
    \end{itemize}
\end{itemize}

\subsubsection{Mera tenencia}
Se llama mera tenencia la que se ejerce sobre una cosa, no como dueño, sino en lugar o a nombre del dueño. El mero tenedor detenta la cosa pero no tiene el \textit{corpus} ni el \textit{animus}.
Posesión y mera tenencia son conceptos excluyentes, no pueden coexistir en la misma persona. 
La mera tenencia nunca conduce a la prescripción, porque para prescribir el dominio es necesario poseer la cosa y la mera tenencia no posee.

\subsection{El dominio o la propiedad}

La propiedad se define en el C.C. art. 582, esta solo se refiere solo a los bienes corporales, pero también se aplica a los bienes incorporales; asi se dispone en el art. 583 y en la Constitución Política de la República en el art. 19 Nº 24 al garantizar la protección del derecho de propiedad en sus diversas especies sobre toda clase de bienes corporales o incorporales.

\subsubsection{Características}
\begin{itemize}
    \item \textbf{Es un derecho real:} Derecho real por excelencia, ya que recae directamente sobre la cosa ademas de estar amparado por una acción real (reivindicatoria).
    \item \textbf{Es un derecho absoluto:} Presenta 2 alcances:
    \begin{itemize}
        \item El dueño puede ejercer sobre la cosa todas sus facultades, sin limitación alguna.
        \item El dueño tiene un poder soberano sobre al cosa para usar, gozar y disponer de ella a su arbitrio, sin mas limitaciones que las impuestas por la ley.
    \end{itemize}
    \item \textbf{Es un derecho Exclusivo:} El derecho de dominio es exclusivo, ya que implica un único titular con la facultad de usar, gozar y disponer de la cosa, excluyendo la intromisión de otros. No pueden existir dos propietarios con poderes absolutos sobre el mismo bien, aunque sí puede haber copropiedad, donde cada dueño tiene una parte, limitando los derechos de los demás.
    \item \textbf{Es un derecho perpetuo:} La propiedad es perpetua, no tiene término, no se extingue por el no uso, ni por el no ejercicio de las facultades que otorga. El propietario solo podría perder el dominio si se deja de poseer la cosa por el tiempo determinado que un tercero necesita para adquirir la propiedad por prescripción.
    \item \textbf{Es un derecho Abstracto o flexible y elástico:} Es \textbf{abstracto} porque permanece en el titular, independientemente de las facultades cedidas, como el uso o el goce, que se transfieren temporalmente a otros. Es \textbf{flexible} porque puede contraerse al coexistir con otros derechos, como el usufructo, y expandirse al extinguirse dichos derechos, recuperando el titular pleno ejercicio de sus facultades. 
\end{itemize}

\subsubsection{Facultades del dominio}
\noindent Existen 2 clases:
\begin{itemize}
    \item Facultades Materiales
    \begin{itemize}
        \item Facultad de uso (Ius utendi): Capacidad que tiene el propietario de servirse o utilizar la cosa. Aplicar la cosa misma a todos los servicios que es capaz de proporcionar, sin tocar sus productos ni realizar una utilización que implique su destrucción inmediata. 
        \item Facultad de goce (Ius fruendi): Facultad que proporciona la capacidad de adueñarse de los frutos y los productos de la cosa. 
        \item Facultad de disposición (Ius abutendi): Capacidad para destruir, transformar o degradar la cosa. La facultad de disposición material representa la facultad característica del dominio. Los demás derechos reales, si bien autorizan a sus titulares para usar y gozar de una cosa ajena de una manera más o menos completa, jamás dan poder para destruirla o transformarla; siempre implican la obligación de conservar su forma y sustancia.
    \end{itemize}
    \item Facultades Jurídicas
    \begin{itemize}
        \item Facultad de disposición jurídica: La facultad de disposición jurídica permite al titular desprenderse de su derecho sobre una cosa, ya sea mediante renuncia, abandono o enajenación, por actos entre vivos o por causa de muerte.
        \begin{itemize}
            \item Renuncia: Puede ser un acto de disposición cuando el titular decide desprenderse de un derecho ya en su patrimonio, como condonar una deuda. Sin embargo, no es disposición si el derecho nunca se adquirió, como rechazar una herencia.
            \item Abandono: El dueño se despoja de una cosa, como un bien mueble, que otro puede adquirir mediante ocupación.
            \item Enajenación: Puede ser en sentido amplio, transfiriendo el derecho o gravándolo con otro derecho real, o en sentido estricto, transfiriendo el derecho de un patrimonio a otro.
        \end{itemize}
    \end{itemize}
\end{itemize}

\subsubsection{Plena o nuda propiedad}
La propiedad puede ser:
\begin{itemize}
    \item Plena: Permite al propietario ejercer todas las facultades (uso, goce y disposición).
    \item Nuda: Restringe las facultades de uso y goce debido a un derecho real de usufructo, quedando solo la facultad de disposición, jurídica y material.
\end{itemize}

\subsection{Modos de adquirir el dominio}

Existen dos teorías sobre cómo se adquiere la propiedad:  

\begin{enumerate}
    \item Teoría del título o modo: Propone que basta con un título (Francia) o un modo (Alemania) para adquirir el dominio.  
    \item Teoría del título y modo (adoptada por el Derecho Romano y nuestra legislación): Requiere tanto un título (acto que antecede la adquisición) como un modo (acto que la efectúa).  
\end{enumerate}

En bienes muebles, la distinción entre título y modo puede ser poco clara, como en la compraventa consensual. En inmuebles, se distingue claramente: el contrato (título) consta en escritura pública, y la inscripción en el Conservador de Bienes Raíces (modo) concreta la adquisición.  

Nuestro Código exige un título como antecedente de la Tradición, pero respecto a otros modos, su necesidad es debatida.

\begin{itemize}
    \item Enumeración de los modos de adquirir el dominio:
    \begin{itemize}
        \item \textbf{La Ocupación} (art. 606)
        \item \textbf{La Accesión} (art. 643)
        \item \textbf{La Tradición} (art. 670)
        \item \textbf{La Sucesión por causa de muerte} (art. 951)
        \item \textbf{La Prescripción adquisitiva} (art. 2492)
        \item A los anteriores, cabe añadir \textbf{la Ley} que, si bien no está mencionada por el art. 588, se agrega entre los modos de adquirir, pues en ciertos casos opera como tal: por ejemplo, el usufructo legal del padre o madre sobre los bienes del hijo no emancipado y el del marido sobre los bienes de la mujer (art. 810). 
    \end{itemize}
\end{itemize}

\subsubsection{La ocupación}
La ocupación es un modo de adquirir el dominio de cosas corporales muebles que no tienen dueño, mediante su aprehensión material acompañada de la intención de adquirirlas, siempre que no esté prohibido por leyes nacionales o internacionales.

\noindent\textbf{Requisitos:}
\begin{itemize}
    \item Cosas sin dueño: Puede tratarse de bienes que nunca tuvieron propietario (res nullius, como animales salvajes) o bienes abandonados voluntariamente (res derelictae, como tesoros o monedas arrojadas).
    \item Cosas corporales muebles: En Chile, solo estas pueden adquirirse por ocupación, ya que los inmuebles sin dueño pertenecen al Estado y las cosas incorporales no son susceptibles de aprehensión material.
    \item Legalidad: La adquisición no debe contravenir leyes chilenas o internacionales.
    \item Aprehensión material con intención: Requiere un acto físico de toma de posesión y la intención de adquirir el dominio.
\end{itemize}

\subsubsection{La accesión}
La accesión es un modo de adquirir el dominio por el cual el dueño de una cosa se convierte en propietario de lo que esta produce o de lo que se une a ella, aplicando únicamente a cosas corporales.  

\noindent\textbf{Tipos de accesión:}
\begin{enumerate}
    \item Accesión discreta (por producción): Se refiere a la generación de frutos o productos derivados de la cosa madre.  
    \item Accesión continua (por unión): Ocurre cuando dos cosas diferentes se unen formando un todo indivisible, y puede clasificarse en:  
    \begin{itemize}
        \item De inmueble a inmueble (natural): Ejemplo: aluvión, avulsión, cambio de cauce de un río, formación de islas.  
        \item De mueble a inmueble (industrial): Producto de la acción humana, no de la naturaleza.  
        \item De mueble a mueble: Cuando dos bienes muebles se unen, y la cosa accesoria pertenece al propietario de la principal (ejemplos: adjunción, especificación, mezcla).  
    \end{itemize}
\end{enumerate}

\subsubsection{La sucesión por causa de muerte}
La sucesión por causa de muerte es un modo de adquirir bienes cuando una persona fallece.

\noindent\textbf{Tipos de sucesión:}
\begin{enumerate}
    \item Por título universal: Se heredan todos los bienes, derechos y obligaciones transmisibles del difunto o una cuota de ellos (ejemplo: mitad, tercio).
    \item Por título singular: Se heredan bienes específicos (ejemplo: una casa, un caballo, o una cantidad de dinero).
\end{enumerate}

\noindent\textbf{Formas de sucesión:}
\begin{itemize}
    \item Testamentaria: Basada en un testamento.
    \item Intestada o abintestato: Determinada por la ley, cuando no hay testamento.
    \item \underline{Ambas 2 pueden coexistir en una misma sucesión.}
\end{itemize}

\subsubsection{La prescripción adquisitiva}
La prescripción adquisitiva es un modo de adquirir el dominio de cosas ajenas al poseerlas durante un período de tiempo establecido por la ley, siempre que se cumplan los requisitos legales.

\noindent\textbf{Tipos de prescripción según el C.C.:}
\begin{itemize}
    \item Adquisitiva: Permite adquirir dominio.
    \item Extintiva: Extingue acciones y derechos ajenos (no es parte del estudio en este contexto).
\end{itemize}
\noindent\textbf{Características:}
\begin{itemize}
    \item Debe alegarse: Puede ser utilizada como acción (al demandar) o como excepción (al contestar una demanda).
\end{itemize}

\subsection{\hl{La tradición} (modo de adquirir el dominio importante)}
La tradición es un modo de adquirir el dominio de las cosas, regulado en los artículos 670 a 699 del Código Civil. Consiste en la entrega de una cosa por parte del dueño (tradente) a otro (adquirente), con la intención de transferir y adquirir el dominio.

\noindent\textbf{Elementos clave:}
\begin{itemize}
    \item Sujetos: Tradente (quien transfiere) y adquirente (quien recibe).
    \item Entrega: No es suficiente por sí sola; se requiere la intención de transferir el dominio del tradente y la intención de adquirirlo del adquirente.
    \item Facultad y capacidad: El tradente debe tener la facultad de transferir el dominio, mientras que el adquirente solo necesita la capacidad para celebrar válidamente la transacción.
    \item Alcance: La tradición no solo aplica para transferir el dominio, sino también otros derechos reales y personales.
\end{itemize}

\subsubsection{Entrega y tradición}
La entrega es el traspaso material de una cosa de una persona a otra, pero puede ser entrega propiamente tal o tradición, con algunas diferencias clave:

\begin{enumerate}
    \item \textbf{Intención}
    \begin{itemize}
        \item En la tradición, tanto el tradente como el adquirente tienen la intención de transferir y adquirir el dominio.
        \item En la entrega propiamente tal, no hay esta intención, solo se transfiere la tenencia.
    \end{itemize}
    \item \textbf{Titulo}
    \begin{itemize}
        \item En la tradición, existe un título traslaticio de dominio (como en una compraventa).
        \item En la entrega, el título es de mera tenencia (como en un contrato de arrendamiento).
    \end{itemize}
    \item \textbf{Efecto}
    \begin{itemize}
        \item Con la tradición, se adquiere el dominio o posesión.
        \item En la entrega, solo se obtiene la mera tenencia, que generalmente no permite adquirir por prescripción.
    \end{itemize}
\end{enumerate}

\subsubsection{Requisitos de la tradición}
\begin{enumerate}[label=\alph*)]
    \item \textbf{Presencia de dos partes}: Deben intervenir el tradente y el adquirente, ya que la tradición es un acto jurídico bilateral (art. 671).
    \item \textbf{Consentimiento de ambas partes}: Es necesario que el tradente tenga la intención de transferir y el adquirente de adquirir el dominio. Este consentimiento debe recaer sobre:
    \begin{itemize}
        \item La cosa: Debe ser entregada en su totalidad, salvo que se haya acordado lo contrario.
        \item El título: Debe ser válido y suficiente para transferir el dominio.
        \item La persona a quien se transfiere (art. 672 y 673).
    \end{itemize}
    \item \textbf{Existencia de un título traslaticio de dominio}: El título traslaticio de dominio es el documento que, por su naturaleza, sirve para transferir el dominio, aunque por sí mismo no lo transfiere (art. 675 y 703), porque ese rol lo cumplen los modos de adquirir, pero este titulo sirve de antecedente para la adquisición del dominio. El título debe ser válido y no nulo.
    \item \textbf{Entrega de la cosa}: La entrega de la cosa, acompañada de la intención de transferir el dominio, es el elemento material esencial de la tradición.
\end{enumerate}

\subsubsection{Efectos}
Los efectos de la tradición dependen de si el tradente es o no el dueño de la cosa:

\begin{enumerate}
    \item El tradente es dueño de la cosa:
    \begin{itemize}
        \item Se transfiere el dominio del tradente al adquirente (arts. 670, 671 y 1575).
        \item El dominio transferido incluye todos los derechos y cargas que pesaban sobre la cosa, como hipotecas u otras obligaciones. Ejemplo: Si el departamento vendido tiene una hipoteca, el adquirente también la asume.
    \end{itemize}
    \item El tradente no es dueño de la cosa:
    \begin{itemize}
        \item La tradición es válida según el art. 1815, que permite la venta de cosa ajena.
        \item Tres situaciones pueden ocurrir:
        \begin{itemize}
            \item Poseedor regular: El adquirente de buena fe puede adquirir la posesión regular y, eventualmente, el dominio por prescripción.
            \item Poseedor irregular: Si el adquirente está de buena fe y tiene justo título, mejora su posición y puede adquirir el dominio.
            \item Tenedor de la cosa: Si el tradente es solo tenedor, el adquirente puede adquirir el dominio por prescripción, dependiendo de su buena fe. La prescripción será ordinaria si está de buena fe, y extraordinaria si no lo está.
        \end{itemize}
    \end{itemize}
    \item El tradente adquiere el dominio después de la tradición:
    \begin{itemize}
        \item La transferencia del dominio se considera efectiva desde el momento de la tradición, aunque el tradente adquiera el dominio posteriormente (arts. 682 inc. 2° y 1819).
    \end{itemize}
\end{enumerate}

\subsubsection{Especies de tradición o formas de efectuarla}
\begin{enumerate}
    \item \textbf{Tradición de derechos reales sobre cosas muebles}:
    \begin{itemize}
        \item Real: Implica una entrega física, con la intención de transferir y adquirir el derecho.
        \item Ficta: Se hace mediante un símbolo o señal que representa la cosa, poniéndola bajo el control del adquirente.
    \end{itemize}
    \item \textbf{Tradición de derechos reales sobre cosas inmuebles}:
    \begin{itemize}
        \item Se efectúa mediante la inscripción del título en el Registro del Conservador de Bienes Raíces (art. 686 CC).
        \item En el caso de la tradición del derecho de servidumbre, se realiza mediante escritura pública (art. 820 y 698 CC).
    \end{itemize}
    \item Tradición del derecho de herencia: No se examina en este contexto.
    \item Tradición de derechos personales: Regulada por los artículos 1901 y siguientes del Código Civil.
\end{enumerate} 

\newpage
\section{Contratos}
\subsection{Etimología y concepto}

El término ``contrato'' proviene del latín \textit{contractus}, que en Roma se refería al vínculo entre dos personas, ya sea por un hecho voluntario o no. El concepto moderno de contrato, sin embargo, se origina del ``nudo pacto'' (conventio), reconocido por el derecho pretorio, que destacaba el elemento voluntario.
\\\\
Dos factores clave contribuyeron a la evolución del contrato:

\begin{enumerate}
    \item La figura del contrato innominado, que implicaba dos prestaciones no determinadas pero susceptibles de resolverse.
    \item El ``nudo pacto'', que inicialmente no generaba obligaciones ni acción, pero que, durante la Edad Media, fue influenciado por los canonistas, el derecho natural y los comercialistas, evolucionando hacia una concepción de la voluntad de las partes como elemento esencial para crear un contrato.
\end{enumerate}

Así, el contrato moderno se define como un acuerdo de voluntades que, independientemente de su contenido, genera derechos y obligaciones. Este concepto está basado en la doctrina de la autonomía de la voluntad, un principio fundamental en el Código Civil chileno, al igual que en el francés.

\subsubsection{Concepto legal}
El Código Civil chileno considera a los contratos como una de las fuentes de las obligaciones, definiéndolos en el artículo 1438 como un acto en el que una parte se obliga con otra a dar, hacer o no hacer algo. Aunque en la ley se utilizan los términos ``contrato'' y ``convención'' de manera intercambiable, la doctrina critica esta sinonimia, ya que existe una relación de género a especie. 

Las convenciones son acuerdos de voluntades que pueden crear, modificar, transferir o extinguir derechos y obligaciones, pero los contratos son una subcategoría de ellas, específicamente aquellos que crean derechos y obligaciones. 

Ejemplos como la resciliación, el pago y la tradición son actos bilaterales que extinguen o transfieren derechos, pero no los crean, mientras que la novación es tanto una convención como un contrato, ya que crea y extingue obligaciones.

\subsection{Autonomía de la voluntad}
En derecho privado, las personas tienen libertad para actuar, salvo lo expresamente prohibido por la ley. El principio de autonomía de la voluntad establece que las obligaciones y derechos de un contrato dependen de la voluntad de las partes. Los principios fundamentales de la contratación son:

\begin{enumerate}
    \item Consensualismo: El contrato se forma con el simple acuerdo de las partes, salvo excepciones como contratos solemnes y reales.
    \item Libertad contractual: Las partes son libres para decidir si contratar, con quién y cómo configurar el contrato.
    \item Fuerza obligatoria: Los contratos deben cumplirse según lo pactado.
    \item Efecto relativo: Los contratos solo generan derechos y obligaciones para quienes los celebran, no afectando a terceros.
    \item Buena fe: Obliga a las partes a actuar con lealtad y corrección desde las negociaciones hasta después de la finalización del contrato. Incluye:
    \begin{itemize}
        \item Buena fe subjetiva: Convicción personal de actuar correctamente.
        \item Buena fe objetiva: Comportamiento acorde a estándares de equidad y razonabilidad.
    \end{itemize}
\end{enumerate}

El \textbf{consensualismo} y la \textbf{libertad contractual} se relacionan con la formación del contrato, mientras que la \textbf{fuerza obligatoria} y el \textbf{efecto relativo} conciernen a sus efectos. La \textbf{buena fe}, de gran relevancia, abarca todas las etapas contractuales y es un principio independiente.

\subsection{Principio de la fuerza obligatoria de los contratos (Ley del contrato)}
El principio de fuerza obligatoria de los contratos se basa en el aforismo ``pacta sunt servanda'': los acuerdos deben cumplirse estrictamente. Según el art. 1545, un contrato legalmente celebrado tiene fuerza de ley para las partes y solo puede ser modificado por mutuo acuerdo o causas legales.

\noindent Aunque se compara con la ley, existen diferencias clave:

\begin{enumerate}
    \item \textbf{Ámbito}: Los contratos afectan solo a las partes, mientras que las leyes son generales y abstractas.
    \item \textbf{Formación}: Las leyes involucran poderes públicos; los contratos, acuerdos privados.
    \item \textbf{Duración}: Los contratos suelen ser temporales; las leyes son permanentes.
    \item \textbf{Terminación}: Algunos contratos no admiten resciliación o pueden finalizar por actos unilaterales; las leyes pueden derogarse.
    \item \textbf{Interpretación}: Leyes y contratos tienen normas interpretativas distintas.
\end{enumerate}

La obligatoriedad del contrato también implica que, en general, ni el legislador ni el juez pueden modificarlo, debiendo respetar lo pactado por las partes.

\subsection{Elementos del contrato}
Los elementos de los contratos, según el artículo 1444 del Código Civil, se dividen en:

\begin{enumerate}
    \item \textbf{Elementos esenciales}: Fundamentales para la existencia del acto jurídico.
    \begin{itemize}
        \item \textbf{Generales}: Requisitos comunes a todo acto jurídico.
        \item \textbf{Particulares}: Propios de un tipo específico de contrato. Ejemplos:
        \begin{itemize}
            \item Compraventa: cosa y precio en dinero (si el precio incluye especies y su valor supera al dinero, es permuta).
            \item Comodato: gratuidad (si hay precio, es arrendamiento).
            \item Sociedad: ánimo societario, aportes, utilidades y pérdidas.
            \item Usufructo: plazo.
            \item Transacción: derecho dudoso y concesiones recíprocas.
        \end{itemize}
    \end{itemize}
    \item \textbf{Elementos de la naturaleza}: Propios del tipo de contrato, pero no esenciales. Ejemplos:
    \begin{itemize}
        \item Saneamiento de evicción y vicios redhibitorios en compraventa.
        \item Delegación y remuneración en mandato.
        \item Condición resolutoria tácita en contratos bilaterales.
    \end{itemize}
    \item \textbf{Elementos accidentales}: No son necesarios, pero pueden añadirse para condicionar los efectos del contrato. Ejemplo: modalidades como plazo, condición o modo.
\end{enumerate}

Solo los \textbf{elementos esenciales} constituyen la base del acto jurídico, mientras que los \textbf{naturales} afectan sus efectos y los \textbf{accidentales} condicionan su eficacia.

\subsection{Clasificación de los contratos}
\subsubsection{Clasificacion del Código Civil}
\begin{enumerate}
    \item \textbf{Unilaterales y bilaterales}:
    \begin{itemize}
        \item Unilaterales: Solo una parte tiene obligaciones.
        \item Bilaterales (sinalagmáticos): Ambas partes asumen obligaciones.
    \end{itemize}
    \item \textbf{Plurilaterales o asociativos}: Participan dos o más partes con un objetivo común, como en los contratos de sociedad.
    \item \textbf{Gratuitos y onerosos (art. 1440)}:
    \begin{itemize}
        \item Gratuitos: Una parte obtiene beneficio sin contraprestación. Ejemplo: donación, comodato.
        \item Onerosos: Ambas partes tienen beneficios y cargas.
    \end{itemize}
    \item \textbf{Conmutativos y aleatorios (art. 1441)}:
    \begin{itemize}
        \item Conmutativos: Los resultados económicos son predecibles (arrendamiento, compraventa).
        \item Aleatorios: Los resultados dependen del azar (renta vitalicia, apuestas, seguros).
    \end{itemize}
    \item \textbf{Principales y accesorios (art. 1442)}:
    \begin{itemize}
        \item Principales: Existen por sí mismos (compraventa, arrendamiento).
        \item Accesorios: Garantizan el cumplimiento de otro contrato (hipoteca, prenda).
        \item Dependientes: Requieren otro contrato, pero no lo garantizan (capitulaciones matrimoniales, subcontratos).
    \end{itemize}
    \item \textbf{Reales, solemnes y consensuales (art. 1443)}:
    \begin{itemize}
        \item Consensuales: Se perfeccionan con el acuerdo de voluntades (compraventa de bienes muebles).
        \item Solemnes: Exigen formalidades legales (compraventa de inmuebles).
        \item Reales: Requieren la entrega de la cosa para perfeccionarse (comodato, mutuo).
    \end{itemize}
\end{enumerate}

Cada tipo de contrato se define por sus características y los requisitos necesarios para su perfección y cumplimiento.

\subsubsection{Clasificación doctrinaria}
\begin{enumerate}
    \item \textbf{Por su regulación:}
    \begin{itemize}
        \item \textbf{Nominados o típicos}: Regulados por la ley (ej. compraventa, arrendamiento).
        \item \textbf{Innomidados o atípicos}: No están reglamentados, surgen de la autonomía de las partes (ej. leasing, cuota litis, franquicia).
    \end{itemize}
    \item \textbf{Por su ejecución}:
    \begin{itemize}
        \item \textbf{Instantánea}: Obligaciones cumplidas al momento de celebrarse (ej. compraventa de bienes muebles).
        \item \textbf{Diferida}: Cumplimiento posterior en un plazo establecido (ej. compraventa a plazos).
        \item \textbf{Tracto sucesivo}: Obligaciones renovadas periódicamente (ej. arrendamiento, contrato de trabajo).
        \item \textbf{Duración indefinida}: Sin plazo de vigencia fijado (ej. contrato de sociedad, arrendamiento indefinido).
    \end{itemize}
    \item \textbf{Por el número de personas afectadas}:
    \begin{itemize}
        \item \textbf{Individuales}: Vinculan solo a las partes que los celebran (ej. compraventa).
        \item \textbf{Colectivos}: Obligaciones para terceros, incluso sin su consentimiento (ej. contrato colectivo de trabajo).
    \end{itemize}
    \item \textbf{Por su negociación}:
    \begin{itemize}
        \item \textbf{Libre discusión}: Negociados en igualdad de condiciones.
        \item \textbf{Adhesión}: Redactados por una parte, la otra se limita a aceptar (ej. contratos de seguros).
    \end{itemize}
    \item \textbf{Por su etapa}:
    \begin{itemize}
        \item \textbf{Preparatorios}: Promesa de celebrar un contrato futuro (ej. promesa de compraventa).
        \item \textbf{Definitivos}: Cumplen la obligación generada por el contrato preparatorio.
    \end{itemize}
    \item \textbf{Por la consideración de las partes}:
    \begin{itemize}
        \item \textbf{Intuitu personae}: Relevante la identidad de los contratantes (ej. contrato de trabajo, matrimonio).
        \item \textbf{Impersonales}: La identidad de las partes no es esencial.
    \end{itemize}
    \item \textbf{Por su naturaleza}:
    \begin{itemize}
        \item \textbf{De familia}: Relacionados con el ámbito familiar (ej. matrimonio).
        \item \textbf{Puramente patrimoniales}: Relacionados con derechos evaluables en dinero (ej. compraventa).
    \end{itemize}
    \item \textbf{Por su formación}:
    \begin{itemize}
        \item \textbf{Instantánea}: Nacen de un solo acto (ej. compraventa simple).
        \item \textbf{Progresiva}: Requieren negociación previa (ej. fusiones empresariales).
    \end{itemize}
    \item \textbf{Por sus efectos}:
    \begin{itemize}
        \item \textbf{Puros y simples}: Derechos inmediatos al celebrarse (ej. compraventa común).
        \item \textbf{Sometidos a modalidad}: Sujetos a condiciones, plazos o modos (ej. compraventa con pago diferido).
    \end{itemize}
\end{enumerate}

\subsubsection{Interpretación de los contratos}
Interpretar un contrato implica determinar el sentido y alcance de sus cláusulas. Esto ocurre cuando:

\begin{enumerate}[label=\alph*)]
    \item Los términos son ambiguos.
    \item Los términos claros no concuerdan con la naturaleza del contrato o la intención de las partes.
    \item Surgen dudas al relacionar las cláusulas del contrato.
\end{enumerate}

\noindent La normativa sobre interpretación está en los artículos 1560 a 1566 del Código Civil.

\subsubsection{Métodos de interpretación}
Existen dos métodos principales para interpretar contratos:

\begin{itemize}
    \item \textbf{Método objetivo}: Se enfoca en el sentido literal de las cláusulas según la costumbre, usos y prácticas comerciales, sin indagar la intención real de las partes. La declaración tiene autonomía respecto a la intención.
    \item \textbf{Método subjetivo}: Busca la voluntad real de los contratantes, considerando que esta puede no coincidir con lo expresado. Es el enfoque del Código Civil.
\end{itemize}

En resumen:

\begin{itemize}
    \item \textbf{Al contratar}, prima la intención de las partes (método subjetivo y buena fe subjetiva).
    \item \textbf{Al ejecutar}, si la intención no es clara, se aplica la buena fe objetiva, ajustándose a la ley.    
\end{itemize}

\subsubsection{Objetivo fundamental: intención de los contratantes}

El objetivo principal en la interpretación de un contrato es respetar la intención de las partes, incluso sobre el texto literal, como establece el artículo 1560. Si las cláusulas son ambiguas y generan discrepancias, el juez interpretará según las reglas de los artículos 1561 y siguientes del Código Civil.

\subsubsection{Reglas de interpretación de los contratos}
La interpretación de los contratos busca priorizar la intención de las partes sobre el texto literal, según el artículo 1560. En caso de ambigüedad, el juez aplica reglas específicas:

\begin{itemize}
    \item \textbf{Aplicación restringida (art. 1561)}: Los términos generales del contrato se limitan al objeto contratado.
    \item \textbf{Extensión natural (art. 1565)}: Una obligación expresada no excluye casos similares no mencionados.
    \item \textbf{Utilidad de las cláusulas (art. 1562)}: Se prefiere el sentido que produzca efectos.
    \item \textbf{Sentido natural (art. 1563)}: Se interpreta acorde a la naturaleza del contrato.
    \item \textbf{Armonía de cláusulas (art. 1564)}: Se consideran las cláusulas como un todo coherente.
    \item \textbf{Interpretación entre contratos (art. 1564)}: Pueden usarse otros contratos entre las mismas partes para interpretar el actual.
    \item \textbf{Aplicación práctica (art. 1564)}: Se analiza cómo las partes han ejecutado el contrato.
    \item \textbf{Cláusulas usuales (art. 1563)}: Se presumen aunque no estén explícitas.
    \item \textbf{Cláusulas ambiguas (art. 1566)}: Se interpretan a favor del deudor, salvo que la ambigüedad sea causada por una de las partes, en cuyo caso se resuelve en su contra.
\end{itemize}

\subsection{Disolución de los contratos}
Los contratos pueden disolverse por mutuo acuerdo entre las partes o por causas legales (resolución, nulidad, muerte en contratos ``intuitu personae'' o plazo extintivo), según los artículos 1545 y 1567 del Código Civil. La disolución implica que el contrato no produjo sus efectos completos o parciales, retornando las partes al estado previo al acuerdo. No se considera disolución cuando las obligaciones se extinguen tras ejecutar el contrato, ya que este cumplió su propósito.

\subsubsection{Resciliación o Mutuo disenso}
Es la disolución de un contrato por el consentimiento de las partes que lo celebraron. Se trata de una convención en la que las partes acuerdan dejar sin efecto un contrato, extinguiendo las obligaciones vigentes. No es un contrato, sino una acción para extinguir derechos y obligaciones.

La resciliación supone que el contrato es válido, y solo el legislador o el juez pueden declarar su nulidad. Solo puede resciliarse si las obligaciones aún están vigentes, es decir, no se han cumplido en su totalidad.

\noindent \textbf{Efectos:}

\begin{itemize}
    \item Entre las partes: Tiene efecto retroactivo, restableciendo el estado previo al contrato. Ambas partes deben restituir lo recibido, como en una compraventa de un inmueble, donde se revierte la propiedad y el dinero recibido.
    \item Respecto a terceros: La resciliación no afecta los derechos ya adquiridos por terceros, los cuales permanecen intactos. Los efectos de la resciliación son para el futuro, no para el pasado.
\end{itemize}

\noindent\textbf{Actos no resciliables y actos que pueden extinguirse por la voluntad de una sola parte:}

La regla general del art. 1545 que consagra la resciliación, tiene excepciones, sin embargo, desde dos puntos de vista: 

\begin{enumerate}[label=\alph*)]
    \item Algunos contratos no pueden resciliarse, como el matrimonio o las capitulaciones matrimoniales.
    \item Otros contratos pueden extinguirse por la voluntad de una sola parte, como en el caso de sociedades, mandatos, arrendamientos, donaciones, y acuerdos de unión civil.
\end{enumerate}

\subsubsection{Causas legales de disolución de los contratos}
\begin{enumerate}
    \item \textbf{La resolución}: Ocurre cuando se obtiene una sentencia judicial que termina el contrato debido al incumplimiento, total o parcial, de una de las partes.
    \item \textbf{La nulidad}: Puede ser absoluta o relativa, dependiendo de si falta alguno de los requisitos legales para la validez del contrato, como la calidad o el estado de las partes involucradas.
    \item \textbf{La muerte de uno de los contratantes en contratos intuitu personae}: En contratos como el mandato, la sociedad, el comodato y el matrimonio, la muerte de una de las partes puede disolver el contrato, aunque no en todos los casos (por ejemplo, la muerte del comodante no extingue el comodato).
    \item \textbf{El plazo extintivo}: Un contrato puede disolverse cuando un hecho futuro y cierto, dependiente de un plazo, extingue un derecho. Esto es común en contratos como la sociedad, el arrendamiento y el comodato.
\end{enumerate}

\newpage
\section{Propiedad Intelectual e Industrial}
Los Derechos de Propiedad Intelectual otorgan a las personas derechos exclusivos sobre las creaciones de su mente, como obras artísticas o literarias, por un plazo determinado. Los Derechos de Propiedad Industrial se refieren a la explotación exclusiva de nombres comerciales, marcas y patentes, también por un tiempo limitado.

En la legislación chilena, ambos tipos de propiedad están protegidos constitucional y legalmente. El artículo 19, número 25, de la Constitución Política asegura a las personas la libertad de crear y difundir artes, y garantiza el derecho de autor sobre sus obras intelectuales y artísticas, así como la propiedad industrial sobre patentes, marcas y otras creaciones similares.

La Ley N° 17.336 regula la propiedad intelectual, mientras que el DFL N° 4 de 2022 establece la Ley N° 19.039 sobre propiedad industrial.

\subsection{Propiedad Intelectual}
Como se dijo anteriormente, en nuestro país la propiedad intelectual está regulada por la Ley n° 17.336, cuya última modificación corresponde al año 2017. Esta Ley protege y regula los derechos de autor, los sujetos y la duración de la protección. 

\subsubsection{Objeto de protección}
La ley protege los derechos de autor sobre obras literarias, artísticas y científicas, sin importar su forma de expresión. Estos derechos se dividen en dos categorías:

\begin{enumerate}[label=\alph*)]
    \item \textbf{Derechos Patrimoniales}: Otorgan al titular el control sobre el uso y explotación de la obra, permitiéndole publicarla, reproducirla, adaptarla, ejecutarla públicamente y distribuirla. Además, la primera venta de una obra agota el derecho de distribución.
    \item \textbf{Derechos Morales}: Relacionados con la protección del vínculo personal del autor con su obra, incluyen la reivindicación de la paternidad, la oposición a modificaciones no autorizadas, la conservación de la obra inédita, y el derecho a autorizar la terminación de una obra inconclusa. Los derechos morales son inalienables y transmisibles solo por causa de muerte.
\end{enumerate}

El artículo 3 establece una lista de creaciones protegidas, como libros, obras dramáticas, composiciones musicales, fotografías, esculturas y programas informáticos.

\subsubsection{Sujetos del derecho}
Los sujetos del derecho de autor se dividen en dos tipos:
\begin{enumerate}
    \item Titular Original: El autor de la obra, quien tiene la facultad de decidir sobre la divulgación de la misma.
    \item Titular Secundario: Quien adquiere los derechos del autor, ya sea por cesión o transferencia (art. 6 y 7).
\end{enumerate}
En cuanto a los programas computacionales, el titular del derecho será la persona natural o jurídica cuyos empleados los hayan creado, salvo estipulación en contrario (art. 8). Si un programa es creado por encargo, los derechos del autor se ceden al encargado, a menos que se acuerde lo contrario.

\textbf{Derechos adicionales sobre programas computacionales:}
\begin{itemize}
    \item Los autores tienen el derecho exclusivo de autorizar o prohibir el arrendamiento comercial de sus programas (art. 37 bis).
    \item Se permiten ciertas actividades sin autorización del autor, como la adaptación o copia para uso personal o de respaldo, la ingeniería inversa para compatibilidad, y pruebas o correcciones para seguridad, siempre que no se comercialice la obra resultante ni se infrinja la ley.
\end{itemize}

\subsubsection{Duración de la protección}
\begin{itemize}
    \item Vida del autor y 70 años más: El derecho de autor se extiende durante toda la vida del autor y por 70 años adicionales tras su fallecimiento (art. 10).
    \item Programas computacionales: Si el empleador es una persona jurídica, la protección dura 70 años desde la primera publicación.
    \item Obras colaborativas: El plazo de 70 años se cuenta desde la muerte del último coautor (art. 12).
    \item Obras anónimas o seudónimas: Protección de 70 años desde la primera publicación, salvo que el autor se identifique, en cuyo caso se aplica el artículo 10.
\end{itemize}

\subsubsection*{Obras de dominio público}
Cuando finaliza la protección, las obras pasan al patrimonio cultural común, pudiendo ser utilizadas libremente respetando su paternidad e integridad. Esto incluye:
\begin{itemize}
    \item Obras con protección expirada.
    \item Obras de autor desconocido (como folklore).
    \item Obras cuyos titulares renunciaron a sus derechos.
    \item Obras de autores extranjeros no protegidas por la ley local.
    \item Obras expropiadas por el Estado, salvo que la ley disponga otro beneficiario (art. 11).
\end{itemize}

\subsubsection{Registro de propiedad intelectual}

El \textbf{Registro de Propiedad Intelectual}, regulado por los artículos 72 y siguientes de la Ley n.° 17.336, es donde deben inscribirse los derechos de autor y conexos. El titular puede usar el símbolo ``\copyright'' seguido del año de publicación y su nombre para identificar sus derechos patrimoniales, presumiéndose como titular quien figure de esta manera (art. 72 bis). Las transferencias de derechos deben inscribirse en el Registro en un plazo de 60 días desde el acto o contrato, que puede formalizarse mediante instrumento público o privado ante notario.


\subsection{Propiedad Industrial}
Como se señaló al principio de esta unidad, la Propiedad Industrial esta regulada en nuestro país por el DFL n° 4 del año 2022, el cual establece y fija el texto actual de la Ley n° 19.039 sobre la materia

\subsubsection{Objeto de protección}
La \textbf{propiedad industrial}, según el artículo 1°, comprende marcas, patentes de invención, modelos de utilidad, diseños y dibujos industriales, esquemas de trazado de circuitos integrados, indicaciones geográficas, denominaciones de origen y otros títulos que la ley establezca.

\begin{itemize}
    \item \textbf{Marcas}: Son signos que distinguen productos o servicios en el mercado. Pueden ser denominativas, figurativas, sonoras, tridimensionales, olfativas y otras marcas no tradicionales. Se inscriben en el Registro de Marcas del INAPI.

    \item \textbf{Patentes}: Otorgan derechos exclusivos para proteger invenciones, permitiendo al titular comercializarlas o impedir su uso sin permiso durante un tiempo limitado, a cambio de divulgar la invención.
    
    \item \textbf{Modelos de utilidad}: Protegen invenciones de menor complejidad técnica, como herramientas o dispositivos, que aporten una mejora técnica útil. También se conocen como ``pequeñas patente''.
    
    \item \textbf{Diseños y dibujos industriales}: Protegen formas tridimensionales o patrones bidimensionales con fines ornamentales, siempre que otorguen una apariencia novedosa.
    
    \item \textbf{Esquemas de trazado de circuitos integrados}: Son disposiciones tridimensionales de elementos destinados a realizar funciones electrónicas en un circuito integrado.
    
    \item \textbf{Indicaciones geográficas y denominaciones de origen}: Protegen productos cuya calidad o reputación se asocia a su origen geográfico. Las denominaciones de origen incluyen factores naturales y humanos.
\end{itemize}

\subsubsection{Sujeto de derecho}
El artículo 2 dispone que “Cualquier persona natural o jurídica, nacional o extranjera, podrá gozar de los derechos de la propiedad industrial que garantiza la Constitución Política, debiendo obtener previamente el título de protección correspondiente de acuerdo con las disposiciones de esta ley…”

\subsubsection{Duración de la protección}
La duración de la protección varía según el tipo de propiedad industrial:
\begin{itemize}
    \item \textbf{Marcas comerciales}: Registro válido por 10 años desde su inscripción, renovable por períodos iguales dentro de los 6 meses antes o después de su vencimiento. Si no se renueva, caduca automáticamente (art. 24).
    \item \textbf{Patentes de invención}: Vigencia de 20 años desde la solicitud, sin posibilidad de renovación (art. 39).
    \item \textbf{Modelos de utilidad}: Protegidos por 10 años desde la solicitud, sin renovación (art. 57).
    \item \textbf{Diseños y dibujos industriales}: Registro válido por 15 años desde la solicitud, no renovable (art. 65).
    \item \textbf{Esquemas de trazado de circuitos integrados}: Protección por 10 años desde la solicitud o primera explotación comercial, no renovable (art. 78).
    \item \textbf{Indicaciones geográficas y denominaciones de origen}: Vigencia indefinida mientras persistan las condiciones que justificaron su reconocimiento (art. 100).
\end{itemize}

\subsection{Instituto Nacional de Propiedad Industrial (INAPI)}

El Instituto Nacional de Propiedad Industrial (INAPI) gestiona la tramitación de solicitudes, otorgamiento de títulos y servicios relacionados con la propiedad industrial (art. 3). Toda solicitud admitida debe publicarse en extracto en el Diario Oficial (art. 4).
\\\\
\noindent\textbf{Oposición}: Cualquier interesado puede oponerse a las solicitudes dentro de:
\begin{itemize}
    \item 30 días desde la publicación de marcas.
    \item 45 días para patentes, modelos de utilidad, diseños industriales, esquemas de trazado, indicaciones geográficas y denominaciones de origen (art. 5).
\end{itemize}

\newpage
\section{Contratos Tecnológicos}
\subsection{Principios imperantes en la legislación chilena}

Los contratos tecnológicos no pueden regirse completamente por los principios del Código Civil. Se deben aplicar normas constitucionales y civiles compatibles con la realidad tecnológica.
\begin{itemize}
    \item Fundamento constitucional: El artículo 19 n.º 21 de la Constitución asegura la libertad económica, permitiendo actividades respetuosas de la moral, el orden público y la seguridad nacional.
    \item Fundamento legal: Según el artículo 1437 del Código Civil, las obligaciones en estos contratos nacen del acuerdo entre partes.
    \item Estos contratos, ajustados a la Constitución y al Código Civil, tienen fuerza de ley para las partes (art. 1545 del Código Civil), manifestando el Principio de Autonomía de la Voluntad, aunque con ciertas limitaciones respecto a contratos civiles tradicionales.
\end{itemize}

\subsection{Definición}
Los contratos tecnológicos son acuerdos voluntarios que implican la transferencia, desarrollo o licencia de tecnología, conocimientos técnicos o derechos de propiedad intelectual. Aunque se clasifican como contratos innominados, existen múltiples definiciones y denominaciones, como:
\begin{itemize}
    \item Contratos de Tecnología
    \item Contratos de Transferencia de Tecnología
    \item Contratos de Traspaso Tecnológico
    \item Acuerdos de Licencia “Licensing Agreement”
    \item Contratos de Licencia
    \item Licencia de Tecnología
\end{itemize}

\noindent\textbf{Definiciones destacadas:}
\begin{enumerate}
    \item \textbf{Leticia Mota Antúnez}: Contrato donde un proveedor realiza investigación, transmite conocimientos especializados o concede derechos de explotación (como patentes o derechos de autor) a un adquirente, generalmente a cambio de remuneración.
    \item \textbf{Félix Moreno}: Acuerdo en el que un cedente comparte tecnología o licencia conocimientos técnicos con un concesionario.
    \item \textbf{Jaime Álvarez Soberanis}: Contrato donde un proveedor transfiere a un receptor conocimientos organizados para la producción industrial.
\end{enumerate}

\subsection{Elementos constitutivos de los contratos tecnológicos}
\subsubsection{Las partes}

En los contratos tecnológicos, las partes involucradas no se denominan vendedor y comprador, sino:

\begin{enumerate}
    \item \textit{Proveedor, Concedente, Licenciante Comunicante o Transmisor}: Transfiere la tecnología o conocimientos a cambio de un precio.
    \item \textit{Adquirente, Usuario, Licenciatario, Beneficiario o Receptor}: Paga por recibir la tecnología.
\end{enumerate}

Una característica clave es la verticalidad, donde una parte (el proveedor) tiene mayor conocimiento y está en una posición superior de negociación, debido a factores como monopolios o asimetrías de información.

\subsection*{Obligación entre las partes}
\subsubsection*{Obligaciones del proveedor de tecnología}
\begin{itemize}
    \item \textbf{De Dar}: Transferencia de propiedad o posesión de bienes, ya sean corporales (maquinaria, herramientas) o incorporales (derechos sobre patentes, marcas y licencias).
    \item \textbf{De Hacer}: Prestación de servicios técnicos, administrativos o de operación en áreas como producción, finanzas o ventas, dependiendo del ámbito del contrato y la industria involucrada.
\end{itemize}

\subsubsection*{Obligaciones del adquirente de tecnología}
\begin{itemize}
    \item \textbf{De Dar}: Pago de una remuneración en dinero o especie, comúnmente mediante regalías o un porcentaje basado en los resultados obtenidos.
    \item \textbf{De Hacer}: Realización de actividades específicas, como aplicar un sistema de publicidad o métodos de venta determinados.
    \item \textbf{De no Hacer}: Respetar el secreto comercial, evitando divulgar conocimientos o sistemas adquiridos a través del contrato, especialmente si no están patentados.
\end{itemize}

\subsubsection*{El precio}
El \textbf{precio} en los contratos tecnológicos, conocido como \underline{\textbf{royalty}} o \underline{\textbf{regalía}}, es la suma de dinero acordada que el adquirente debe pagar por la transferencia de la tecnología, y generalmente es fijada de común acuerdo entre las partes.


\subsection{Características de los contratos tecnológicos}
Dentro de las características de los contratos tecnológicos, destacan:
\begin{enumerate}
    \item \textbf{Contratos Innomidados}: No están contemplados en el Código Civil.
    \item \textbf{Contratos Consensuales}: Se perfeccionan con el solo consentimiento de las partes, a menos que lo disponga la ley. El perfeccionamiento con el solo consentimiento es debido a que son contratos innominados no regulados por la ley.
    \item \textbf{Contratos Principales}: Generalmente regulan la transferencia de tecnología, aunque pueden ser dependientes de otros en casos específicos (el caso de aquellos contratos en que la transferencia de tecnología que se está regulando sean parte de las actividades de una empresa transnacional extranjera que opera desde hace tiempo en el país \{art. 1442\})
    \item \textbf{Contratos Bilaterales}: Implican obligaciones recíprocas entre las partes.
    \item \textbf{Contratos Onerosos}: Benefician a ambos contratantes, con reciprocidad de gravámenes.
    \item \textbf{Contratos Conmutativos}: Las prestaciones se consideran equivalentes, aunque la ``verticalidad'' puede generar desigualdad debido a la posición dominante de una de las partes.
\end{enumerate}

\subsection{Clasificación de los contratos tecnológicos}
Dentro de los contratos tecnológicos, se pueden distinguir varios tipos según su objeto y finalidad:
\begin{enumerate}
    \item Contratos de Licencia y Títulos de propiedad industrial (patentes)
    \item Contrato de Know-how o Licencia o provisión de conocimientos técnicos no patentados
    \item Contrato de Franchising o Acuerdo de Franquicia
    \item Contrato de Asistencia técnica
    \item Contrato de Prestación de servicios de ingeniería tanto básica como de detalle
    \item Contratos Informáticos
\end{enumerate}

\subsubsection{Contratos de Licencia y Títulos de propiedad industrial (patentes)}
Un contrato de licencia es un acuerdo en el que el \textbf{licenciante}, propietario de un conocimiento técnico (patentado o no), autoriza al \textbf{licenciatario} a usar y explotar comercialmente dicha tecnología, generalmente a cambio de una retribución económica.

\subsubsection*{Elementos generales}
\begin{itemize}
    \item Precio: El pago realizado por el licenciatario al licenciante, usualmente denominado regalía.
    \item Derecho de uso y goce: Autorización temporal para usar la tecnología, dado que las patentes tienen una duración limitada.
\end{itemize}

\subsubsection*{Clasificación}
\begin{itemize}
    \item \textbf{Gratuitos u onerosos}: Las gratuitas no implican pago, mientras que las onerosas requieren una regalía.
    \item \textbf{Exclusiva o no exclusiva}: En las exclusivas, el licenciante no puede otorgar licencias a otros; en las no exclusivas sí puede hacerlo.
    \item \textbf{Según la voluntad que los genera; Contractuales, Obligatorias, de Pleno derecho y de Oficio}: Las contractuales son las más comunes y surgen por la voluntad de las partes. Las obligatorias pueden imponerse por la ley, por ejemplo, si el titular no explota su patente. Las de pleno derecho permiten al titular ofrecer la licencia a quien lo solicite, y las de oficio son otorgadas por la autoridad. \hl{Detalle:}
    \begin{itemize}
        \item Contractuales: Son las más comunes y se celebran libremente entre las partes, basadas en su voluntad.
        \item Obligatorias: Impuestas por la ley, obligan al titular de una patente a ponerla en explotación localmente. Si no lo hace, la autoridad puede concederla a terceros, especialmente en casos de prácticas contrarias a la competencia, razones de salud pública, seguridad nacional o emergencias, o cuando se requiere para explotar una patente posterior.
        \item De Pleno derecho: Se generan por la voluntad del licenciante, quien decide permitir que su invención sea licenciada a quien lo solicite, dejando las condiciones a las partes.
        \item De Oficio: Cualquier persona puede solicitar autorización para explotar una invención en una actividad industrial previamente declarada por la autoridad, sin la intervención del titular de la patente.
    \end{itemize}
    \item \textbf{Expresa o Tácita}: La expresa se establece en un contrato, mientras que la tácita se asume por la existencia de otro contrato relacionado, como la venta de una máquina que implica la autorización para fabricar los productos patentados.
    \item \textbf{Duración menor o igual a la patente}: La licencia puede durar menos o igual que la patente, dependiendo de los términos acordados.
\end{itemize}

\subsubsection*{Terminación}
Las causales de terminación del contrato de licencia son:
\begin{itemize}
    \item \textbf{Plazo}: Es la más común. Consiste en el término del contrato por cumplirse el plazo estipulado de duración, si es que se le ha fijado un plazo; o bien, no se ha considerado una renovación tácita del contrato.
    \item \textbf{Rescisión (nulidad)}: Se declara judicialmente cuando el contrato tiene vicios de nulidad según el Código Civil, retrotrayendo a las partes al estado anterior a su celebración.
    \item \textbf{Resolución}: Se declara judicialmente si no se cumple lo pactado por alguna de las partes, o si ambas no cumplen; produce efectos solo para el futuro.
\end{itemize}

\subsubsection{Contrato de Know-how o Licencia o provisión de conocimientos técnicos no patentados}

El Know-how proviene de las palabras inglesas "to know" (saber) y "how" (cómo), y se refiere a la habilidad de llevar una idea a la práctica. Diversos autores lo han conceptualizado:

\begin{itemize}
    \item José Giral lo define como el conjunto de conocimientos técnicos organizados que resultan en una aplicación industrial o comercial.
    \item Denis Borges Barros lo describe como conocimientos técnicos originales, secretos o escasos, que otorgan una posición privilegiada en el mercado.
\end{itemize}

\textbf{Características del Know-how:}

\begin{itemize}
    \item Es un conocimiento técnico inmaterial, aunque puede materializarse en un documento.
    \item No es patentable o quienes lo poseen prefieren no patentarla.
    \item No necesariamente debe ser secreto, ya que puede existir un monopolio de hecho si se mantiene en secreto, lo que implica riesgo de copia.
    \item El contrato de Know-how es un acuerdo en el que una persona cede los derechos sobre ciertos conocimientos secretos a cambio de un precio durante un tiempo determinado.
\end{itemize}

\subsubsection*{Objeto}

El objeto del contrato de Know-how es proporcionar información técnica sobre un tema específico, la cual se mantiene en secreto o con divulgación restringida, permitiendo al adquirente realizar el procedimiento técnico de manera eficiente, ahorrando tiempo y dinero, a cambio de un precio.

\subsubsection*{Duración}
La duración del contrato debe ser acordada por las partes, ya sea para un acto único o continuo, y se recomienda que dependa del tiempo en que se mantenga el secreto.

\subsubsection*{Confidencialidad o cláusula de secreto}
La confidencialidad o cláusula de secreto establece que el beneficiario debe proteger los conocimientos transferidos para evitar que lleguen a la competencia, ya que el Know-how no tiene protección jurídica formal.


\subsubsection{Contrato de Franchising o Acuerdo de Franquicia}
La franquicia es el derecho otorgado por el \textbf{franquiciante} (proveedor) al \textbf{franquiciatario} (distribuidor) para hacer o usar algo, generalmente intangible.
\begin{itemize}
    \item Carlos María Correa define la franquicia como un sistema de distribución de productos o servicios basado en el prestigio y modalidades del titular de una marca. 
    \item Héctor Martínez la describe como un sistema de clonación en el que cada negocio franquiciado reproduce el modelo original de la empresa madre, con una relación de reciprocidad continua entre ambas partes.
    \item El Reglamento de la Ley de Transferencia de Tecnología mexicana define el acuerdo de franquicia como un contrato en el que el proveedor concede el uso de marcas o nombres comerciales y transmite conocimientos técnicos o asistencia para operar de manera uniforme y con los mismos métodos comerciales y operativos del franquiciante.
\end{itemize}

Este modelo es una técnica de marketing utilizada principalmente por empresas transnacionales para expandirse a nuevos mercados, manteniendo el mismo producto en nuevas plazas comerciales (ejemplo: McDonald's, Pizza Hut).
\\\\
\noindent\textbf{Existen tres tipos de franquicia:}
\begin{enumerate}
    \item \textbf{De Producto}: El franquiciatario distribuye productos del franquiciante, usando su nombre o marca.
    \item \textbf{De Procesamiento o Manufactura}: El franquiciante proporciona ingredientes esenciales y conocimientos técnicos a una firma para que fabrique el bien o servicio.
    \item \textbf{De Sistema}: El franquiciante transfiere un sistema único de hacer negocios, que el franquiciatario utiliza en la operación de su propio negocio. Este tipo ha crecido principalmente en el sector de servicios, como cafeterías o restaurantes.
\end{enumerate}

\subsubsection*{Las Partes}
\begin{itemize}
    \item \textit{Franquiciante, Franquiciador o Franchisor}: es el proveedor, aquella parte que proporciona los conocimientos técnicos, gerenciales, comerciales o de marketing.
    \item \textit{Franquiciatario, Franquiriente o Franchisee}: corresponde al receptor o adquirente de los conocimientos que se traspasan.
\end{itemize}

\subsubsection*{Tipos de franquicia}
\begin{enumerate}
    \item \textbf{De acuerdo al territorio}:
    \begin{itemize}
        \item Local: Operan individualmente en un área restringida.
        \item Maestra: Controlan un área geográfica mayor, gestionando varias franquicias locales, actuando como intermediarios entre la corporación madre y los franquiciatarios.
    \end{itemize}
    \item \textbf{De acuerdo a su origen}:
    \begin{itemize}
        \item Nacionales: Las partes están domiciliadas en el mismo país.
        \item Importadas: Surgen en un país y luego se expanden a otro. Las franquicias de alimentos (McDonald’s, KFC, etc.) son las más comunes.
    \end{itemize}
\end{enumerate}

\subsubsection*{Ventajas del acuerdo de franquicia}
\begin{itemize}
    \item \textbf{General:} Reduce costos de expansión, permite una redistribución en todos los niveles comerciales, facilita la inversión extranjera y combate el desempleo.
    \item \textbf{Para el franquiciante:}
    \begin{itemize}
        \item Fortalece la marca.
        \item Requiere baja inversión de capital.
        \item Expande rápidamente el mercado.
        \item Minimiza su riesgo (el riesgo recae en el franquiciatario).
        \item Genera ingresos y incentivos económicos.
    \end{itemize}
    \item \textbf{Para el franquiciatario:}
    \begin{itemize}
        \item Mantiene su iniciativa empresarial.
        \item Accede a una imagen sólida y soporte técnico.
        \item Menor riesgo de quiebra.
        \item Recibe entrenamiento y asesoría continua.
        \item Gana ingresos y beneficios económicos.
        \item Conserva independencia jurídica.
    \end{itemize}
    \item \textbf{Para el consumidor:}
    \begin{itemize}
        \item Productos y servicios más baratos.
        \item Acceso a productos de alta calidad en áreas no industrializadas.
        \item Mejor canalización de sus preferencias y gustos.
    \end{itemize}
\end{itemize}

\subsubsection*{Forma de pago de la franquicia}
El precio inicial consiste en un \textbf{valor fijo}, determinado por la empresa madre en función de diversos factores (financieros, comerciales, etc.).
Posteriormente, todo franquiciatario tiene que pagar al franquiciante una \textbf{cuota por derechos de uso, de marca, y de transferencias tecnológicas}. Es un valor establecido en el contrato y que generalmente se fija como un \textbf{porcentaje sobre las ventas netas}.


\subsubsection{Contrato de Asistencia técnica}
Es el apoyo especializado en una materia específica para mejorar actividades humanas, abarcando aspectos culturales, económicos y sociales.

\begin{itemize}
    \item Según Juan M. Fariña, la asistencia técnica se refiere a contratos en los cuales una empresa proveedora ofrece sus conocimientos, experiencias y personal especializado para colaborar en la producción.
\end{itemize}

\textbf{Diferencias con el Know-How}: Aunque ambas implican la transferencia de conocimientos, la principal diferencia es que la asistencia técnica incluye la participación activa del proveedor en el proceso de fabricación, mientras que en el Know-How solo se transfiere información sin intervención directa. Además, la asistencia técnica generalmente tiene una obligación de resultado, mientras que el Know-How no garantiza resultados.

\begin{table}[H]
    \centering
    \begin{minipage}[t]{0.45\textwidth}
        \textbf{Obligaciones del Proveedor:}
        \begin{itemize}
            \item Prestar servicios técnicos o dirigir actividades de producción, ingeniería, ventas o distribución.
            \item Asesorar en diversos aspectos industriales.
            \item Investigar en ramas científicas específicas.
            \item Supervisar la aplicación de técnicas y control de calidad.
            \item Informar sobre mejoras en productos y procesos.
        \end{itemize}
    \end{minipage}
    \hfill
    \begin{minipage}[t]{0.45\textwidth}
        \textbf{Obligaciones del Receptor:}
        \begin{itemize}
            \item Pagar por los servicios prestados.
            \item Asegurar la calidad y prestigio del producto.
            \item Vender el producto de manera específica.
            \item Seguir las instrucciones del proveedor.
        \end{itemize}
    \end{minipage}
\end{table}

\subsubsection{Contrato de Prestación de servicios de ingeniería tanto básica como de detalle}
\noindent Definiciones importantes para entender este tipo de contrato:
\begin{itemize}
    \item \textbf{Ingeniería}: Es la aplicación de las ciencias físico-matemáticas para inventar, mejorar y utilizar la técnica en la explotación de recursos naturales o para actividades industriales.
    \item \textbf{Servicios de ingeniería}: Son aquellos que utilizan conocimientos técnicos para incorporar mejoras al sistema productivo, ya sea creando capacidad instalada o modificando las condiciones de producción.
\end{itemize}

\noindent\textbf{Tipos de Ingeniería}:
\begin{itemize}
    \item \textbf{Ingeniería Básica}: Define las etapas de un proceso productivo basado en la investigación y el conocimiento empírico.
    \item \textbf{Ingeniería de Detalle}: Aplica los datos de la ingeniería básica para el diseño y ejecución de proyectos industriales, incluyendo construcción, montaje de maquinaria, instalaciones eléctricas, etc.
    \item \textbf{Contrato de Servicios de Ingeniería}: Es un acuerdo en el que una empresa proveedora ofrece los conocimientos necesarios para la fabricación y puesta en marcha de un equipo o planta, a cambio de una contraprestación económica.
\end{itemize}

\subsubsection*{Clases de Contratos de prestación de servicios de ingeniería}

\begin{itemize}
    \item \textbf{Consulting Engineering}: La empresa de ingeniería realiza estudios técnicos - económicos para proyectos industriales o para reorganización, modernización o expansión de una empresa.
    \item \textbf{Commercial Engineering}: La empresa no solo realiza estudios, sino que también ejecuta los proyectos, entregando un establecimiento industrial completo o un producto específico (por ejemplo, contratos ``llave en mano'' o ``producto en mano'').
\end{itemize}

\subsubsection{Contratos Informáticos}
Contratos informáticos son “aquellos acuerdos entre partes, que crean, conservan, modifican o extinguen derechos y obligaciones relacionados con los sistemas, subsistemas o elementos destinados al tratamiento sistematizado de la información”. 

\subsubsection*{Partes del contrato}

\begin{enumerate}
    \item \textit{Proveedores:} Son las personas o empresas dedicadas a la construcción, distribución y venta de equipos, o a la prestación de servicios informáticos. Sus principales obligaciones incluyen:
    \begin{itemize}
        \item Salvaguardar los intereses del cliente y ofrecer consejo e información.
        \item Cumplir con la entrega de bienes o servicios dentro de los plazos acordados.
        \item Ejecutar los servicios conforme a lo estipulado en el contrato.
        \item Garantizar la ausencia de vicios ocultos y asegurar la calidad de la prestación (obligación postcontractual de saneamiento).
    \end{itemize}
    \item \textit{Usuarios:} Son entidades públicas, privadas o individuos que necesitan satisfacer sus necesidades mediante bienes informáticos. Sus principales obligaciones son:
    \begin{itemize}
        \item Pagar el precio acordado según las modalidades del contrato.
        \item Informarse y documentarse sobre los equipos o servicios informáticos.
        \item Determinar y comunicar sus necesidades de automatización.
        \item Capacitar adecuadamente a su personal para el manejo de los servicios informáticos.
        \item Aceptar y adaptar el material o servicio a sus necesidades.
    \end{itemize}
\end{enumerate}

\subsubsection*{Tipos de contratos informáticos}
Los contratos informáticos se pueden clasificar según el \textbf{objeto} y la \textbf{modalidad}:

\begin{enumerate}
    \item \textbf{En cuanto al objeto del contrato}:
    \begin{itemize}
        \item \textbf{Desarrollo de Software}.
        \item \textbf{Bienes}: Compraventa de software (programas), hardware (equipos), suministro de refacciones, arrendamiento comercial.
        \item \textbf{Servicios}: Mantenimiento correctivo y preventivo, asistencia técnica.
    \end{itemize}
    \item \textbf{En cuanto a la modalidad del contrato}:
    \begin{itemize}
        \item \textbf{Llave en mano}: El proveedor se encarga de construir e instalar el sistema informático listo para funcionar (obligación de resultado).
        \item \textbf{Prestación informática}: Arrendamiento de programas o servicios informáticos para procesos productivos.
        \item \textbf{Contratos complejos}: Incluyen múltiples prestaciones informáticas, como la provisión de equipos, asistencia técnica y mantenimiento.
    \end{itemize}
\end{enumerate}

\noindent\textbf{Tipos de contratos específicos}:
\begin{itemize}
    \item Licencia de uso de programas.
    \item Desarrollo de programas.
    \item Compatibilización de equipos.
    \item Contrato programa-producto.
    \item Adquisición de programas.
    \item Contrato de material o sistema.
    \item Contrato de mantenimiento.
\end{itemize}

\subsubsection*{Clausulas de un contrato informático}
Las principales cláusulas comunes en los contratos informáticos son:
\begin{enumerate}
    \item \textbf{Definiciones}: Se deben definir términos técnicos utilizados en el contrato para evitar confusiones, incluyendo anexos con definiciones claras y completas.
    \item \textbf{Control, Supervisión y Acceso}: Es necesario ejercer control sobre los bienes y servicios adquiridos, con asesoramiento especializado y un buen mantenimiento para evitar problemas o gastos imprevistos.
    \item \textbf{Asistencia y Capacitación}: Se incluyen cláusulas para la capacitación del personal usuario debido a la complejidad de los servicios, detallando la duración, lugar y condiciones del servicio.
    \item \textbf{Secreto y Confidencialidad}: El proveedor debe mantener la confidencialidad de la información del usuario, con sanciones en caso de divulgación no autorizada.
    \item \textbf{Garantías y Responsabilidad}: El proveedor garantiza el buen funcionamiento de los bienes o servicios durante un tiempo determinado, después del cual se celebra un contrato de mantenimiento. Además, responde por la calidad y funcionamiento de los equipos o servicios proporcionados.
    \item \textbf{Cláusulas diversas}: Incluyen acuerdos adicionales como:
    \begin{itemize}
        \item \textbf{Cláusula de no solicitud de personal}: El usuario no puede contratar al personal del proveedor.
        \item \textbf{Cláusula de restricción de acceso al equipo}: El proveedor no es responsable por defectos si terceros no autorizados intervienen en los equipos o servicios.
    \end{itemize}
\end{enumerate}

\end{document}

