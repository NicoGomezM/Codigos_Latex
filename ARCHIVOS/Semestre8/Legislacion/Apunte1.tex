\documentclass{templateNote}
\usepackage{tcolorbox}
\usepackage{hyperref}
\usepackage{amsmath}
\usepackage{amssymb}
\usepackage{soul}
\usepackage{circuitikz}


\begin{document}
\imagenlogoU{img/logoNGMFormal_sinF.png}
\linklogoU{https://github.com/NicoGomezM} 
% \imagenlogoD{img/logo-ubb-txt-face.png} 
\titulo{Apunte 1}
\asignatura{Legislación}
\autor{
    \indent
    Nicolás {Gómez Morgado}
}


\portada
\margenes 
\tableofcontents
\newpage


\section{Importante Certamen}
\noindent\textbf{\textit{\href{https://www.bcn.cl/portal/}{Articulo 1º:}} La ley es una declaración de la voluntad soberana que, manifestada en la forma prescrita por la Constitución, manda, prohíbe o permite.}
\begin{itemize}
    \item \textbf{Norma:} Regla de conducta que se establece en la conciencia del individuo.
    \item \textbf{Contrato:} Acuerdo de voluntades entre 2 o mas personas que crea o transfiere derechos y obligaciones.
\end{itemize}


\newpage
\section{Ambientes}
\begin{itemize}
    \item \textbf{Ambiente natural}
    \begin{itemize}
        \item No cabe intervención del hombre, es decir, no se puede modificar.
    \end{itemize}
    
    \item \textbf{Ambiente social}
    \begin{itemize}
        \item Grupo de individuos que establecen vínculos y relaciones recíprocas.
    \end{itemize}
\end{itemize}

\section{Orígenes de la sociedad}
\noindent\textbf{$1^{era}$ Teoría: Institución natural}
\noindent Necesita vivir en sociedad ya que necesita de otros para sobrevivir y satisfacer sus necesidades.

\noindent\textbf{$2^{da}$ Teoría: Institución convencional}
\noindent El hombre es un ser racional y libre, por lo que decide vivir en sociedad por voluntad propia instituida por un pacto social.


\begin{tcolorbox}
    El establecimiento de sociedades siempre va a conllevar a futuros conflictos, por lo que se necesita de un ente regulador que establezca normas y sanciones para mantener el orden.
\end{tcolorbox}

\section{Normas de conducta}
\begin{itemize}
    \item \textbf{Normas de trato social}
    \begin{itemize}
        \item Reglas de cortesía, respeto, decoro y educación.
    \end{itemize}
    
    \item \textbf{Normas morales}
    \begin{itemize}
        \item Reglas de conducta que se establecen en la conciencia del individuo.
    \end{itemize}
    
    \item \textbf{Normas religiosas}
    
    \item \textbf{Normas jurídicas}
    \begin{itemize}
        \item \textbf{Características:}
        \begin{itemize}
            \item Exteriores: Se manifiestan en la conducta de las personas.
            \item Coercibles: Se pueden hacer cumplir por la fuerza.
            \item Heterónomas: Son impuestas por un ente superior.
            \item Bilaterales: Establecen derechos y obligaciones.
        \end{itemize}
    \end{itemize}
    
    \item \textbf{Reglas técnicas}
    \begin{itemize}
        \item \textcolor{red}{reglas técnicas $\neq$ normas de conducta}
    \end{itemize}
\end{itemize}

\newpage
\section{Derecho}

Viene del latín \textbf{directum} (lo que esta conforme a la ley o norma). Según \textit{Francisco Carnelutti}, el derecho es el conjunto de normas y mandatos jurídicos que se constituyen para garantizar, en un entorno social, la paz.


\subsection*{Clasificación del derecho}

\begin{itemize}
    \item Frente
    \begin{itemize}
        \item \textbf{Escrito/Positivo:} Casi completamente redactado en papel, es decir, se puede leer y estudiar.
        \item \textbf{Constitudinario:} Se basan en la costumbre, es decir, no está escrito. Comúnmente se da en países de origen anglosajones.
        \begin{center}
            \textcolor{red}{\textbf{Derecho positivo $\neq$ Derecho constitudinario}}    
        \end{center}
    \end{itemize}
    \item País o extranjero
    \begin{itemize}
        \item \textbf{Nacional}
        \item \textbf{Internacional}
    \end{itemize}
    \item Ámbito de aplicación
    \begin{itemize}
        \item \textbf{Publico:} \hl{Estado}.
        \begin{itemize}
            \item Derecho constitucional
            \item Derecho administrativo
            \item Derecho penal
            \item Derecho financiero
            \item Derecho económico
            \item Derecho internacional público
            \item Derecho procesal
        \end{itemize}
        \item \textbf{Privado:} Regula las relaciones entre particulares siempre que el estado no actué como ente soberano.
        \begin{itemize}
            \item \textbf{Derecho civil:} Principios y preceptos sobre la personalidad, relaciones patrimoniales (compra/venta de productos) y familiares (divorcio, pensiones, sucesión por causa de muerte).
            \item \textbf{Derecho comercial:} Rige los actores de comercio, los comerciantes y los actos de comercio.
            \item \textbf{Derecho laboral/del trabajo:} Rige las relaciones laborales para equilibrar la relación entre empleador y empleado.
        \end{itemize}
    \end{itemize}
\end{itemize}

\begin{tcolorbox}[colback=red!5!white,colframe=red!75!black]
    En chile la pena de muerte si existe, pero solo se ha de aplicar con quorum calificado, es decir, con 2/3 de los votos de los parlamentarios.
\end{tcolorbox}

\subsection*{Fuentes del derecho (origen/precedencia)}

\begin{itemize}
    \item \textbf{Materiales del derecho:} Conjunto de factores que influyen en la creación de las normas jurídicas, como la religión, la moral, la economía, la política, etc.
    \item \textbf{Formales del derecho:} Formas que tiene el derecho para manifestarse.
    \begin{itemize}
        \item La ley
        \item La costumbre (no constituye ley a no ser que esta misma lo constituya)
        \item Jurisprudencia (Sentencias solo tienen causa en las cuales se pronuncian)
        \item Doctrina (Opiniones, comentarios, y de actores relativos a derecho)
        \item Actos de las personas jurídicas 
        \item Tratados internaciones
        \begin{itemize}
            \item Bilaterales
            \item Multilaterales
            
            \vspace*{0.4cm}
            \begin{minipage}{0.45\textwidth}
                \begin{itemize}
                    \item Generales
                    \item Restringidos
                \end{itemize}
            \end{minipage}
            \hfill
            \begin{minipage}{0.45\textwidth}
                \begin{itemize}
                    \item Abiertos 
                    \item Cerrados
                \end{itemize}
            \end{minipage}
        \end{itemize}
    \end{itemize}
\end{itemize}

\newpage
\section{Estado}


\end{document}

