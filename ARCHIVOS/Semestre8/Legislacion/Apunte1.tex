\documentclass{templateNote}
\usepackage{tcolorbox}
\usepackage{hyperref}
\usepackage{amsmath}
\usepackage{amssymb}
\usepackage{soul}
\usepackage{circuitikz}


\begin{document}
\imagenlogoU{img/logoNGMFormal_sinF.png}
\linklogoU{https://github.com/NicoGomezM} 
% \imagenlogoD{img/logo-ubb-txt-face.png} 
\titulo{Apunte 1}
\asignatura{Legislación}
\autor{
    \indent
    Nicolás {Gómez Morgado}
}


\portada
\margenes 
\tableofcontents
\newpage


\section{Importante Certamen}
\noindent\textbf{\textit{\href{https://www.bcn.cl/portal/}{Articulo 1º:}} La ley es una declaración de la voluntad soberana que, manifestada en la forma prescrita por la Constitución, manda, prohíbe o permite.}
\begin{itemize}
    \item \textbf{Norma:} Regla de conducta que se establece en la conciencia del individuo.
    \item \textbf{Contrato:} Acuerdo de voluntades entre 2 o mas personas que crea o transfiere derechos y obligaciones.
\end{itemize}


\newpage
\section{Ambientes}
\begin{itemize}
    \item \textbf{Ambiente natural}
    \begin{itemize}
        \item No cabe intervención del hombre, es decir, no se puede modificar.
    \end{itemize}
    
    \item \textbf{Ambiente social}
    \begin{itemize}
        \item Grupo de individuos que establecen vínculos y relaciones recíprocas.
    \end{itemize}
\end{itemize}

\section{Orígenes de la sociedad}
\noindent\textbf{$1^{era}$ Teoría: Institución natural}
\noindent Necesita vivir en sociedad ya que necesita de otros para sobrevivir y satisfacer sus necesidades.

\noindent\textbf{$2^{da}$ Teoría: Institución convencional}
\noindent El hombre es un ser racional y libre, por lo que decide vivir en sociedad por voluntad propia instituida por un pacto social.


\begin{tcolorbox}
    El establecimiento de sociedades siempre va a conllevar a futuros conflictos, por lo que se necesita de un ente regulador que establezca normas y sanciones para mantener el orden.
\end{tcolorbox}

\section{Normas de conducta}
\begin{itemize}
    \item \textbf{Normas de trato social}
    \begin{itemize}
        \item Reglas de cortesía, respeto, decoro y educación.
    \end{itemize}
    
    \item \textbf{Normas morales}
    \begin{itemize}
        \item Reglas de conducta que se establecen en la conciencia del individuo.
    \end{itemize}
    
    \item \textbf{Normas religiosas}
    
    \item \textbf{Normas jurídicas}
    \begin{itemize}
        \item \textbf{Características:}
        \begin{itemize}
            \item Exteriores: Se manifiestan en la conducta de las personas.
            \item Coercibles: Se pueden hacer cumplir por la fuerza.
            \item Heterónomas: Son impuestas por un ente superior.
            \item Bilaterales: Establecen derechos y obligaciones.
        \end{itemize}
    \end{itemize}
    
    \item \textbf{Reglas técnicas}
    \begin{itemize}
        \item \textcolor{red}{reglas técnicas $\neq$ normas de conducta}
    \end{itemize}
\end{itemize}

\end{document}

