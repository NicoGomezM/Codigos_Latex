\documentclass{templateNote}
\usepackage{tcolorbox}
\usepackage{hyperref}
\usepackage{amsmath}
\usepackage{amssymb}
\usepackage{soul}
\usepackage{circuitikz}
\usepackage{enumitem}


\begin{document}
\imagenlogoU{img/logoNGMFormal_sinF.png}
\linklogoU{https://github.com/NicoGomezM} 
% \imagenlogoD{img/logo-ubb-txt-face.png} 
\titulo{Apunte 1}
\asignatura{Formulación y Evaluación de Proyectos}
\autor{
    \indent
    Nicolás {Gómez Morgado}
}


\portada
\margenes 
\tableofcontents
\newpage

\section{Importante}

\begin{itemize}
    \item Es necesario saber calcular porcentajes y tasas de interés, asi como también saber calcular el valor presente y futuro de una inversión.
    \item Es necesario saber calcular el Reajuste dados los valores que se nos entregan.
\end{itemize}

\subsection{Conceptos clave}

\begin{itemize}
    \item \textbf{Proyectos:} Conjunto de actividades interrelacionadas que se realizan para alcanzar un objetivo.
    \item \textbf{Inversión:} Es el desembolso de recursos financieros para la adquisición de bienes y servicios.
    \item \textbf{Rentabilidad:} Es la relación entre los beneficios obtenidos y los recursos invertidos.
\end{itemize}

\newpage
\section{Criterios para evaluar inversiones}
\subsection*{Método de evaluación inicial}
\begin{enumerate}[label=\alph*)]
    \item Incluir \textbf{todos los flujos de caja} que ocurren durante la vida del proyecto.
    \item Considerar el \textbf{valor del dinero} en el tiempo. Esto se refiere a que el dinero hoy vale más que el dinero en el futuro.
    \item Incorporar la \textbf{tasa de retorno requerida} en el proyecto.
\end{enumerate}

\end{document}
