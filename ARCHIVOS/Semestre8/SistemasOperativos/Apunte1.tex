\documentclass{templateNote}
\usepackage{tcolorbox}
\usepackage{hyperref}
\usepackage{amsmath}
\usepackage{amssymb}
\usepackage{soul}
\usepackage{circuitikz}
\usepackage{comment}
\usepackage{verbatim}


\begin{document}
\imagenlogoU{img/logoNGMFormal_sinF.png}
\linklogoU{https://github.com/NicoGomezM} 
% \imagenlogoD{img/logo-ubb-txt-face.png} 
\titulo{Apunte 1}
\asignatura{Legislación}
\autor{
    \indent
    Nicolás {Gómez Morgado}
}


\portada
\margenes 
\tableofcontents
\newpage


\section{Importante}

\begin{itemize}
    \item Se sugiere tener instalada una distribución de Linux en el computador, en mayor medida Debian, ya que el Profesor trabaja con esta distribución.
    \item Linux y Unix están orientados al multiuser, a diferencia de Windows que esta orientado al single user.
    \item El S.O. es el intermediario entre el hardware y el usuario.
    \item Linux es el núcleo del S.O. y las distribuciones son como se adorna/agrupa este mismo.
    \item Para la tarea 1 se debe crear un programa con GNU GCC++ en Linux.
    \item Para la tarea 2 se debe crear un programa.
    \item \textbf{API:} Interfaz de programación de aplicaciones que permite a los programas comunicarse con el sistema operativo.
\end{itemize}

\newpage
\section{Introducción}
En principios de la computación no existía el software que comunicara el hardware con el usuario, por lo que se debía programar directa y físicamente en lenguaje maquina (tarjetas perforadas). La operación de estas computadoras era completamente manual.Tiempo de uso de la computadora demasiado alto, involucrando colocar tarjetas, imprimir respuestas, etc.

Con el avance de la tecnología se separo el trabajo para operar la maquina, el operador se encargaba de insertar las tarjetas y el programador a crearlas/ordenarlas. A pesar de esta optimización la CPU seguía sin utilizarse cuando el trabajo se detenía, por lo cual se desarrollan mecanismos de Secuenciamiento automático de trabajos (primeros SO).

Se implementa un monitor residente para transferir el control al trabajo siguiente. A pesar de estas mejoras aun existía tiempo muerto en la CPU debido a la diferencia de velocidades en los dispositivos de entrada/salida.

\subsection*{E/S solapada}

Solución para reducir el tiempo muerto en la CPU, se implementa un mecanismo de E/S solapada, donde la CPU puede ejecutar otro trabajo mientras se realiza una operación de E/S ya que se guarda la información (instrucciones de las tarjetas) en las cintas magnéticas para procesarse posteriormente.
\\\textbf{Spooling:} Sistema de colas de trabajos, donde se almacenan los trabajos en una cola de entrada y se envían a la cola de salida para ser procesados por la CPU.

\subsection*{Que es un SO?}
Programa que actúa como intermediario entre el usuario y el hardware de la maquina cuyo propósito es proporcionar un entorno en el cual se puedan ejecutar programas de forma eficiente y controlada.


\subsection*{Estructura de un Sistema informático}
\noindent El SO es parte del sistema general.
\begin{itemize}
    \item Hardware: Conjunto de componentes físicos que componen la computadora.
    \item Sistema Operativo
    \item Programas de aplicación
    \item Usuario: Persona, maquinas u otras computadoras que utiliza la computadora.
\end{itemize}

En ningún caso las aplicaciones que utiliza el usuario pasan directamente al hardware, por temas de seguridad y optimización. 
\\Actualmente cualquier cosa que se haga en una computadoras se hace a trave del sistema operativo.

\subsection*{Papel de un SO}
\begin{itemize}
    \item Punto de vista del usuario
    \begin{itemize}
        \item SO diseñado para maximizar trabajo
        \item SO diseñado para maximizar recursos
    \end{itemize}
    \item Punto de vista del sistema
    \begin{itemize}
        \item SO es un asignador de recursos
        \item SO es un Programa de control
    \end{itemize}
\end{itemize}

\noindent\textbf{Interrupción:} Forma que tiene el SO para enterarse de que algo ha pasado en el sistema y como reaccionar ante esto.

\subsection*{Iniciando una Computadora}
\noindent\textbf{Bootstrap:} Programa de arranque que se encuentra en la memoria ROM de la computadora, este programa se encarga de cargar el SO en la memoria RAM.
El programa de arranque se localiza y se carga el kernel a través del \textbf{GRUB} (Grand Unified Boatload) que es un gestor de arranque que permite elegir entre varios sistemas operativos.
\\\\
\noindent\textbf{Llamada a sistema:} Puente entre las aplicaciones y el SO, permite a las aplicaciones solicitar servicios al SO.
\\\\
\noindent\textbf{Controladores de dispositivos:} El hardware sabe como comunicarse con estos dispositivos de e/s gracias a estos controladores.
\\El sistema guarda los tipos de interrupciones y sus significados por lo que esa es su forma de saber como reaccionar ante estas mismas.
\\\\
\noindent\textbf{DMA:} Permite a los dispositivos de e/s acceder a la memoria sin pasar por la CPU.

\subsection*{Almacenamiento}

\begin{itemize}
    \item HDD
    \item NVM
    \begin{itemize}
        \item SSD
    \end{itemize}
\end{itemize}

\subsection*{Caching}
Información es copiada desde almacenamientos lentos a almacenamientos rápidos para mejorar la velocidad de acceso a la información.

\subsection*{Sistema multiprocessor}
Dos o mas procesadores que comparten el bus del computador, estos procesadores pueden compartir la memoria y los dispositivos de e/s.

\subsection*{Sistemas NUMA (Non-Uniform Memory Access)}
Division no uniforme de acceso a la memoria

\subsection*{Cluster}
Varios nodos donde cada uno puede tener uno o varios procesadores.
\begin{itemize}
    \item Asimétrico: Sistema en modo de espera y monitorea a la activa.
    \item Simétrico: Todos los sistemas activos y se monitorear entre significados
\end{itemize}

\subsection*{Multiprogramación}
Organiza los trabajos para que la CPU no este ociosa, se ejecutan varios trabajos a la vez. CUando el trabajo en ejecución tiene que esperar el sistema cambia a otro trabajo.

\subsection*{Tiempo compartido}
Asigna tiempos (menos de 1 seg) entre tareas de manera equitativa, conmutando entre programas.
\\
\noindent\textbf{Swapping:} Mecanismo que permite mover procesos entre la memoria principal y la secundaria.


\end{document}
